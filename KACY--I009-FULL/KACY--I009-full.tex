\documentclass[12pt,a4paper]{memoir}
\maxtocdepth{subsubsection}
\setsecnumdepth{subsubsection}
%%% Wallpaper & quotes
\usepackage{wallpaper}
\usepackage{csquotes}
\usepackage{graphicx}
\usepackage{color}
\usepackage{amsmath}
\usepackage{systeme}
\usepackage{verbatim}
\usepackage{fancyhdr}
%\usepackage{esvect}
\usepackage{amssymb, amsmath}
\usepackage{subcaption}
\usepackage{amsmath}
\usepackage{amsthm}
\usepackage{amsfonts}
\usepackage{hyperref}
\usepackage{enumerate}
\usepackage{amssymb}
\usepackage{enumitem}
\usepackage{mathdots}
\usepackage{tikz}
\usepackage{graphicx}
\usepackage{float}
\usepackage{cancel}

\usepackage{tikz}
% \usepackage{forest}
% \usetikzlibrary{arrows.meta}
\usepackage{graphicx}
\usepackage{float}
\usepackage{multicol}
\usetikzlibrary{decorations.fractals}

\DeclareUnicodeCharacter{670}{}


\usepackage{systeme}
\usepackage{geometry}
\geometry{ margin=1in}
\def\C{\mathbb{C}}
\def\N{\mathbb{N}}
\def\Z{\mathbb{Z}}
\def\Q{\mathbb{Q}}
\def\R{\mathbb{R}}
\def\K{\mathbb{K}}
\def\F{\mathbb{F}}
\def\U{\mathcal{U}}
\def\P{\mathcal{P}}
\usepackage{lastpage}
\theoremstyle{definition}
%\newtheorem{question}{Question}
\newtheorem*{definition}{Definition}
%\newtheorem*{solution}{Solution}
\newtheorem{answer}{Answer}
\newtheorem{lemma}{Lemma}
\newtheorem{proposition}{Proposition}
\newtheorem{google}{Google}
\newtheorem*{claim}{Claim}
\newtheorem*{hint}{Hint}
\newtheorem{theorem}{Theorem}
\newtheorem{tool}{Tool}
%\newtheorem{example}{Example}
\newtheorem{corollary}{Corollary}
\newtheorem{identity}{Identity}

\newcommand{\red}{\texr{red}}
\newcommand{\abs}[1]{\lvert#1\rvert}
\newcommand{\norm}[1]{\left\lVert#1\right\rVert}
\setlength{\headheight}{15pt}

%% For copyleft symbol
\usepackage{textcomp}

%%% Boxes around text
%\usepackage{tcolorbox}
%%%% ExSheets package

\usepackage{enumerate}
\usepackage[auto-label]{exsheets}
%\usepackage{paralist}
\newcommand{\correct}{
	\PrintSolutionsTF{%
		\ref{ans:\CurrentQuestionID}%
	}{%
		\label{ans:\CurrentQuestionID}%
	}%
}
\SetupExSheets{
	headings = runin,
	skip-below = 0.4em,
	subtitle-format = {\bfseries}, % default is \itshape
}


\usepackage[skins]{tcolorbox}

\tcbset{
	% Defaults
	my box/main style/.style={},
	my box/title style/.style={},
	% Use the 'append' variants if you want to add to the defaults instead of
	% overriding them.
	my box/main/.style={/tcb/my box/main style/.style={#1}},
	my box/title/.style={/tcb/my box/title style/.style={#1}},
	my box/append main/.style={/tcb/my box/main style/.append style={#1}},
	my box/append title/.style={/tcb/my box/title style/.append style={#1}},
	%
	my box/.style={
		my box/.cd, #1,
		/tcb/.cd,
		enhanced,
		my box/main style,
		attach boxed title to top left={xshift=0.2cm, yshift=-0.2cm},
		boxed title style={
			outer arc=0pt,
			arc=0pt,
			top=3pt,
			bottom=3pt,
			my box/title style,
		},
	},
}

\newtcolorbox{idea}[1][]{
	my box={
		main={colframe=blue, colback=gray!50!blue!10},
		title={colback=blue!60, colframe=gray},
	},
	title={KACY Summer League},
	#1,
}


\SetupExSheets[question]{
	pre-hook = \begin{idea},
		post-hook = \end{idea}}

\SetupExSheets[solution]{print=false, name=Solution}
\SetupExSheets[question]{type=exam, name=KACY--I}

\NewQuSolPair{example}[name=Example]{exsol}
%\SetupExSheets{counter-format=se.qu[1],counter-within=part}

\SetupExSheets{totoc=true}

%%%% 3D TikZ Settings


%%% Lines of Latitude in Globe
\usetikzlibrary{calc,fadings,decorations.pathreplacing}
\newcommand\pgfmathsinandcos[3]{%
	\pgfmathsetmacro#1{sin(#3)}%
	\pgfmathsetmacro#2{cos(#3)}%
}
\newcommand\LongitudePlane[3][current plane]{%
	\pgfmathsinandcos\sinEl\cosEl{#2} % elevation
	\pgfmathsinandcos\sint\cost{#3} % azimuth
	\tikzset{#1/.style={cm={\cost,\sint*\sinEl,0,\cosEl,(0,0)}}}
}

\newcommand\LatitudePlane[3][current plane]{%
	\pgfmathsinandcos\sinEl\cosEl{#2} % elevation
	\pgfmathsinandcos\sint\cost{#3} % latitude
	\pgfmathsetmacro\yshift{\cosEl*\sint}
	\tikzset{#1/.style={cm={\cost,0,0,\cost*\sinEl,(0,\yshift)}}} %
}
\newcommand\NewLatitudePlane[4][current plane]{%
	\pgfmathsinandcos\sinEl\cosEl{#3} % elevation
	\pgfmathsinandcos\sint\cost{#4} % latitude
	\pgfmathsetmacro\yshift{#2*\cosEl*\sint}
	\tikzset{#1/.style={cm={\cost,0,0,\cost*\sinEl,(0,\yshift)}}} %
}
\newcommand\DrawLongitudeCircle[2][1]{
	\LongitudePlane{\angEl}{#2}
	\tikzset{current plane/.prefix style={scale=#1}}
	% angle of "visibility"
	\pgfmathsetmacro\angVis{atan(sin(#2)*cos(\angEl)/sin(\angEl))} %
	\draw[current plane] (\angVis:1) arc (\angVis:\angVis+180:1);
	\draw[current plane,dashed] (\angVis-180:1) arc (\angVis-180:\angVis:1);
}
\newcommand\DrawLatitudeCircle[2][1]{
	\LatitudePlane{\angEl}{#2}
	\tikzset{current plane/.prefix style={scale=#1}}
	\pgfmathsetmacro\sinVis{sin(#2)/cos(#2)*sin(\angEl)/cos(\angEl)}
	% angle of "visibility"
	\pgfmathsetmacro\angVis{asin(min(1,max(\sinVis,-1)))}
	\draw[current plane] (\angVis:1) arc (\angVis:-\angVis-180:1);
	\draw[current plane,dashed] (180-\angVis:1) arc (180-\angVis:\angVis:1);
}


%% document-wide tikz options and styles

\tikzset{%
	>=latex, % option for nice arrows
	inner sep=0pt,%
	outer sep=2pt,%
	mark coordinate/.style={inner sep=0pt,outer sep=0pt,minimum size=3pt,
		fill=black,circle}%
}

%%% Great circles
% Style to set camera angle, like PGFPlots `view` style
\tikzset{viewport/.style 2 args={
		x={({cos(-#1)*1cm},{sin(-#1)*sin(#2)*1cm})},
		y={({-sin(-#1)*1cm},{cos(-#1)*sin(#2)*1cm})},
		z={(0,{cos(#2)*1cm})}
}}

% Convert from spherical to cartesian coordinates
\newcommand{\ToXYZ}[2]{
	{sin(#1)*cos(#2)}, % X coordinate
	{cos(#1)*cos(#2)}, % Y coordinate
	{sin(#2)}          % Z (vertical) coordinate
}

%%% Polar Triangles

\usepackage    {tikz}
\usetikzlibrary{3d}
\usetikzlibrary{calc}
\usetikzlibrary{math}
\usetikzlibrary{angles,quotes} % for pic (angle labels)
\usepackage{tikz-3dplot}

% isometric axes
\pgfmathsetmacro\xx{1/sqrt(2)}
\pgfmathsetmacro\xy{1/sqrt(6)}
\pgfmathsetmacro\zy{sqrt(2/3)}

% some functions (cross products)
\tikzmath%
{%
	function crossx(\mx,\my,\mz,\nx,\ny,\nz)
	{% cross product, x coordinate, normailized
		\pxx = \my*\nz-\mz*\ny;
		\pyy = \mz*\nx-\mx*\nz;
		\pzz = \mx*\ny-\my*\nx;
		return {\pxx/sqrt(\pxx*\pxx+\pyy*\pyy+\pzz*\pzz)};
	};
	function crossy(\mx,\my,\mz,\nx,\ny,\nz)
	{% cross product, y coordinate, normailized
		\pxx = \my*\nz-\mz*\ny;
		\pyy = \mz*\nx-\mx*\nz;
		\pzz = \mx*\ny-\my*\nx;
		return {\pyy/sqrt(\pxx*\pxx+\pyy*\pyy+\pzz*\pzz)};
	};
	function crossz(\mx,\my,\mz,\nx,\ny,\nz)
	{% cross product, z coordinate, normailized
		\pxx = \my*\nz-\mz*\ny;
		\pyy = \mz*\nx-\mx*\nz;
		\pzz = \mx*\ny-\my*\nx;
		return {\pzz/sqrt(\pxx*\pxx+\pyy*\pyy+\pzz*\pzz)};
	};
}

\newcommand{\greatcircle}[6] % pole x, y, z, color, two orientation factors (+1/-1)
{%
	\coordinate (P)  at (#1,#2,#3);                  % pole
	\coordinate (N)  at ($(0,0,0)!#6*1.25cm!(P)$);   % these points are
	\coordinate (S)  at ($-1*(N)$);                  % used to clip the
	\coordinate (E)  at ($(0,0,0)!-1.25cm!270:(P)$); % ellipses
	\coordinate (W)  at ($-1*(E)$);                  % ...
	\coordinate (NW) at ($(N)+(W)$);
	\coordinate (NE) at ($(N)+(E)$);
	\coordinate (SW) at ($(S)+(W)$);
	\coordinate (SE) at ($(S)+(E)$);
	\pgfmathsetmacro\ptheta{atan(#2/#1)} % pole, spherical coordinate theta
	\pgfmathsetmacro\pphi  {#5*acos(#3)} % pole, spherical coordinate phi
	\begin{scope}
		% \clip (W) -- (SW) -- (SE) -- (E) -- cycle;
		\draw[rotate around z=\ptheta,rotate around y=\pphi,%
		canvas is xy plane at z=0,#4] (0,0) circle (1);
	\end{scope}
	\begin{scope}
		% \clip (W) -- (NW) -- (NE) -- (E) -- cycle;
		\draw[rotate around z=\ptheta,rotate around y=\pphi,%
		canvas is xy plane at z=0,#4,densely dotted] (0,0) circle (1);
	\end{scope}
}

\usepackage{pgfplots}
\usetikzlibrary{calc,3d,intersections, positioning,intersections,shapes}
\pgfplotsset{compat=1.11} 


\newcommand{\InterSec}[3]{%
	\path[name intersections={of=#1 and #2, by=#3, sort by=#1,total=\t}]
	\pgfextra{\xdef\InterNb{\t}}; }




\begin{document}
	\begin{titlingpage}
		% \thispagestyle{Rplain}
		%\ThisCenterWallPaper{1.2}{Ganesha}
		\author{\scalebox{2}{\fontsize{15pt}{0pt}\selectfont \textbf{\sffamily \color{red} Amir \color{red} Parvardi}}}
		% \title{ \color{white} THE OLYMPIAD ALGEBRA BOOK (VOL I): \\ \color{white} 1220 Polynomials and Trigonometry Problems}
		\title{\scalebox{2}{\fontsize{12pt}{0pt}\selectfont \textbf{\sffamily \color{teal} Summer Kaywañan  \color{black} Algebra Competitions}} \\ \scalebox{2}{\fontsize{13pt}{0pt}\selectfont \textbf{\scshape \color{teal} A.K.A.}} \\  \scalebox{2}{\fontsize{10pt}{0pt}\selectfont \textbf{\sffamily \color{teal} Summer  KACY}}\\ \color{lime}\texttt{KACY--I009}:\\ \color{teal}\textbf{Olympiad Pre-Algebra Contest 009}}
		\date{\color{teal} \scshape August 5, 2023}
		\maketitle
	\end{titlingpage}
	
%	\frontmatter
%	
%	
%	\chapter*{Preface}
%	\addcontentsline{toc}{chapter}{Preface}
%	\section*{Foreword}
%	\Large
%	\textbf{Azermalohg} speaks to \textbf{Rima}:
%	\begin{displayquote}
%		Why are you so afraid of the IllLyrans?\\
%		What has your fear had you achieved?\\
%		You have made yourself weary for lack of sleep,\\
%		You only fill your flesh with grief,\\
%		You only bring the distant days closer.\\
%		Humankind's fame is cut down\\
%		like reeds in a reed-bed.\\
%		A fine young man, a fine young girl...\\
%		at grip of Death.\\
%		You have seen Death,\\
%		You have touched the face of Death,\\
%		You hear the voice of Death lamenting in your ears,\\
%		Savage Death just cuts humankind down.\\
%		Sometimes we have hope,\\
%		sometimes we make a wish,\\
%		but then our airplanes are shot in the air.\\
%		Sometimes there is hostility in the land,\\
%		but in the end, only the most benevolent will remain.\\
%		The ruthless IllLyrans bring Death with themselves;\\
%		but the merciful Lyrans will always prevail.\\
%		Remember, the night is darkest just before dawn.
%	\end{displayquote}
%	
%	
%	\newpage
	
	\section*{Synopsis}
		The Olympiad Algebra Book comes in two volumes. The first volume, dedicated to \href{https://www.academia.edu/101938068/The_Olympiad_Algebra_Book_Vol_I_1220_Polynomials_and_Trigonometry_Problems}{Polynomials and Trigonometry}, is a collection of lesson plans containing $1220$ beautiful problems, around two-thirds of which are polynomial problems and one-third are trigonometry problems. The second volume of The Olympiad Algebra Book contains $1220$ Problems on Functional Equations and Inequalities, and I hope to finish it before the end of Summer 2023. The current volumes has $843$ Polynomial problems and $377$ Trigonometry questions, the last $63$ of which are bizarre spherical geometry problems! I also added $407$ complementary review problems to the first volume on July $16^{th}$, 2023.
		
		
		\vspace{0.5em}
		
		The Olympiad Algebra Book is supposed to be a problem bank for Algebra, and it forms the resource for the first series of the KAYWAÑAN Algebra Contest. I suggest you start with Polynomials, and before you get bored or exhausted, also start solving Trigonometry problems. If you find these problems easy and not challenging enough, the Spherical Trigonometry lessons and problems are definitely going to be a must try!
		
		
		\vspace{0.5em}
		
		This booklet contains problems and solutions of $\texttt{KACY--I009}$ (Olympiad Pre--Algebra Contests), including the problems from the first book: $$\text{KACY--I}\left\{33,55,57,72,73,85,101,102,112,121\right\}.$$ The numbers referred here are the question number out of the 1220 questions labeled from \href{https://github.com/parvardi/KACY/blob/main/KACY-VOL-I.pdf}{1 to 1220}. The competition's full title is ``Kaywañan Olympiad Pre--Algebra Summer Contest 009,'' held on Saturday August $5^{th}$, 2023.
%	
%		\noindent ``Kaywañan Olympiad Pre--Algebra Summer Contest 001''
%		\begin{tasks}
%			\task $\texttt{KACY--I001}$: $\text{KACY--I}\left\{2,37,38,58,74,75,86,103,113\right\}$.
%%			\task $\texttt{KACY--I002}$: $\text{KACY--I}\left\{3,39,40,59,76,77,87,104,114\right\}$.
%%			\task $\texttt{KACY--I003}$: $\text{KACY--I}\left\{4,41,42,60,61,78,88,105,115\right\}$.
%%			\task $\texttt{KACY--I004}$: $\text{KACY--I}\left\{5,43,44,62,79,89,90,106,116\right\}$.
%%			\task $\texttt{KACY--I005}$: $\text{KACY--I}\left\{6,45,46,63,80,91,92,107,117\right\}$.
%%			\task $\texttt{KACY--I006}$: $\text{KACY--I}\left\{13,47,64,81,93,95,108,109,118\right\}$.
%%			\task $\texttt{KACY--I007}$: $\text{KACY--I}\left\{14,48,65,66,82,96,97,98,110,119\right\}$.
%%			\task $\texttt{KACY--I008}$: $\text{KACY--I}\left\{32,49,67,68,83,84,99,100,111,120\right\}$.
%%			\task $\texttt{KACY--I009}$: $\text{KACY--I}\left\{33,55,57,72,73,85,101,102,112,121\right\}$.
%		\end{tasks}  
%		Dates of \texttt{KACY--I001} to \texttt{KACY--I009}:
%		\begin{enumerate}
%			\item \texttt{KACY--I001}: June 3, 2023.
%%			\item \texttt{KACY--I002}: June 10, 2023.
%%			\item \texttt{KACY--I003}: June 17, 2023.
%%			\item \texttt{KACY--I004}: June 24, 2023.
%%			\item \texttt{KACY--I005}: July 1, 2023.
%%			\item \texttt{KACY--I006}: July 8. 2023.
%%			\item \texttt{KACY--I007}: July 15, 2023.
%%			\item \texttt{KACY--I008}: July 22, 2023.
%%			\item \texttt{KACY--I009}: July 29, 2023.
%		\end{enumerate}  
\Large
		\begin{flushright}
			Amir Parvardi,\\
			Vancouver, British Columbia,\\
			August 5, 2023
		\end{flushright}
	
	\newpage
	\normalsize
	\tableofcontents\label{TOC}
	%\listoffigures
	% \newpage
	% \listoftables
	% \cleardoublepage
	% \addcontentsline{toc}{chapter}{Index}
	% \printindex
	
%	\mainmatter
	\normalsize
	\pagestyle{fancy}
	\fancyhf{}
	% \fancyhead[LE]{\nouppercase{\rightmark\hfill\leftmark}}
	% \fancyhead[RO]{\nouppercase{\leftmark\hfill\rightmark}}
	\fancyfoot[LE,RO]{Amir Parvardi\hfill\thepage/\pageref{LastPage}\hfill KAYWAÑAN\textsuperscript{\textcopyleft}}
	\fancyhead[LE,RO]{Olympiad Algebra (Vol. I):\hfill 1220 Problems\hfill \texttt{KACY--I009} Booklet}

	\newpage

	\begin{tcolorbox}
		\begin{displayquote}
			``Let No One Ignorant of Algebra Enter!''
			\begin{flushright}
				\LARGE \textsc{Kaywan}
			\end{flushright}
		\end{displayquote}
	\end{tcolorbox}
	
\vspace{1em}
	
	The rules of the KACY Competitions are simple: 
	\begin{idea}
		\begin{tasks}
			\task All problems whose titles contain \textbf{KACY--I} are questions of the Summer KACY Series, and all problems with a title containing \textbf{KACY--II} are questions of the Winter KACY Series.
			\task This is the first volume of KACY, and it contains the SUMMER KACY questions. For the SUMMER KACY 2023 held weekly in Summer and Fall of 2023, only questions with title containing ``\textbf{KACY--I}'' are to be used in the actual KAYWAÑAN competitions.
		\end{tasks}
	\end{idea}
	
	\vspace{0.5em}

	This is because all the questions whose source does not contain \textbf{KACY--I} are either from a legit mathematical competition such as IMO, IMO Shortlist/Longlist, MAA Series (AMC, AIME, USAMO, USATST, USATSTST, USAMTS, etc.), National or Regional Olympiads (USA, APMC, Canada, etc.), or maybe from a book/paper I found and referenced in the question's title. 
	
	\vspace{0.5em}
	
	This assures that no famous problems are used in KACY, and that we actually identify and solve the non--KACY problems as exercises and examples in our journey of learning algebra during KAYWAÑAN Algebra Contest.
	
	\newpage
	
	\section*{KACY--I009 Problems}
	%\text{KACY--I}\left\{33,55,57,72,73,85,101,102,112,121\right\}
	\setcounter{question}{32}
	
	\begin{question}
		% https://artofproblemsolving.com/community/c4h14275p100982
		How many numbers in the $100^{th}$ row of the Pascal triangle (the one starting with $1, 100, \dots$) are not divisible by $3$?
	\end{question}
	
	\begin{solution}[name=Solution by \href{https://artofproblemsolving.com/community/c4h14275p101907}{Boris}]
		% https://artofproblemsolving.com/community/c4h14275p101907
		The answer is $12$. To find the numbers in the $100^{th}$ row of the Pascal triangle (the one starting with $1, 100, \dots$) that are not divisible by $3$, we need to find the number of coefficients in the polynomial
		\[ (1+x)^{100}= 1 + {100\choose 1}x + {100\choose 2}x^2 + \dots + x^{100}, \]
		which are not equal to $0$ modulo $3$. Note that by Binomial Theorem, and taking modulo $3$, one has,
		\[(1+x)^3= 1+ 3x+3x^2+ x^3 \equiv 1+x^3 \pmod 3.\]
		and so also
		\[(1+x)^9 \equiv (1+x^3)^3 \equiv 1+x^9 \pmod 3,\]
		and so on, for any power of $3$. Now, $100= 81+2\cdot 9 + 1$. Therefore, modulo $3$ one has
		\[ (1+x)^{100}= (1+x)^{81} \left((1+x)^9\right)^2 (1+x) =
		(1+x^{81}) (1+ 2x^{9} + x^{18}) (1+x). \]
		In this product all $2\cdot 3 \cdot 2=12 $ powers of $x$ are different (because every integer can be written in base $3$ in a unique way), and the coefficients are all nonzero modulo $3$. So, the answer is $12$.
	\end{solution}
	
	
	
	\setcounter{question}{54}
	
	
	
	\begin{tcolorbox}
		\SetupExSheets{headings=runin}
		\begin{question}%[name=(${2^k}^{th}$ Negative Double--Variable Identity)]
			Factorize $x^{2^k}-y^{2^k}$ for all $k$.
		\end{question}
	\end{tcolorbox}
	
	\begin{solution}[name=Solution by Amir Parvardi]
		Since $x=y$ yields $x^{2^k}-y^{2^k}=0$, we know that $(x-y)$ is a factor of $x^{2^k}-y^{2^k}$. We also get the quotient as in the $n^{th}$ Negative Double--Variable Identity:
		$$\frac{x^{2^k}-y^{2^k}}{x-y} = x^{2^k-1}+x^{2^k-2}y+\cdots+xy^{2^k-2}+y^{2^k-1}.$$
		We see that the degree of $x$ in the quotient is $2^k-1$, which happens to be equal to $1+2+2^2+\cdots+2^{k-1}$, meaning that the leading term in the quotient, $x^{2^k-1}$, is in fact a product of $k$ terms $x \cdot x^2 \cdot x^{2^2} \cdots x^{2^{k-1}}$, and there must be an identity in this form:
		$$ x^{2^k-1}+x^{2^k-2}y+\cdots+xy^{2^k-2}+y^{2^k-1} = (x+\dots)(x^2+\dots)(x^{2^2}+\dots)\cdots (x^{2^{k-1}}+\dots),$$
		and the same technique could be applied on $y$ since everything is symmetric, and the missing terms are easily found:
		$$\frac{x^{2^k}-y^{2^k}}{x-y} =  (x+y)(x^2+y^2)(x^{2^2}+y^{2^2})\cdots (x^{2^{k-1}}+x^{2^{k-1}}),$$
		giving us the magical ${2^k}^{th}$ Negative Double--Variable Identity:
		$$x^{2^k}-y^{2^k} = (x-y)(x+y)(x^2+y^2)(x^{2^2}+y^{2^2})\cdots (x^{2^{k-1}}+y^{2^{k-1}}).$$
	\end{solution}
	
		\setcounter{question}{56}
	
	\begin{tcolorbox}
		\SetupExSheets{headings=runin}
		\begin{question}[name=Sophie Parker Identity]
			Factorize $x^4+x^2y^2+y^4$.
		\end{question}
	\end{tcolorbox}
	
	\begin{solution}[name=Solution by Sophie Parker]
		
		\begin{itemize}
			\item \textbf{Difference of Squares}: It is easy to see that adding $x^2y^2$ to the given expression completes the square, making it
			$(x^2+y^2)^2 = x^4+2x^2y^2+y^4$. The Difference of Squares Identity yields the final factorization:
			\begin{align*}
				x^4+x^2y^2+y^4 &= (x^2+y^2)^2 - (xy)^2\\ &=(x^2+y^2-xy)(x^2+y^2+xy).
			\end{align*}
			
			\item \textbf{Difference of Squares and Cubes}: What if we begin with $x^6-y^6$? If I apply The Difference of Squares on this expression, I would have on one hand $x^6-y^6=(x^3-y^3)(x^3+y^3)$, and on the other hand I can apply the $n^{th}$ Negative Double--Variable Identity for $n=3$ on $x^6-y^6= (x^2-y^2)(x^4+x^2y^2+y^4)$.
			\begin{align*}
				x^6 - y^6 &= (x^3-y^3)(x^3+y^3)\\
				&= (x-y)(x^2+xy+y^2)\cdot (x+y)(x^2-xy+y^2)\\
				x^6 - y^6 &= (x^2-y^2)(x^4+x^2y^2+y^4)\\
				&= (x-y)(x+y)(x^4+x^2y^2+y^4).
			\end{align*}
			Therefore,
			\begin{align*}
				(x-y)(x^2+xy+y^2)\cdot (x+y)(x^2-xy+y^2) &= (x-y)(x+y)(x^4+x^2y^2+y^4).
			\end{align*}
			Assuming $x\neq \pm y$, we can cancel the terms $x-y$ and $x+y$ from both sides of the equation and obtain the factorization of $x^4+x^2y^2+y^4$ as a consequence:
			\begin{align*}
				(x^2+xy+y^2)\cdot(x^2-xy+y^2) &= x^4+x^2y^2+y^4.
			\end{align*}
		\end{itemize}
	\end{solution}
	
	
	

	
	
	
	
	
	
	\setcounter{question}{71}	
	
	
	
	\begin{tcolorbox}
		\SetupExSheets{headings=runin}
		\begin{question}
			Factorize $a^4+b^4+c^4 - 2a^2b^2 - 2a^2c^2 - 2b^2c^2$.
		\end{question}
	\end{tcolorbox}
	
	\begin{solution}%[name=Solution by Parviz Shahriari]
		Answer: $(a+b+c)(a-b-c)(a+b-c)(a-b+c)$.
	\end{solution}
	
	
	\begin{tcolorbox}
		\SetupExSheets{headings=runin}
		\begin{question}
			Factorize $x^2y^2z^2 + (x^2+yz)(y^2+zx)(z^2+xy)$.
		\end{question}
	\end{tcolorbox}
	
	\begin{solution}
		Answer: $(xy^2+yz^2+zx^2)(x^2y+y^2x+z^2x)$.
	\end{solution}
	
	\setcounter{question}{84}
	
	
	
	\begin{tcolorbox}
		\SetupExSheets{headings=runin}
		\begin{question}
			Factorize $(x+y+z)^5-x^5-y^5-z^5$.
		\end{question}
	\end{tcolorbox}
	
	\begin{solution}%[name=Solution by Parviz Shahriari]
		Answer: $5(x+y)(y+z)(z+x)(x^2+y^2+z^2+xy+yz+zx)$.
	\end{solution}
	
	
	
	\setcounter{question}{100}
	
	\begin{tcolorbox}
		\SetupExSheets{headings=runin}
		\begin{question}
			If $f(x)=ax^2+bx+c$, what is the value of the following expression?
			\begin{align*}
				g(x) = f(x+3) - 3f(x+2) + 3f(x+1) - f(x).
			\end{align*}
		\end{question}
	\end{tcolorbox}
	
	\begin{solution}%[name=Solution by Parviz Shahriari]
		Answer: $g(x) \equiv 0$.
	\end{solution}
	
	\begin{tcolorbox}
		\SetupExSheets{headings=runin}
		\begin{question}
			Find a function in the form of $f(x)=a+bc^x$ such that
			\begin{align*}
				f(0)=15, f(2)=30, f(4)=90.
			\end{align*}
		\end{question}
	\end{tcolorbox}
	
	\begin{solution}%[name=Solution by Parviz Shahriari]
		Answer: $f(x) = 10 + 5 \cdot 2^x$.
	\end{solution}
	
	\setcounter{question}{111}
	

	
	\begin{tcolorbox}
		\SetupExSheets{headings=runin}
		\begin{question}
			The function $f(x)$ is defined for $x>1$ as
			\begin{align*}
				f(x) = \log(x+\sqrt{x^2-1}).
			\end{align*}
			Find $f(2x^2-1)$ and $f(4x^3-3x)$ in terms of $f(x)$.
		\end{question}
	\end{tcolorbox}
	
	\begin{solution}%[name=Solution by Parviz Shahriari]
		Answer: $f(2x^2-1)=2f(x)$ and $f(4x^3-3x)=3f(x)$.
	\end{solution}
	

	
	
	
	\setcounter{question}{120}
	
	
	\begin{tcolorbox}
		\SetupExSheets{headings=runin}
		\begin{question}
			What is the sum of coefficients of the following polynomial after expansion?
			\begin{align*}
				p(x)=(12x^3-54x^2+19x+22)^{71}.
			\end{align*}
		\end{question}
	\end{tcolorbox}
	
	\begin{solution}%[name=Solution by Parviz Shahriari]
		Answer: $p(1)=-1$.
	\end{solution}
	
	

	\newpage
	\section*{KACY--I009 Answers}
	%	
	The last problems and solutions of KACY--I009 are courtesy of Parviz Shahriari, and they are taken from his eternal two-volume Farsi contribution to mathematics: ``Methods of Algebra.'' May he rest in peace!
	
	\vspace{1em}
	
	\printsolutions
\end{document}


