\documentclass[12pt,a4paper]{memoir}
\maxtocdepth{subsubsection}
\setsecnumdepth{subsubsection}
%%% Wallpaper & quotes
\usepackage{wallpaper}
\usepackage{csquotes}
\usepackage{graphicx}
\usepackage{color}
\usepackage{amsmath}
\usepackage{systeme}
\usepackage{verbatim}
\usepackage{fancyhdr}
%\usepackage{esvect}
\usepackage{amssymb, amsmath}
\usepackage{subcaption}
\usepackage{amsmath}
\usepackage{amsthm}
\usepackage{amsfonts}
\usepackage{hyperref}
\usepackage{enumerate}
\usepackage{amssymb}
\usepackage{enumitem}
\usepackage{mathdots}
\usepackage{tikz}
\usepackage{graphicx}
\usepackage{float}
\usepackage{cancel}

\usepackage{tikz}
% \usepackage{forest}
% \usetikzlibrary{arrows.meta}
\usepackage{graphicx}
\usepackage{float}
\usepackage{multicol}
\usetikzlibrary{decorations.fractals}

\DeclareUnicodeCharacter{670}{}


\usepackage{systeme}
\usepackage{geometry}
\geometry{ margin=1in}
\def\C{\mathbb{C}}
\def\N{\mathbb{N}}
\def\Z{\mathbb{Z}}
\def\Q{\mathbb{Q}}
\def\R{\mathbb{R}}
\def\K{\mathbb{K}}
\def\F{\mathbb{F}}
\def\U{\mathcal{U}}
\def\P{\mathcal{P}}
\usepackage{lastpage}
\theoremstyle{definition}
%\newtheorem{question}{Question}
\newtheorem*{definition}{Definition}
%\newtheorem*{solution}{Solution}
\newtheorem{answer}{Answer}
\newtheorem{lemma}{Lemma}
\newtheorem{proposition}{Proposition}
\newtheorem{google}{Google}
\newtheorem*{claim}{Claim}
\newtheorem*{hint}{Hint}
\newtheorem{theorem}{Theorem}
\newtheorem{tool}{Tool}
%\newtheorem{example}{Example}
\newtheorem{corollary}{Corollary}
\newtheorem{identity}{Identity}

\newcommand{\red}{\texr{red}}
\newcommand{\abs}[1]{\lvert#1\rvert}
\newcommand{\norm}[1]{\left\lVert#1\right\rVert}
\setlength{\headheight}{15pt}

%% For copyleft symbol
\usepackage{textcomp}

%%% Boxes around text
%\usepackage{tcolorbox}
%%%% ExSheets package

\usepackage{enumerate}
\usepackage[auto-label]{exsheets}
%\usepackage{paralist}
\newcommand{\correct}{
	\PrintSolutionsTF{%
		\ref{ans:\CurrentQuestionID}%
	}{%
		\label{ans:\CurrentQuestionID}%
	}%
}
\SetupExSheets{
	headings = runin,
	skip-below = 0.4em,
	subtitle-format = {\bfseries}, % default is \itshape
}


\usepackage[skins]{tcolorbox}

\tcbset{
	% Defaults
	my box/main style/.style={},
	my box/title style/.style={},
	% Use the 'append' variants if you want to add to the defaults instead of
	% overriding them.
	my box/main/.style={/tcb/my box/main style/.style={#1}},
	my box/title/.style={/tcb/my box/title style/.style={#1}},
	my box/append main/.style={/tcb/my box/main style/.append style={#1}},
	my box/append title/.style={/tcb/my box/title style/.append style={#1}},
	%
	my box/.style={
		my box/.cd, #1,
		/tcb/.cd,
		enhanced,
		my box/main style,
		attach boxed title to top left={xshift=0.2cm, yshift=-0.2cm},
		boxed title style={
			outer arc=0pt,
			arc=0pt,
			top=3pt,
			bottom=3pt,
			my box/title style,
		},
	},
}

\newtcolorbox{idea}[1][]{
	my box={
		main={colframe=blue, colback=gray!50!blue!10},
		title={colback=blue!60, colframe=gray},
	},
	title={KACY Summer League},
	#1,
}


\SetupExSheets[question]{
	pre-hook = \begin{idea},
		post-hook = \end{idea}}

\SetupExSheets[solution]{print=false, name=Solution}
\SetupExSheets[question]{type=exam, name=KACY--I}

\NewQuSolPair{example}[name=Example]{exsol}
%\SetupExSheets{counter-format=se.qu[1],counter-within=part}

\SetupExSheets{totoc=true}

%%%% 3D TikZ Settings


%%% Lines of Latitude in Globe
\usetikzlibrary{calc,fadings,decorations.pathreplacing}
\newcommand\pgfmathsinandcos[3]{%
	\pgfmathsetmacro#1{sin(#3)}%
	\pgfmathsetmacro#2{cos(#3)}%
}
\newcommand\LongitudePlane[3][current plane]{%
	\pgfmathsinandcos\sinEl\cosEl{#2} % elevation
	\pgfmathsinandcos\sint\cost{#3} % azimuth
	\tikzset{#1/.style={cm={\cost,\sint*\sinEl,0,\cosEl,(0,0)}}}
}

\newcommand\LatitudePlane[3][current plane]{%
	\pgfmathsinandcos\sinEl\cosEl{#2} % elevation
	\pgfmathsinandcos\sint\cost{#3} % latitude
	\pgfmathsetmacro\yshift{\cosEl*\sint}
	\tikzset{#1/.style={cm={\cost,0,0,\cost*\sinEl,(0,\yshift)}}} %
}
\newcommand\NewLatitudePlane[4][current plane]{%
	\pgfmathsinandcos\sinEl\cosEl{#3} % elevation
	\pgfmathsinandcos\sint\cost{#4} % latitude
	\pgfmathsetmacro\yshift{#2*\cosEl*\sint}
	\tikzset{#1/.style={cm={\cost,0,0,\cost*\sinEl,(0,\yshift)}}} %
}
\newcommand\DrawLongitudeCircle[2][1]{
	\LongitudePlane{\angEl}{#2}
	\tikzset{current plane/.prefix style={scale=#1}}
	% angle of "visibility"
	\pgfmathsetmacro\angVis{atan(sin(#2)*cos(\angEl)/sin(\angEl))} %
	\draw[current plane] (\angVis:1) arc (\angVis:\angVis+180:1);
	\draw[current plane,dashed] (\angVis-180:1) arc (\angVis-180:\angVis:1);
}
\newcommand\DrawLatitudeCircle[2][1]{
	\LatitudePlane{\angEl}{#2}
	\tikzset{current plane/.prefix style={scale=#1}}
	\pgfmathsetmacro\sinVis{sin(#2)/cos(#2)*sin(\angEl)/cos(\angEl)}
	% angle of "visibility"
	\pgfmathsetmacro\angVis{asin(min(1,max(\sinVis,-1)))}
	\draw[current plane] (\angVis:1) arc (\angVis:-\angVis-180:1);
	\draw[current plane,dashed] (180-\angVis:1) arc (180-\angVis:\angVis:1);
}


%% document-wide tikz options and styles

\tikzset{%
	>=latex, % option for nice arrows
	inner sep=0pt,%
	outer sep=2pt,%
	mark coordinate/.style={inner sep=0pt,outer sep=0pt,minimum size=3pt,
		fill=black,circle}%
}

%%% Great circles
% Style to set camera angle, like PGFPlots `view` style
\tikzset{viewport/.style 2 args={
		x={({cos(-#1)*1cm},{sin(-#1)*sin(#2)*1cm})},
		y={({-sin(-#1)*1cm},{cos(-#1)*sin(#2)*1cm})},
		z={(0,{cos(#2)*1cm})}
}}

% Convert from spherical to cartesian coordinates
\newcommand{\ToXYZ}[2]{
	{sin(#1)*cos(#2)}, % X coordinate
	{cos(#1)*cos(#2)}, % Y coordinate
	{sin(#2)}          % Z (vertical) coordinate
}

%%% Polar Triangles

\usepackage    {tikz}
\usetikzlibrary{3d}
\usetikzlibrary{calc}
\usetikzlibrary{math}
\usetikzlibrary{angles,quotes} % for pic (angle labels)
\usepackage{tikz-3dplot}

% isometric axes
\pgfmathsetmacro\xx{1/sqrt(2)}
\pgfmathsetmacro\xy{1/sqrt(6)}
\pgfmathsetmacro\zy{sqrt(2/3)}

% some functions (cross products)
\tikzmath%
{%
	function crossx(\mx,\my,\mz,\nx,\ny,\nz)
	{% cross product, x coordinate, normailized
		\pxx = \my*\nz-\mz*\ny;
		\pyy = \mz*\nx-\mx*\nz;
		\pzz = \mx*\ny-\my*\nx;
		return {\pxx/sqrt(\pxx*\pxx+\pyy*\pyy+\pzz*\pzz)};
	};
	function crossy(\mx,\my,\mz,\nx,\ny,\nz)
	{% cross product, y coordinate, normailized
		\pxx = \my*\nz-\mz*\ny;
		\pyy = \mz*\nx-\mx*\nz;
		\pzz = \mx*\ny-\my*\nx;
		return {\pyy/sqrt(\pxx*\pxx+\pyy*\pyy+\pzz*\pzz)};
	};
	function crossz(\mx,\my,\mz,\nx,\ny,\nz)
	{% cross product, z coordinate, normailized
		\pxx = \my*\nz-\mz*\ny;
		\pyy = \mz*\nx-\mx*\nz;
		\pzz = \mx*\ny-\my*\nx;
		return {\pzz/sqrt(\pxx*\pxx+\pyy*\pyy+\pzz*\pzz)};
	};
}

\newcommand{\greatcircle}[6] % pole x, y, z, color, two orientation factors (+1/-1)
{%
	\coordinate (P)  at (#1,#2,#3);                  % pole
	\coordinate (N)  at ($(0,0,0)!#6*1.25cm!(P)$);   % these points are
	\coordinate (S)  at ($-1*(N)$);                  % used to clip the
	\coordinate (E)  at ($(0,0,0)!-1.25cm!270:(P)$); % ellipses
	\coordinate (W)  at ($-1*(E)$);                  % ...
	\coordinate (NW) at ($(N)+(W)$);
	\coordinate (NE) at ($(N)+(E)$);
	\coordinate (SW) at ($(S)+(W)$);
	\coordinate (SE) at ($(S)+(E)$);
	\pgfmathsetmacro\ptheta{atan(#2/#1)} % pole, spherical coordinate theta
	\pgfmathsetmacro\pphi  {#5*acos(#3)} % pole, spherical coordinate phi
	\begin{scope}
		% \clip (W) -- (SW) -- (SE) -- (E) -- cycle;
		\draw[rotate around z=\ptheta,rotate around y=\pphi,%
		canvas is xy plane at z=0,#4] (0,0) circle (1);
	\end{scope}
	\begin{scope}
		% \clip (W) -- (NW) -- (NE) -- (E) -- cycle;
		\draw[rotate around z=\ptheta,rotate around y=\pphi,%
		canvas is xy plane at z=0,#4,densely dotted] (0,0) circle (1);
	\end{scope}
}

\usepackage{pgfplots}
\usetikzlibrary{calc,3d,intersections, positioning,intersections,shapes}
\pgfplotsset{compat=1.11} 


\newcommand{\InterSec}[3]{%
	\path[name intersections={of=#1 and #2, by=#3, sort by=#1,total=\t}]
	\pgfextra{\xdef\InterNb{\t}}; }




\begin{document}
	\begin{titlingpage}
		% \thispagestyle{Rplain}
		%\ThisCenterWallPaper{1.2}{Ganesha}
		\author{\scalebox{2}{\fontsize{15pt}{0pt}\selectfont \textbf{\sffamily \color{red} Amir \color{red} Parvardi}}}
		% \title{ \color{white} THE OLYMPIAD ALGEBRA BOOK (VOL I): \\ \color{white} 1220 Polynomials and Trigonometry Problems}
		\title{\scalebox{2}{\fontsize{12pt}{0pt}\selectfont \textbf{\sffamily \color{teal} Summer Kaywañan  \color{black} Algebra Competitions}} \\ \scalebox{2}{\fontsize{13pt}{0pt}\selectfont \textbf{\scshape \color{teal} A.K.A.}} \\  \scalebox{2}{\fontsize{10pt}{0pt}\selectfont \textbf{\sffamily \color{teal} Summer  KACY}}\\ \color{lime}\texttt{KACY--I005}:\\ \color{teal}\textbf{Olympiad Pre-Algebra Contest 005}}
		\date{\color{teal} \scshape July 8, 2023}
		\maketitle
	\end{titlingpage}
	
%	\frontmatter
%	
%	
%	\chapter*{Preface}
%	\addcontentsline{toc}{chapter}{Preface}
%	\section*{Foreword}
%	\Large
%	\textbf{Azermalohg} speaks to \textbf{Rima}:
%	\begin{displayquote}
%		Why are you so afraid of the IllLyrans?\\
%		What has your fear had you achieved?\\
%		You have made yourself weary for lack of sleep,\\
%		You only fill your flesh with grief,\\
%		You only bring the distant days closer.\\
%		Humankind's fame is cut down\\
%		like reeds in a reed-bed.\\
%		A fine young man, a fine young girl...\\
%		at grip of Death.\\
%		You have seen Death,\\
%		You have touched the face of Death,\\
%		You hear the voice of Death lamenting in your ears,\\
%		Savage Death just cuts humankind down.\\
%		Sometimes we have hope,\\
%		sometimes we make a wish,\\
%		but then our airplanes are shot in the air.\\
%		Sometimes there is hostility in the land,\\
%		but in the end, only the most benevolent will remain.\\
%		The ruthless IllLyrans bring Death with themselves;\\
%		but the merciful Lyrans will always prevail.\\
%		Remember, the night is darkest just before dawn.
%	\end{displayquote}
%	
%	
%	\newpage
	
	\section*{Synopsis}
		The Olympiad Algebra Book comes in two volumes. The first volume, dedicated to \href{https://www.academia.edu/101938068/The_Olympiad_Algebra_Book_Vol_I_1220_Polynomials_and_Trigonometry_Problems}{Polynomials and Trigonometry}, is a collection of lesson plans containing $1220$ beautiful problems, around two-thirds of which are polynomial problems and one-third are trigonometry problems. The second volume of The Olympiad Algebra Book contains $1220$ Problems on Functional Equations and Inequalities, and I hope to finish it before the end of Summer 2023. I hope I can finish collecting the FE and INEQ problems by June $29^{th}$, as a reminder of the \href{https://www.academia.edu/29934442/1220_Number_Theory_Problems_J29_Project}{$1220$ Number Theory Problems} published as the first 1220 set of J29 Project. The current volumes has $843$ Polynomial problems and $377$ Trigonometry questions, the last $63$ of which are bizarre spherical geometry problems! 
		
		
		\vspace{0.5em}
		
		The Olympiad Algebra Book is supposed to be a problem bank for Algebra, and it forms the resource for the first series of the KAYWAÑAN Algebra Contest. I suggest you start with Polynomials, and before you get bored or exhausted, also start solving Trigonometry problems. If you find these problems easy and not challenging enough, the Spherical Trigonometry lessons and problems are definitely going to be a must try!
		
		
		\vspace{0.5em}
		
		This booklet contains problems and solutions of $\texttt{KACY--I005}$ (Olympiad Pre--Algebra Contests), including the problems from the first book: $$\text{KACY--I}\left\{6,45,46,63,80,91,92,107,117\right\}.$$ The numbers referred here are the question number out of the 1220 questions labeled from \href{https://github.com/parvardi/KACY/blob/main/KACY-VOL-I.pdf}{1 to 1220}. The competition's full title is ``Kaywañan Olympiad Pre--Algebra Summer Contest 005,'' held on Saturday July $8^{th}$, 2023.
%	
%		\noindent ``Kaywañan Olympiad Pre--Algebra Summer Contest 001''
%		\begin{tasks}
%			\task $\texttt{KACY--I001}$: $\text{KACY--I}\left\{2,37,38,58,74,75,86,103,113\right\}$.
%%			\task $\texttt{KACY--I002}$: $\text{KACY--I}\left\{3,39,40,59,76,77,87,104,114\right\}$.
%%			\task $\texttt{KACY--I003}$: $\text{KACY--I}\left\{4,41,42,60,61,78,88,105,115\right\}$.
%%			\task $\texttt{KACY--I004}$: $\text{KACY--I}\left\{5,43,44,62,79,89,90,106,116\right\}$.
%%			\task $\texttt{KACY--I005}$: $\text{KACY--I}\left\{6,45,46,63,80,91,92,107,117\right\}$.
%%			\task $\texttt{KACY--I006}$: $\text{KACY--I}\left\{13,47,64,81,93,95,108,109,118\right\}$.
%%			\task $\texttt{KACY--I007}$: $\text{KACY--I}\left\{14,48,65,66,82,96,97,98,110,119\right\}$.
%%			\task $\texttt{KACY--I008}$: $\text{KACY--I}\left\{32,49,67,68,83,84,99,100,111,120\right\}$.
%%			\task $\texttt{KACY--I009}$: $\text{KACY--I}\left\{33,55,57,72,73,85,101,102,112,121\right\}$.
%		\end{tasks}  
%		Dates of \texttt{KACY--I001} to \texttt{KACY--I009}:
%		\begin{enumerate}
%			\item \texttt{KACY--I001}: June 3, 2023.
%%			\item \texttt{KACY--I002}: June 10, 2023.
%%			\item \texttt{KACY--I003}: June 17, 2023.
%%			\item \texttt{KACY--I004}: June 24, 2023.
%%			\item \texttt{KACY--I005}: July 1, 2023.
%%			\item \texttt{KACY--I006}: July 8. 2023.
%%			\item \texttt{KACY--I007}: July 15, 2023.
%%			\item \texttt{KACY--I008}: July 22, 2023.
%%			\item \texttt{KACY--I009}: July 29, 2023.
%		\end{enumerate}  
\Large
		\begin{flushright}
			Amir Parvardi,\\
			Vancouver, British Columbia,\\
			July 8, 2023
		\end{flushright}
	
	\newpage
	\normalsize
	\tableofcontents\label{TOC}
	%\listoffigures
	% \newpage
	% \listoftables
	% \cleardoublepage
	% \addcontentsline{toc}{chapter}{Index}
	% \printindex
	
%	\mainmatter
	\normalsize
	\pagestyle{fancy}
	\fancyhf{}
	% \fancyhead[LE]{\nouppercase{\rightmark\hfill\leftmark}}
	% \fancyhead[RO]{\nouppercase{\leftmark\hfill\rightmark}}
	\fancyfoot[LE,RO]{Amir Parvardi\hfill\thepage/\pageref{LastPage}\hfill KAYWAÑAN\textsuperscript{\textcopyleft}}
	\fancyhead[LE,RO]{Olympiad Algebra (Vol. I):\hfill 1220 Problems\hfill \texttt{KACY--I005} Booklet}

	\newpage

	\begin{tcolorbox}
		\begin{displayquote}
			``Let No One Ignorant of Algebra Enter!''
			\begin{flushright}
				\LARGE \textsc{Kaywan}
			\end{flushright}
		\end{displayquote}
	\end{tcolorbox}
	
\vspace{1em}
	
	The rules of the KACY Competitions are simple: 
	\begin{idea}
		\begin{tasks}
			\task All problems whose titles contain \textbf{KACY--I} are questions of the Summer KACY Series, and all problems with a title containing \textbf{KACY--II} are questions of the Winter KACY Series.
			\task This is the first volume of KACY, and it contains the SUMMER KACY questions. For the SUMMER KACY 2023 held weekly in Summer and Fall of 2023, only questions with title containing ``\textbf{KACY--I}'' are to be used in the actual KAYWAÑAN competitions.
		\end{tasks}
	\end{idea}
	
	\vspace{0.5em}

	This is because all the questions whose source does not contain \textbf{KACY--I} are either from a legit mathematical competition such as IMO, IMO Shortlist/Longlist, MAA Series (AMC, AIME, USAMO, USATST, USATSTST, USAMTS, etc.), National or Regional Olympiads (USA, APMC, Canada, etc.), or maybe from a book/paper I found and referenced in the question's title. 
	
	\vspace{0.5em}
	
	This assures that no famous problems are used in KACY, and that we actually identify and solve the non--KACY problems as exercises and examples in our journey of learning algebra during KAYWAÑAN Algebra Contest.
	
	\newpage
	
	\section*{KACY--I005 Problems}
	
	\setcounter{question}{5}
	
	\begin{tcolorbox}
		\begin{question}
		Solve for $x$:
		\begin{align*}
			x^4-3x^3-8x^2+12x+16=0.
		\end{align*}
	\end{question}
	\end{tcolorbox}
	
	\begin{solution}
		Simplifies to $(x - 4) (x - 2) (x + 1) (x + 2) =0$ which has solutions $$x=-1,\pm 2, 4.$$
	\end{solution}
	
	
	
	\setcounter{question}{44}
	
		
	\begin{tcolorbox}
		\SetupExSheets{headings=runin}
		\begin{question}
			If $n$ is a positive integer, factorize $$a^{5n}+a^n+1.$$
		\end{question}
	\end{tcolorbox}
	
	\begin{solution}%[name=Solution by Parviz Shahriari]
		Answer: $(a^{2n}+a^n+1)(a^{3n}-a^{2n}+1)$.
	\end{solution}
	
	
	
	
	\begin{tcolorbox}
		\SetupExSheets{headings=runin}
		\begin{question}
			Factorize $$(x+1)(x+3)(x+5)(x+7)+15.$$
		\end{question}
	\end{tcolorbox}
	
	\begin{solution}%[name=Solution by Parviz Shahriari]
		Answer: $(x+2)(x+6)(x+4+\sqrt{6})(x+4-\sqrt{6})$.
	\end{solution}
	
	
	
	
	
	\setcounter{question}{62}
	
	
	
	\begin{tcolorbox}
		\SetupExSheets{headings=runin}
		\begin{question}
			Show that $(x+y)^n - x^n - y^n$ always has a factor of \[nxy(x+y)(x^2+xy+y^2)^2,\] if $n=6k+1$ for some integer $k\geq 1$.
		\end{question}
	\end{tcolorbox}
	
	\begin{solution}
		It is easy using the Binomial Theorem to prove that for all positive integers $n$, $(x+y)^n - x^n - y^n$ has a factor of $n$. Define
		\[f(x,y)=(x+y)^n - x^n - y^n.\]
		Since $f(0,y)=f(x,0)=0$, we find that $f(x,y)$ is always divisible by $xy$. Furthermore, when $n$ is odd, we have $f(x,-x)=0$, which means that $f(x,y)$ is divisible by $x+y$ when $n$ is odd. Finally, assume that $x^2+xy+y^2$ is a quadratic in $x$ and solve it using the third roots of unity $\omega$ and $\omega^2$ (these are the roots of $\omega^3=1$):
		\[x^2+xy+y^2=0 \implies x_{1,2} = y \cdot \frac{-1 \pm i\sqrt{3}}{2},\]
		which gives us $x_1=y\omega$ and $x_2=y\omega^2$. Now calculate $f(y\omega,y)$:
		\[f(y\omega,y)=(y\omega+y)^n - (y\omega)^n - y^n = y^n(\omega+1)^n - y^n(\omega^n +1).\]
		Since $\omega^2+\omega+1=0$, we can replace $\omega+1$ with $-\omega^2$ and rewrite the above equation as
		\[f(y\omega,y)=-y^n(\omega^{2n}+\omega^n+1).\]
		As a result, since $\omega^{2n}+\omega^n+1$ equals zero if and only if $n\equiv 1 \pmod 3$, we find that $f(x,y)$ is divisible by $x^2+xy+y^2$. So far, we have proved that $f(x,y)$ is divisible by $xy(x+y)(x^2+xy+y^2)$ whenever $n=6k+1$. To finish, we need to prove that $f(x,y)$ is divisible by $(x^2+xy+y^2)^2$, which can be done by proving that the derivative of $f(x,y)$ (with respect to either $x$ or $y$) is divisible by $(x^2+xy+y^2)$. If we let
		$\varphi(x)=(x+y)^n - x^n - y^n$, then
		\[\varphi'(x)=n(x+y)^{n-1}-nx^{n-1}.\]
		Now, calculate either $\varphi'(y\omega)$ or $\varphi'(y\omega^2)$:
		\begin{align*}
			\varphi'(y\omega)= ny^{n-1}\left((\omega+1)^{n-1}-\omega^{n-1}\right).
		\end{align*}
		Since $n-1$ is even, $(\omega+1)^{n-1} = (-\omega^2)^{n-1} = \omega^{2(n-1)}$, and if we replace this in the previous equation (and let $n=6k+1$):
		\[\varphi'(y\omega) = ny^{n-1} \left(\omega^{12k} - \omega^{6k}\right)=0.\]
		Since both $\varphi)(x)$ and $\varphi'(x)$ are divisible by $x^2+xy+y^2$, it means that $\varphi(x)$ is divisible by $(x^2+xy+y^2)^2$ and we are done.
	\end{solution}
	

	\setcounter{question}{79}	
	
	
	\begin{tcolorbox}
		\SetupExSheets{headings=runin}
		\begin{question}
			Factorize $$(x-y)^5 + (y-z)^5 + (z-x)^5.$$
		\end{question}
	\end{tcolorbox}
	
	\begin{solution}%[name=Solution by Parviz Shahriari]
		Answer: $5(x-y)(y-z)(z-x)(x^2+y^2+z^2-xy-yz-zx)$.
	\end{solution}
	
	

	
	\setcounter{question}{90}
	
		
	\begin{tcolorbox}
		\SetupExSheets{headings=runin}
		\begin{question}
			Factorize $$(x^2+y^2)^3+(z^2-x^2)^3-(y^2+z^2)^3.$$
		\end{question}
	\end{tcolorbox}
	
	\begin{solution}%[name=Solution by Parviz Shahriari]
		Answer: $3(y^2+z^2)(x^2+y^2)(x-z)(x+z)$.
	\end{solution}
	
	
	\begin{tcolorbox}
		\SetupExSheets{headings=runin}
		\begin{question}
			Factorize $$x^3(y-z) + y^3(z-x) + z^3(x-y).$$
		\end{question}
	\end{tcolorbox}
	
	\begin{solution}%[name=Solution by Parviz Shahriari]
		Answer: $-(x+y+z)(x-y)(y-z)(z-x)$.
	\end{solution}
	
	
	\setcounter{question}{106}
	
	\begin{tcolorbox}
		\SetupExSheets{headings=runin}
		\begin{question}
			Assuming
			\begin{align*}
				f(x+1)= x^2-3x+2,
			\end{align*}
			Find $f(x)$.
		\end{question}
	\end{tcolorbox}
	
	\begin{solution}%[name=Solution by Parviz Shahriari]
		Answer: $f(x)=x^2-5x+7$.
	\end{solution}
	
	\setcounter{question}{116}
			


\begin{tcolorbox}
	\SetupExSheets{headings=runin}
	\begin{question}
		\begin{itemize}
			\item[(a)] Find two roots for the following equation:
			\begin{align*}
				f(x)=f\left(\frac{x+8}{x-1}\right).
			\end{align*}
			\item[(b)] If $f(x)=x^2-12x+3$, find all the roots of the equation given in (a).
		\end{itemize}
	\end{question}
\end{tcolorbox}

\begin{solution}%[name=Solution by Parviz Shahriari]
	Answer: (a) $x=-2, 4$, (b) $x=-2, 2, 4, 10$.
\end{solution}			
			
	\newpage
	\section*{KACY--I005 Answers}
	%	
	The problems and solutions of KACY--I005 are courtesy of Parviz Shahriari, and they are taken from his eternal two-volume Farsi contribution to mathematics:: ``Methods of Algebra.'' May he rest in peace!
	
	\vspace{1em}
	
	\printsolutions
\end{document}


