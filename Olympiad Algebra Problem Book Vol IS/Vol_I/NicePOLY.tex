\section{Nice Polynomial Problems}

We present $400$ polynomial problems in this section. The first series of these nice polynomial problems is a \href{https://artofproblemsolving.com/community/c6h397768p2212439}{Collection} of $100$ polynomial problems that I collected from AoPS around $2011$. The second series is a collection of $300$ ancient polynomial problems, which, I believe, should make the most complete and extensive resource for studying polynomials.

\subsection{100 Nice Polynomial Problems}



\begin{question}
Find all polynomials $P(x)$ with real coefficient such that:
\[P(0)=0 , \quad \text{and} \quad \lfloor P \lfloor P(n)\rfloor \rfloor +n=4\lfloor P(n) \rfloor \quad \forall n \in \mathbb N.\]
\end{question}

\begin{question}
Find all functions $f: \mathbb R\to R$ such that
\[f(x^n+2f(y))=(f(x))^n +y+f(y) \quad  \forall x, y \in \mathbb R, \quad  n \in \mathbb Z_{\geq 2}.\]
\end{question}

\begin{question}
Find all functions $ f : \mathbb R\to \mathbb R$ such that
\[ x^2y^2 \left( f(x+y)-f(x)-f(y) \right)=3(x+y)f(x)f(y).\]
\end{question}

\begin{question}
Find all polynomials $P(x)$ with real coefficients such that
\[P(x)P(x + 1) = P(x^2) \quad \forall x \in \mathbb R.\]
\end{question}

\begin{question}
Find all polynomials $P(x)$ with real coefficient such that
\[P(x)Q(x)=P(Q(x)) \quad \forall x \in \mathbb R.\]
\end{question}


\begin{question}
Find all polynomials $P(x)$ with real coefficients such that if $P(a)$ is an integer, then so is $a$, where $a$ is any real number.
\end{question}


\begin{question}
Find all the polynomials $f\in \mathbb{R}[X]$ such that
\[\sin f(x)=f(\sin x),\ (\forall)x\in \mathbb{R}.\]
\end{question}


\begin{question}
Find all polynomial $f(x) \in \mathbb{R}[x] $ such that 
\[f(x)f(2x^2)=f(2x^3+x^2) \quad \forall  x\in \mathbb{R}.\]
\end{question}


\begin{question}
Find all real polynomials $f$ and $g$, such that:
\[(x^2+x+1)\cdot f(x^2-x+1)=(x^2-x+1)\cdot g(x^2+x+1), \]
for all $x\in\mathbb{R}$.
\end{question}


\begin{question}
Find all polynomials $P(x)$ with integral coefficients such that $P(P'(x))=P'(P(x))$ for all real numbers $x$.
\end{question}


\begin{question}
Find all polynomials with integer coefficients f such that for all $n>2005$ the number $ f(n)$ is a divisor of $ n^{n-1}-1$.
\end{question}


\begin{question}
Find all polynomials with complec coefficients f such that we have the equivalence: for all complex numbers z, $ z\in[-1,1]$ if and only if $ f(z)\in [-1,1]$.
\end{question}


\begin{question}
Suppose $f$ is a polynomial in $\mathbb{Z}[X]$ and m is integer .Consider the sequence $a_i$ like this $a_1=m$ and $a_{i+1}=f(a_i)$ find all polynomials $f$ and alll integers $m$ that for each $i$:
\[ a_i | a_{i+1}.\]
\end{question}


\begin{question}
$P(x),Q(x) \in \mathbb{R}[x]$ and we know that for real $r$ we have $p(r) \in \mathbb{Q}$ if and only if $Q(r) \in \mathbb{Q}$
I want some conditions between $P$ and $Q$.My conjecture is that there exist ratinal $a,b,c$ that $aP(x)+bQ(x)+c=0$
\end{question}


\begin{question}
Find the $\gcd$ of the polynomials $X^n+a^n$ and $X^m+a^m$, where $a$ is real.
\end{question}


\begin{question}
Find all polynomials $p$ with real coefficients that if for a real $a$,$p(a)$ is integer then $a$ is integer.
\end{question}



\begin{question}
$\mathfrak {question}$ is a real polynomial such that if $\alpha$ is irrational then $\mathfrak {question} (\alpha)$ is irrational. Prove that $\deg [\mathfrak{question}]\leq 1$
\end{question}



\begin{question}
Show that the odd number $n$ is a prime number if and only if the polynomial $ T_n(x)/x$ is irreducible over the integers.
\end{question}



\begin{question}
$P,Q,R$ are non-zero polynomials that for each $z\in\mathbb C$, $P(z)Q(\bar z)=R(z)$.
a) If $P,Q,R\in\mathbb R[x]$, prove that $Q$ is constant polynomial.
b) Is the above statement correct for $P,Q,R\in\mathbb C[x]$?
\end{question}



\begin{question}
Let $P$ be a polynomial such that $P(x)$ is rational if and only if $x$ is rational. Prove that $P(x)=ax+b$ for some rational $a$ and $b$.
\end{question}



\begin{question}
Prove that any polynomial $\in \mathbb{R} [X]$ can be written as a difference of two strictly increasing polynomials.
\end{question}



\begin{question}
Consider the polynomial $W(x) = (x - a)^kQ(x)$, where $a \neq 0$, $Q$ is a nonzero polynomial, and $k$ a natural number. Prove that $W$ has at least $k + 1$ nonzero coefficients.
\end{question}



\begin{question}
Find all polynomials $p(x) \in \mathbb{R}[x]$ such that the equation \[f(x) = n\] has at least one rational solution, for each positive integer $n$.
\end{question}



\begin{question}
Let $f\in\mathbb{Z}[X]$ be an irreducible polynomial over the ring  of integer polynomials, such that $|f(0)|$ is not a perfect square. Prove that if the leading coefficient of $f$ is 1 (the coefficient of the term having the highest degree in $f$) then $f(X^2)$ is also irreducible in the ring of integer polynomials.
\end{question}



\begin{question}
Let $ p$ be a prime number and $ f$ an integer polynomial of degree $ d$ such that $ f(0) = 0,f(1) = 1$ and $ f(n)$ is congruent to $ 0$ or $ 1$ modulo $ p$ for every integer $ n$. Prove that $ d\geq p - 1$.
\end{question}



\begin{question}
Let \[P(x): =x^{n}+\sum\limits_{k=1}^{n}a_kx^{n-k},\] with $0\leq a_n\leq a_{n-1}\leq \cdots \leq a_2\leq a_1\le 1$. Suppose that there exists $r\ge 1 ,\; \varphi \in {\mathbb R}$  such that  $P(re^{i\varphi})=0$. Find $r$.
\end{question}



\begin{question}
Let $\mathcal P$ be a polynomial with rational coefficients such that
\[\mathcal P^{-1}(\mathbb{Q}) \subseteq \mathbb{Q}.\]
Prove that $\deg\mathcal P \leq 1$.
\end{question}


\begin{question}
Let $f$ be a polynomial with integer coefficients such that $|f(x)|<1$ on an interval of length at least 4. Prove that  $f=0$.
\end{question}


\begin{question}
prove that $x^n-x-1$ is irreducible over $\mathbb{Q}$ for all $n \geq 2$.
\end{question}



\begin{question}
Find all real polynomials $p(x)$ such that 
\[p^{2}(x)+2p(x)p\left(\frac{1}{x}\right)+p^{2}\left(\frac{1}{x}\right) = p(x^{2})p\left(\frac{1}{x^{2}}\right),\]
for all non-zero real $x$.
\end{question}



\begin{question}
Find all polynomials $P(x)$ with odd degree such that 
\[P(x^{2}-2)=P^{2}(x)-2.\]
\end{question}



\begin{question}
Find all real polynomials that \[p(x+p(x))=p(x)+p(p(x)),\] for all reals $x$.
\end{question}



\begin{question}
Find all polynomials $P\in \mathbb C[X]$ such that \[P(X^{2})=P(X)^{2}+2P(X).\]
\end{question}



\begin{question}
Find all polynomials of two variables $P(x,y)$ which satisfy
\[P(a,b) P(c,d) = P (ac+bd, ad+bc), \qquad \text{for all} \quad a,b,c,d \in \mathbb{R}.\]
\end{question}



\begin{question}
Find all real polynomials $f(x)$ satisfying
\[f(x^{2})=f(x)f(x-1), \qquad \text{for all} \quad x \in \mathbb R.\]
\end{question}



\begin{question}
Find all polynomials of degree $3$, such that for all $x,y\geq 0$, \[p(x+y)\geq p(x)+p(y).\]
\end{question}



\begin{question}
Find all polynomials $P(x)\in \mathbb Z[x]$ such that for any $n\in \mathbb N$, the equation $P(x)=2^{n}$ has an integer root.
\end{question}



\begin{question}
Let $f$ and $g$ be polynomials such that $f(Q)=g(Q)$ for all rationals $Q$. Prove that there exist reals $a$ and $b$ such that $f(X)=g(aX+b)$, for all real numbers $X$.
\end{question}



\begin{question}
Find all positive integers $ n\geq 3 $ such that there exists an arithmetic progression $ a_0 , a_1, \ldots, a_n $ such that the equation $ a_nx^n + a_{n-1}x^{n-1} + \cdots+ a_1x+a_0 = 0 $ has $n$ roots setting an arithmetic progression.
\end{question}



\begin{question}
Given non-constant linear functions $p_1(x), p_2(x), \dots p_n(x)$. Prove that at least $n-2$ of polynomials $p_1p_2\dots p_{n-1}+p_n,  p_1p_2\dots p_{n-2} p_n + p_{n-1},\dots p_2p_3\dots p_n+p_1$ have a real root.
\end{question}



\begin{question}
Find all positive real numbers $a_1,a_2,\ldots,a_k$ such that the number \[ a_1^{\frac{1}{n}}+\cdots+a_k^{\frac{1}{n}},\] is rational for all positive integers $n$, where $k$ is a fixed positive integer.
\end{question}



\begin{question}
Let $f,g$ be real non-constant polynomials such that $ f(\mathbb Z)=g(\mathbb Z) $. Show that there exists an integer $A$ such that $ f(X)=g(A+x) $ or $ f(x)=g(A-x) $. 
\end{question}


\begin{question}
Does there exist a polynomial $f \in \mathbb{Q}[x]$ with rational coefficients such that $f(1) \neq -1$, and $x^nf(x) + 1$ is a reducible polynomial for every $n \in \mathbb{N}$?
\end{question}



\begin{question}
Suppose that  $ f $ is a polynomial of exact degree  $ p.$ Find a rigorous proof that $ S(n) $,  where  $ S(n)= \sum\limits_{k=0}^{n}f(k) , $  is a polynomial function  of (exact) degree  $ p+1 $ in variable $n $ .
\end{question}






\begin{question}
The polynomials $P,Q$ are such that $\deg P=n$,$\deg Q=m$, have the same leading coefficient, and $P^2(x)=(x^2-1)Q^2(x)+1$. Prove that $P'(x)=nQ(x)$
\end{question}





\begin{question}
Given distinct prime numbers $p$ and $q$ and a natural number $n \geq 3$, find all $a \in \mathbb{Z}$ such that the polynomial $f(x) = x^n + ax^{n-1} + pq$ can be factored into 2 integral polynomials of degree at least 1.
\end{question}





\begin{question}
Let $F$ be the set of all polynomials $\Gamma$ such that all the coefficients of $\Gamma (x)$ are integers and $\Gamma (x) = 1$ has integer roots. Given a positive integer $k$, find the smallest integer $m(k) > 1$ such that there exist $\Gamma \in F$ for which $\Gamma (x) = m(k)$ has exactly $k$ distinct integer roots.
\end{question}





\begin{question}
Find all polynomials $P(x)$ with integer coefficients such that the polynomial \[ Q(x)=(x^2+6x+10) \cdot P^2(x)-1, \] is the square of a polynomial with integer coefficients.
\end{question}





\begin{question}
Find all polynomials $p$ with real coefficients such that for all reals $a, b ,c$ such that $ab+bc+ca =1$ we have the relation \[p(a)^{2}+p(b)^{2}+p(c)^{2}=p(a+b+c)^{2}.\]
\end{question}





\begin{question}
Find all real polynomials $ f$ with $ x,y \in \mathbb{R}$ such that
\[ 2 y f(x + y) + (x - y)(f(x) + f(y)) \geq 0.\]
\end{question}





\begin{question}
Find all polynomials such that $P(x^3+1) = P((x+1)^3)$.
\end{question}








\begin{question}
Find all polynomials $P(x) \in \mathbb{R}[x]$ such that $P(x^2+1)=P(x)^2+1$ holds for all $x \in \mathbb{R}$.
\end{question}





\begin{question}
Find all polynomials $p(x)$ with real coefficients such that
\[ (x+1)p(x-1) + (x-1)p(x+1) = 2x p(x),  \]
for all real $x$.
\end{question}





\begin{question}
Find all polynomials $P(x)$ that have only real roots, such that \[ P(x^2-1)=P(x)P(-x). \]
\end{question}





\begin{question}
Find all polynomials $P(x) \in \mathbb R[x]$such that:
\[P(x^2)+x \cdot (3P(x)+P(-x))=(P(x))^2+2x^2, \qquad \text{for all} \quad x\in  \mathbb R.\]
\end{question}





\begin{question}
Find all polynomials $f,g$ which are both monic and have the same degree and
\[f(x)^2-f(x^2)=g(x).\]
\end{question}





\begin{question}
Find all polynomials $P(x)$ with real coefficients such that there exists a polynomial $Q(x)$ with real coefficients that satisfy 
\[P(x^2)=Q(P(x)).\]
\end{question}





\begin{question}
Find all polynomials $p(x,y)\in\mathbb R[x,y]$ such that for each $x,y\in\mathbb R$ we have
\[ p(x+y,x-y)=2p(x,y). \]
\end{question}





\begin{question}
Find all couples of polynomials $(P,Q)$ with real coefficients, such that for infinitely many $x\in\mathbb R$ the condition \[ \frac{P(x)}{Q(x)}-\frac{P(x+1)}{Q(x+1)}=\frac{1}{x(x+2)},\]
holds.
\end{question}





\begin{question}
Find all polynomials $P(x)$ with real coefficients, such that \[P(P(x))=P(x)^k,\] for any given positive integer $k$.
\end{question}





\begin{question}
Find all polynomials 
\[P_{n}(x)=n!x^n + a_{n-1}x^{n-1}+\cdots+ a_{1}x +(-1)^n(n+1)n\]
with integers coefficients and with $n$ real roots $x_1, x_2, \dots, x_n$, such that $k\leq x_k\leq k+1$, for $k=1,2,\dots,n$.
\end{question}





\begin{question}
The function $f(n)$ satisfies $f(0)=0$ and \[f(n)=n-f \left( f(n-1) \right),\] for $n=1,2,3, \dots$. Find all polynomials $g(x)$ with real coefficient such that
\[ f(n)= [ g(n) ], \qquad n=0,1,2, \dots \]
Where $[ g(n) ]$ denote the greatest integer that does not exceed $g(n)$.
\end{question}





\begin{question}
Find all pairs of integers $a,b$ for which there exists a polynomial $P(x) \in \mathbb{Z}[X]$ such that product $(x^2+ax+b)\cdot P(x)$ is a polynomial of a form \[ x^n+c_{n-1}x^{n-1}+\cdots+c_1x+c_0,  \] where each of $c_0,c_1,\dots,c_{n-1}$ is equal to $1$ or $-1$.
\end{question}





\begin{question}
There exists a polynomial $P$ of degree $5$ with the following property: if $z$ is a complex number such that $z^{5}+2004z=1$, then $P(z^{2})=0$. Find all such polynomials $P$.
\end{question}





\begin{question}
Find all polynomials $P(x)$ with real coefficients satisfying the equation \[(x+1)^{3}P(x-1)-(x-1)^{3}P(x+1)=4(x^{2}-1) P(x),\] for all real numbers $x$.
\end{question}


\begin{question}
Find all polynomials $P(x,y)$ with real coefficients such that:
\[P(x,y)=P(x+1,y)=P(x,y+1)=P(x+1,y+1).\]
\end{question}




\begin{question}
Find all polynomials $ P(x)$ with reals coefficients such that
\[(x-8)P(2x)=8(x-1)P(x).\]
\end{question}


\begin{question}
Find all reals $ \alpha$ for which there is a nonzero polynomial $P$ with real coefficients such that 
\[\frac{P(1)+P(3)+P(5)+\cdots+P(2n-1)}{n}=\alpha P(n), \qquad \text{for all} \quad n \in \mathbb N,\]
and find all such polynomials for $\alpha=2$.
\end{question}





\begin{question}
Find all polynomials $ P(x) \in \mathbb{R}[X]$ satisfying 
\[ (P(x))^2-(P(y))^2=P(x+y)\cdot P(x-y) ,\quad  \forall x,y \in \mathbb R.\]
\end{question}





\begin{question}
Find all $ n\in\mathbb{N}$ such that polynomial
\[ P(x) = (x - 1)(x - 2) \cdots (x -n),\]
can be represented as $Q(R(x))$, for some polynomials $Q(x)$ and $R(x)$ with degree greater than $1$.
\end{question}





\begin{question}
Find all polynomials $ P(x) \in \mathbb R[x]$ such that
\[P(x^2-2x)=\left( P(x)-2 \right)^2.\]
\end{question}





\begin{question}
Find all non-constant real polynomials $f(x)$ such that for any real $x$ the following equality holds
\[ f(\sin x +\cos x) = f(\sin x) + f(\cos x).\]
\end{question}





\begin{question}
Find all polynomials $ W(x)\in \mathbb R[x]$ such that
\[W(x^2)W(x^3)=W(x)^5, \qquad \text{for all} \quad x \in \mathbb R.\]
\end{question}





\begin{question}
Find all the polynomials $f(x)$ with integer coefficients such that $f(p)$ is prime for every prime $p.$
\end{question}





\begin{question}
Let $ n \geq 2$ be a positive integer. Find all polynomials $ P(x)=a_0 +a_1 x +\cdots + a_{n} x^n$ having exactly $ n$ roots not greater than $-1$ and satisfying
\[ a^2_0+ a_1 a_n = a^2_n +a_0 a_{n-1}.\]
\end{question}





\begin{question}
Find all polynomials  $ P(x),Q(x)$ such that
\[P(Q(X))=Q(P(x)), \qquad \text{for all} \quad  x \in \mathbb R.\]
\end{question}





\begin{question}
Find all integers $ k$ such that for infinitely many integers $ n \ge 3$ the polynomial
\[ P(x) =x^{n+ 1}+ kx^n - 870x^2 + 1945x + 1995,\]
can be reduced into two polynomials with integer coefficients.
\end{question}





\begin{question}
Find all polynomials $ P(x),Q(x),R(x)$ with real coefficients such that
\[\sqrt{P(x)}-\sqrt{Q(x)}=R(x), \qquad \text{for all} \quad  x \in \mathbb R.\]
\end{question}





\begin{question}
Let $k = \sqrt[3]{3}$. Find a polynomial $p(x)$ with rational coefficients and degree as small as possible such that $p(k+k^{2}) = 3+k$. Does there exist a polynomial $q(x)$ with integer coefficients such that $q(k+k^{2}) = 3+k$?
\end{question}





\begin{question}
Find all values of the positive integer $ m$ such that there exists polynomials $ P(x),Q(x),R(x,y)$ with real coefficient satisfying the condition: For every real numbers $ a,b$ which satisfying $ a^m-b^2=0$, we always have that $ P(R(a,b))=a$ and $ Q(R(a,b))=b$.
\end{question}





\begin{question}
Find all polynomials $ p(x)\in \mathbb{R}[x]$ such that \[p(x^{2008} + y^{2008}) = (p(x))^{2008}+(p(y))^{2008},\] for all real numbers $x$ and $y$.
\end{question}





\begin{question}
Find all Polynomials $P(x)$ satisfying $P(x)^2-P(x^2) =2x^4$.
\end{question}





\begin{question}
Find all polynomials $ p$ of one variable with integer coefficients such that if $ a$ and $ b$ are natural numbers such that $ a +b$ is a perfect square, then $ p\left(a\right)+ p\left(b\right)$ is also a perfect square.
\end{question}





\begin{question}
Find all polynomials $P(x)\in \mathbb Q[x]$ such that
\[P(x)=P\left(\frac{-x+\sqrt{3 -3x^2}}{2}\right), \quad \text{ for all } \quad |x| \le 1.\]
\end{question}





\begin{question}
Find all polynomials $f$ with real coefficients such that for all reals $a,b,c$ such that $ab+bc+ca = 0$ we have the following relations
\[ f(a-b) + f(b-c) + f(c-a) = 2f(a+b+c). \]
\end{question}





\begin{question}
Find All Polynomials $ P(x,y)$ such that for all reals $ x,y$ we have
\[P(x^{2},y^{2}) =P\left(\frac {(x + y)^{2}}{2},\frac {(x - y)^{2}}{2}\right).\]
\end{question}





\begin{question}
Let $ n$ and $ k$ be two positive integers. Determine all monic polynomials $ f\in\mathbb{Z}[X]$ of degree $ n,$ having the property that 
\begin{align*}
    f(n) \text{ divides } f\left (2^{k}\cdot a\right ), \qquad \text{for all} \quad  a\in\mathbb{Z} \text{ with } f(a)\neq 0.
\end{align*}
\end{question}




\begin{question}
Find all polynomials $ P(x)$ such that
\[P(x^2-y^2)=P(x+y)P(x-y).\]
\end{question}





\begin{question}
Let $f(x)=x^4-x^3+8ax^2-ax+a^2$. Find all real number $ a$ such that $f(x)=0$ has four different positive solutions.
\end{question}




\begin{question}
Find all polynomial $ P\in \mathbb {R}[x]$ such that: $P(x^2+2x+1)=(P(x))^2+1$.
\end{question}





\begin{question}
Let $ n\ge 3$ be a natural number. Find all non-constant polynomials with real coefficients $ f_{1}\left(x\right),f_{2}\left(x\right),\dots,f_{n}\left(x\right)$, for which
\[ f_{k}\left(x\right)f_{k+ 1}\left(x\right) = f_{k +1}\left(f_{k + 2}\left(x\right)\right), \quad  1\le k\le n,\]
for every real $ x$ (with $ f_{n +1}\left(x\right)\equiv f_{1}\left(x\right)$ and $ f_{n + 2}\left(x\right)\equiv f_{2}\left(x\right)$).
\end{question}





\begin{question}
Find all integers $n$ such that the polynomial $p(x)=x^5-nx-n-2$ can be written as product  of two non-constant polynomials with integral coefficients.
\end{question}




\begin{question}
Find all polynomials $p(x)$ that satisfy 
\[(p(x))^2-2=2p(2x^2-1), \qquad \text{for all} \quad x \in \mathbb R.\]
\end{question}




\begin{question}
Find all polynomials $p(x)$ that satisfy
\[(p(x))^2-1=4p(x^2-4X+1), \qquad \text{for all} \quad  x \in \mathbb R.\]
\end{question}


% \begin{question}
% Determine the polynomials $P$ of two variables so that:

% {\bf a.)} for any real numbers $t,x,y$ we have $P(tx,ty) = t^n P(x,y)$ where $n$ is a positive integer, the same for all $t,x,y;$

% {\bf b.)} for any real numbers $a,b,c$ we have $P(a + b,c) + P(b + c,a) + P(c + a,b) = 0;$

% {\bf c.)} $P(1,0) =1.$
% \end{question}


\begin{question}
Determine the polynomials $P$ of two variables so that:
\begin{tasks}
    \task for any real numbers $t,x,y$ we have $P(tx,ty) = t^n P(x,y)$ where $n$ is a positive integer, the same for all $t,x,y;$
    \task for any real numbers $a,b,c$ we have \[P(a + b,c) + P(b + c,a) + P(c + a,b) = 0;\]
    \task $P(1,0) =1.$
\end{tasks}
\end{question}



\begin{question}
Find all polynomials $ P(x)$ satisfying the equation
\[(x+1)P(x)=(x-2010)P(x+1).\]
\end{question}




\begin{question}
Find all polynomials of degree $3$ such that for all non-negative reals $x$ and $y$ we have
\[p(x+y) \leq p(x)+p(y).\]
\end{question}



\begin{question}
Find all polynomials $p(x)$ with real coefficients such that
\[p(a + b - 2c) + p(b + c - 2a) + p(c + a - 2b) = 3p(a - b) + 3p(b - c) + 3p(c - a),\]
for all $a, b, c\in\mathbb{R}$.
\end{question}



\begin{question}
Find all  polynomials $P(x)$ with real coefficients such that
\[P(x^2-2x)=\left(P(x-2)\right)^2.\]
\end{question}



\begin{question}
Find all two-variable polynomials $p(x,y)$ such that for each $a,b,c\in\mathbb R$:
\[p(ab,c^2+1)+p(bc,a^2+1)+p(ca,b^2+1)=0.\]
\end{question}


\newpage
\subsection{300 Ancient Polynomial Problems}
The following problems are taken from a resource of Olympiad Algebra in Iran, the book ``\textit{Topics and Discussions in of Algebra in Math Olympiads}'' by \textit{Mehdi Safa}, published by \textit{Khoshkhan} publishing, written in Farsi (titles have been translated to English). Some of the problems are really old, for example the first problem in \textit{Chapter 22: Various Problems on Polynomials}) of the book is dated $1907$, from Hungary. Most problems, however, come from the $1990$'s and not so old when these lines are being written (May $2023$). However, in the scope of math olympiad preparation, even twenty years is a life-time, and new problems are a much more popular choice for students to solve. As a result, some of the classic, ``old'' and forgotten problems here are like gems to those who run out of ``new'' problems! The last hundred problems or so are taken from well-known competitions such as China TST, Austrian--Polish Mathematical Competition (APMC), Czech and Slovak Competition, IMO Shortlist, and IMO Longlist.


\begin{question}[name={1907 Hungary}]
    Prove that if $p$ and $q$ are two odd integers, then the equation $x^2+2px+2q=0$ does not have a rational root.
\end{question}


\begin{question}[name={1907 Hungary}]
    Prove that the polynomial $P(x)=x^4+2x^2+2x+2=0$ cannot be written as a product of two polynomials $x^2+ax+b$ and $x^2+cx+d$ where $a,b,c,d$ are integers.
\end{question}


\begin{question}[name={1996 Bulgaria}]
    Let $a,b,c$ be real numbers and define $M$ as the maximum value of the expression
    \[|4x^3+ax^2+bx+c| \text{ for } x\in [-1,1].\]
    Prove that $M\geq 1$ and find all cases when $M=1$.
\end{question}


\begin{question}[name={1994 Romania}]
    Let $m,n$ be given positive integers. Find all common roots of
    \[P(x)=x^{m+1}-x^n+1 \qquad \text{and} \qquad Q(x)=x^{n+1}-x^m+1.\]
\end{question}


\begin{question}[name={1990 Iran}]
    Can we find four real numbers such that for each two of them like $x$ and $y$,
    \[x^{10} + x^9y + \cdots + xy^9 + y^{10} = 1?\]
\end{question}

\begin{question}[name={1969 USSR}]
    Find the smallest positive integer $a$ for which there exists a quadratic polynomial $P(x)=ax^2+bx+c$ such that its roots are distinct and smaller than $1$.
\end{question}

\begin{question}[name={1987 Iran}]
    Find all polynomials $P(x)$ such that for all $x$, \[xP(x-1)=(x-12)P(x).\]
\end{question}

\begin{question}[name={1997 Bulgaria}]
    Find all real numbers $m$ such that the polynomial
    \[P(x)=(x^2-2mx-4(m^2+1))\cdot (x^2-4x-2m(m^2+1)),\]
    has exactly three distinct roots.
\end{question}


\begin{question}[name={1997 Austrian--Polish}]
    Let $p_1,p_2,p_3,p_4$ be distinct prime numbers. Prove that there does not exist a cubic polynomial $Q(x)$ with integer coefficients such that
    \[|Q(p_1)|=|Q(p_2)|=|Q(p_3)|=|Q(p_4)|=3.\]
\end{question}


\begin{question}[name={1996 Iran}]
    Define $f(x)=ax^2+bx+c$ and assume that for $0\leq x \leq 1$, we have $|f(x)|\leq 1$. Find the maximum value of $2a+b$.
\end{question}

\begin{question}
    Let $P,Q,R,S$ are polynomials satisfying the equation
    \[P(x^4)+xR(x^8) + x^2Q(x^{12}) = (1+x+x^2+x^3)S(x).\]
    Prove that $x-1$ is a factor of $P(x)$.
\end{question}


\begin{question}[name={1998 Iran}]
    Let $P(x)$ be a polynomial with real coefficients such that for all $x\geq 0$ we have $P(x) > 0$. Prove that there exists a positive integer $m$ such that all coefficients of the polynomial $(1+x)^mP(x)$ are non-negative.
\end{question}

\begin{question}
    Prove that there do not exist non-constant polynomials $f,g,h$ such that
    \[\frac{f(x+1)}{g(x+1)}-\frac{f(x)}{g(x)} = h\left(\frac{1}{x}\right).\]
\end{question}


\begin{question}
    Does there exist an integer $c$ such that all the roots of the polynomial \[P(x)=x^3-87x^2+181x+c,\] are integers.
\end{question}




\begin{question}
    For a positive integer $k$, let $P(x)$ be a polynomial with integer coefficients such that the numbers $P(1),P(2),\dots,P(k)$ are not divisible by $k$. Prove that $P(x)$ cannot have any integer roots.
\end{question}



\begin{question}[name={1978 Romania}]
    Prove that for any polynomial $P(x)\neq x$ and any positive integer $n$, the polynomial $Q_n(x)$, defined by
    \begin{align*}
        Q_n(x)=\underbrace{P(P(\dots P(x)))}_{n \text{ times}} - x,
    \end{align*}
    is divisible by $Q_1(x)=P(x)-x$.
\end{question}


\begin{question}
    For two integers $a$ and $b$ such that $x^2-x-1$ is a factor of $ax^{17}+bx^{16}+1$. Find $a$.
\end{question}




\begin{question}
    For any positive integer $n$, prove that the polynomial \[P_n(x)=x^{n+2}-2x+1,\] has exactly one root in the interval $[0,1]$.
\end{question}


\begin{question}[name={1999 Iran}]
    Let $P(x)$ be a polynomial of degree $n$ such that for integer $x$, we know that $P(x)$ is integer. Prove that there exist integers $a_0,a_1,\dots,a_n$ such that
    \begin{align*}
        P(x)=a_n\binom{x}{n}+\cdots+a_1\binom{x}{1} + a_0.
    \end{align*}
\end{question}


\begin{question}
    For two distinct real numbers $a$ and $b$, prove that the polynomial
    \begin{align*}
        P(x)=(a-b)x^n + (a^2-b^2)x^{n-1} + \cdots + (a^{n+1} - b^{n+1}),
    \end{align*}
    has at most one real root.
\end{question}


\begin{question}[name={1995 Iran First Round}]
    Let $F(x)$ and $G(x)$ be two polynomials with integer coefficients such that $F(x)/G(x)$ is an integer for values of $x=1,2,3,\dots$. Prove that $F(x)$ is divisible by $G(x)$.
\end{question}


\begin{question}[name={1989 Iran Second Round}]
    Prove that for all positive integers $n>1$, the polynomial
    \begin{align*}
        P(x)= \frac{x^N}{n!} + \frac{x^{n-1}}{(n-1)!} + \cdots + \frac{x}{1} + 1,
    \end{align*}
    does not have integer roots.
\end{question}


\begin{question}[name={1997 Bulgaria}]
    Find all values of $a$ such that for all $x\in [0,1]$, we have $|f(x)|\leq 1$, where
    \begin{align*}
        f(x) = x^2 - 2ax - a^2 - \frac{3}{4}.
    \end{align*}
\end{question}


\begin{question}[name={1974 International Mathematics Olympiad}]
    Let $P(x)$ be a non-constant polynomial with integer coefficients and denote by $n(P(x))$ the number of integers $k$ such that $(P(k))^2=1$. Prove that
    \begin{align*}
        n(P(x)) - \deg(P(x)) \leq 2.
    \end{align*}
\end{question}

\begin{question}[name={1996 Poland}]
    Find all pairs $(n,r)$ of positive integer $n$ and real number $r$ such that the polynomial $(x+1)^n-r$ is divisible by the quadratic $2x^2+2x+1$.
\end{question}


\begin{question}[name={1914 Hungary}]
    Let $P(x)$ be a quadratic polynomial such that for $x\in [-1,1]$, we have $P(x)\in [-1,1]$. Prove that $P'(x) \in [-4,4]$.
\end{question}




\begin{question}[name={1918 Hungary}]
    If $p,q,r$, and $a,b,c$ are real numbers such that for all reals $x$, we have
    \begin{align*}
        ax^2+2bx+c \geq 0 \qquad \text{and} \qquad px^2+2qx+r \geq 0,
    \end{align*}
    then prove that $apx^2+bqx+cr \geq 0$. 
\end{question}



\begin{question}[name={1988 Iran}]
    If $\alpha$ is a root of the cubic polynomial $x^3+x^2+2x-1$, find the other two roots in terms of $\alpha$.
\end{question}


\begin{question}[name={1997 Iran}]
    Consider all quadratic polynomials $x^2+px+q$ in which $p$ and $q$ are integers with $1 \leq p,q \leq 1997$. Determine the number of which of the two kinds of polynomials is larger: those quadratics that have integer roots, or those quadratics with no integer roots?
\end{question}




\begin{question}
    Does there exist a sequence $\{a_n\}_{n=0}^\infty$ of real non-zero numbers such that for any positive integer $n$, the polynomial $a_0+a_1x+a_2x^2+\cdots+a_nx^n$ has exactly $n$ distinct real roots?
\end{question}


\begin{question}[name={1999 Iran}]
    Let $P(x)$ be a polynomial with degree less than $n$. Prove that
    \begin{align*}
        \sum_{i=0}^n P(i)(-1)^i\binom{n}{i}=0.
    \end{align*}
\end{question}


\begin{question}[name={1996 Bulgaria}]
    Let $f(x)$ and $g(x)$ be quadratic polynomials with real coefficients such that if $g(x)$ is an integer for some $x>0$, then $f(x)$ is also an integer. Prove that there exist integers $m$ and $n$ such that
    \begin{align*}
        f(x)=mg(x)+n.
    \end{align*}
\end{question}




\begin{question}[name={1996 Austrian--Polish}]
    Prove that there does not exist a polynomial $P(x)$ of degree $998$ such that all its coefficients are real and for all $x\in\mathbb R$, we have
    \begin{align*}
        (P(x))^2-1 = P(x^2+1).
    \end{align*}
\end{question}

\begin{question}
    For all positive integers $n>1$, prove that the following polynomial does not have rational roots:
    \begin{align*}
        P(x)= (2n+1)x^n + \cdots + 5x^2 + 3x + 1.
    \end{align*}
\end{question}

\begin{question}[name={1997 IMO Shortlist}]
    Find all positive integers $k$ such that the following statement holds true: for all polynomials $F(x)$ with integer coefficients such that $0 \leq F(c) \leq k$ for all $c \in \{0,1,2,\dots,k\}$, then
    \begin{align*}
        F(0)=F(1)=\cdots = F(k+1).
    \end{align*}
\end{question}



\begin{question}[name={1996 Romania}]
    Let $n\geq 2$ be a given integer. Find all polynomials $P(x) = a_nx^n+\cdots+a_1x+a_0$ with real non-zero coefficients such that the polynomial
    \begin{align*}
        P(x) - [P_1(x)\cdot P_2(x) \cdots P_{n-1}(x)] \text{ is the constant polynomial},
    \end{align*}
    where the polynomials $P_1,P_2,\dots,P_{n-1}$ are defined by
    \begin{align*}
        P_1(x) &= a_1x + a_0,\\
        P_2(x) &= a_2x^2+a_1x+a_0,\\
        \vdots &\phantom{=} \qquad \vdots\\
        P_{n-1}(x) &= a_{n-1}x^{n-1} + \cdots + a_1x+a_0.
    \end{align*}
\end{question}

\begin{question}[name={1999 Iran TST}]
    Given a polynomial $P(x)$ of degree $n\geq 1$ with integer coefficients and $n$ distinct integer roots such that $P(0)=0$, find all integer roots of $P(P(x))=0$.
\end{question}


\begin{question}[name={1999 Iran TST}]
    Given a real number $r\geq 0$, find all polynomials $P(x)$ with real non-negative coefficients such that
    \begin{tasks}
        \task For all $x\geq 0$, we have $P(x) \leq x^r$, and also $P(0)=0, P(1)=1$.
        \task For all $x\geq 0$, we have $P(x) \geq x^r$, and also $P(0)=0, P(1)=1$.
    \end{tasks}
\end{question}


\begin{question}[name={1990 Iran}]
    Let $\alpha$ be a root of the cubic equation $x^3-5x+3=0$ and let $f(x)$ be a polynomial with rational coefficients. Prove that if $f(\alpha)$ is a root of the same mentioned cubic equation, then $f(f(\alpha))$ is also a root of the same cubic $x^3-5x+3=0$.
\end{question}

\begin{question}
    Prove that the following polynomial does not have real roots:
    \begin{align*}
        P(x) = x^6 - x^5 + x^4 - x^3 + x^2 - x + \frac{3}{4}.
    \end{align*}
\end{question}



\begin{question}[name={1976 Bulgaria}]
    Find all polynomials $P(x)$ such that
    \begin{align*}
        P(x^2-2x)=(P(x-2))^2.
    \end{align*}
\end{question}

\begin{question}
    If $f(x)$ is a non-constant polynomial with integer coefficients, prove that one can find infinitely many prime numbers $p$ such that the modular arithmetic equation $f(x) \equiv 0 \pmod p$ has integer solutions for $x$.
\end{question}

\begin{question}[name={1998 Bulgaria}]
    Let $f(x)=x^3-3x+1$. Find the real and distinct roots of $f(f(x))=0$.
\end{question}


\begin{question}[name={1998 India}]
    For all positive integers $m\geq n \geq 2$ prove that the number of polynomials of degree $2n-1$ whose coefficients are distinct and chosen from $\{1,2,\dots,m\}$, and are also divisible by the polynomial $x^{n-1}+\cdots+x+1$, is equal to:
    \begin{align*}
        2^n n! \left(4\binom{m+1}{n+1}-3\binom{m}{n}\right).
    \end{align*}
\end{question}

\begin{question}[name={1999 Hungary}]
    Does there exist a polynomial $P(x)$ with integer coefficients so that
    \begin{align*}
        P(10)=400, \quad P(14)=440, \quad P(18)=520?
    \end{align*}
\end{question}



\begin{question}[name={1997 IMO Shortlist}]
    Let $p$ be a prime number and $f(x)$ a polynomial with integer coefficients and of degree $n$ such that:
    \begin{itemize}
        \item[(i)] $f(0)=0$ and $f(1)=1$; and
        \item[(ii)] For all positive integers $n$, $f(n) \equiv 0 \text{ or } 1 \pmod p$.
    \end{itemize}
    Prove that $d \geq p-1$.
\end{question}



\begin{question}[name={1997 Romania}]
    Let $n\geq 2$ be a given integer. Find all polynomials $P(x) = a_nx^n+\cdots+a_1x+a_0$ with positive integer coefficients such that for each $k=1,2,\dots,n-1$, we have $a_k=a_{n-k}$. Prove that there exist infinitely many pairs $(x,y)$ of positive integers for which
    \begin{align*}
        x \mid P(y) \qquad \text{and} \qquad y \mid P(x).
    \end{align*}
\end{question}


\begin{question}[name={1998 Austrian--Polish}]
    Find all pairs $(a,b)$ of positive integers such that the polynomial $x^3-17x^2+ax-b^2=0$ has three (not necessarily distinct) integer roots.
\end{question}

\begin{question}
    Let $\{a_i\}_{i=0}^{n}$ be a sequence of $n\geq 2$ real numbers with $a_n \neq 0$ and
    \begin{align*}
        a_{n-1}^2 - \frac{2n}{n-1}a_na_{n-2} < 0.
    \end{align*}
    Prove that the polynomial $P(x)$ defined below has at most $n-2$ distinct real roots:
    \begin{align*}
        P(x) = a_nx^n + a_{n-1}x^{n-1} + a_{n-2}x^{n-2} + \cdots + a_1 x + a_0.
    \end{align*}
\end{question}

\begin{question}[name={2000 IMO Shortlist}] 
%https://artofproblemsolving.com/community/c6h219754p1218960
    For a polynomial $ P$ of degree $2000$ with distinct real coefficients let $ M(P)$ be the set of all polynomials that can be produced from $ P$ by permutation of its coefficients. A polynomial $ P$ will be called $ n$-independent if $ P(n) = 0$ and we can get from any $ Q \in M(P)$ a polynomial $ Q_1$ such that $ Q_1(n) = 0$ by interchanging at most one pair of coefficients of $ Q.$ Find all integers $ n$ for which $ n$-independent polynomials exist.
\end{question}

\begin{question}[name={1995 Czech And Slovak Mathematical Olympiad}] 
%https://artofproblemsolving.com/community/c1068820h2011865p14103957
    Find all real numbers $p$ for which the equation \[x^3 -2p(p+1)x^2+(p^4 +4p^3 -1)x-3p^3 = 0,\] has three distinct real roots which are sides of a right triangle.
\end{question}

\begin{solution}[name={Solution by Amir Parvardi}]
    Let $a,b,c$ be the roots of the equation so that $a^2+b^2=c^2$. Since these are also the roots of $x^3 -2p(p+1)x^2+(p^4 +4p^3 -1)x-3p^3 = 0$, we can use Viète's formulas to obtain
    \begin{align*}
    \begin{cases}
    a+b+c &= 2p(p+1),\\
    ab+bc+ca &= p^4+4p^3-1,\\
    abc &= 3p^3.
    \end{cases}
    \end{align*}
    Squaring the first equation and subtracting twice the second equation, we get $a^2+b^2+c^2=4p^2(p+1)^2 - 2(p^4+4p^3-1)$, and since $a^2+b^2=c^2$, we may write $a^2+b^2+c^2=2c^2$ and simplify the last equation as \[2c^2 = 2 p^4 + 4 p^2 + 2,\] or simply $c^2=p^4+2p^2+1$. This simplifies to $c=\pm(p^2+1)$, and since $c$ is a side of a triangle, $c=p^2+1$. From here, it is easy to find $a$ and $b$ from equations $a+b = 2p(p+1)-c$ and $ab=3p^3/c$. Write them out explicitly, having $a\neq b$ in mind:
    \[a+b = 2p(p+1) - (p^2+1) = p^2 +2p - 1 \quad \text{and} \quad ab = \frac{3p^3}{p^2+1}.\]
    Since $a,b$, and $c$ are side-lengths of a triangle, the triangle inequality $a+b>c$ must hold: $p^2+2p-1 > p^2+1$, or simply $p>1$. Moreover, since $(p^2+1)^2=c^2=a^2+b^2=(a+b)^2-2ab$, we find that \[(p^2+1)^2 = ( p^2 +2p - 1)^2 - \frac{6p^3}{p^2+1}.\]
    This results in $(p^2+1)^3 = (p^2+1)( p^2 +2p - 1)^2 - 6p^3$, which simplifies to $-4 p^5 + 6 p^3 + 4 p = 0$. Neglecting the trivial $p=0$ solution, we can divide both sides of the latter equation by $p$ to get $-4p^4+6p^2+4=0$ which factorizes into $-2 (p^2 - 2) (2 p^2 + 1) = 0$. Therefore, the only solution that satisfies $p>1$ is $\boxed{p=\sqrt 2}$.
\end{solution}


\begin{question}[name={1995 Greece}]
    If the equation $ax^2+(c-b)x+(e-d)=0$ has real roots greater than $1$, prove that the equation $ax^4+bx^3+cx^2+dx+e=0$ has at least one real root.
\end{question}

\begin{solution}
Let $ r$ be the root of the first equation and $ f(x)$ be the second polynomial.
It is easy to verify $f(\sqrt r)=(br+d)(\sqrt r+1)$ and $f(-\sqrt r)=(br+d)(-\sqrt r+1)$.
Hence, $f(\sqrt r)f(-\sqrt r)\leq 0$. Then there is a root on $[-\sqrt r,\sqrt r]$ to $f(x)=0$.
\end{solution}


\begin{question}
    Let $m$ be a non-negative integer and let $n$ be an even positive integer. Prove that the polynomial
    \begin{align*}
        P(x) = \frac{x^n}{(n+1)^m} + \frac{x^{n-1}}{n^m} + \cdots + \frac{x}{2^m} + 1,
    \end{align*}
    does not have any real roots, but if $n$ is odd, then this polynomial has precisely one real root.
\end{question}

\begin{question}[name={1996 Austrian--Polish}]
    The sequence of $P_n$ of polynomials is defined initially by $P_0(x)=0$ and $P_1(x)=x$, and then recursively for $n\geq$, 
    \begin{align*}
        P_n(x) = x P_{n-1}(x) + (1-x)P_{n-2}(x).
    \end{align*}
    For any given positive integer $n$, find all $x$ such that $P_n(x)=0$.
\end{question}

\begin{question}[name={2000 Iran}]
    Does there exist a polynomial $f(x)$ of degree $1999$ with integer coefficients such that for all integers $n$, the numbers $f(n), f(f(n)), f(f(f(n))), \dots$ are pairwise coprime. That is, no two of them share integer divisors.
\end{question}

\begin{question}
    Find all polynomials $P(x)$ with real coefficients such that for all $x\in \mathbb R$,
    \[P(x)\cdot P(x+1) = P(x^2).\]
\end{question}



\begin{question}[name={1999 Iran}]
    Find all polynomials $P(x)$ with real coefficients for which there exists a positive integer $n$ such that for all $x\in\mathbb R$,
    \[xP(x-n)=(x-1)P(x).\]
\end{question}

\begin{question}
    Let $P(x)$ be a polynomial of degree $n$ with rational coefficients and $n$ roots $\alpha_1,\alpha_2,\dots,\alpha_n$. Prove that for all positive integers $m$, the expression $\alpha_1^m+\alpha_2^m+\cdots+\alpha_n^m$ is a rational number.
\end{question}



\begin{question}[name={1997 Iran}]
    For three integers $a,b,c$ with $a\neq 1$, we know that one of the roots of the polynomial $P(x)=x^3+ax^2+bx+c$ equals the product of the other two roots. Prove that $2P(-1)$ is divisible by $P(1)+P(-1)-2(1+P(0))$.
\end{question}



\begin{question}[name={2001 Iran Second Round}] 
%https://artofproblemsolving.com/community/c6h369944p2038254
    Find all polynomials $P(x)$ with real coefficients such that for all $x\in\mathbb R$,
    \[P(2P(x)) = 2P(P(x))+2(P(x))^2.\]
\end{question}




\begin{question}[name={1997 Iran}]
    Let $P(z)$ be a polynomial with real coefficients such that $P(0)=1$ and for all complex numbers $z$ with $|z|=1$, we have $|P(z)|=1$. Prove that $P(z) \equiv 1$.
\end{question}



\begin{question}[name={1997 Iran}]
    For two monic polynomials $P(x)$ and $Q(x)$ with rational coefficients which are both irreducible, prove that if $\alpha$ is a root of $P(x)$ and $\beta$ is a root of $Q(x)$, where $\alpha+\beta$ is rational, then the polynomial $(P(x))^2 - (Q(x))^2$ has a rational root.
\end{question}


\begin{question}[name={1993 Iran Second Round}]
    %https://artofproblemsolving.com/community/c6h379410p2098142
    %https://drive.google.com/drive/folders/1caR8dH1hnZ71U2e2REAPdI9ds7s3TOfm
    Let $f(x)$ and $g(x)$ be two polynomials with real coefficients such that for infinitely many rational values of $x$, the fraction $\frac{f(x)}{g(x)}$ is rational. Prove that $\frac{f(x)}{g(x)}$ can be written as the ratio of two polynomials with rational coefficients.
\end{question}

\begin{question}[name={1994 Iran Third Round}]
    %https://drive.google.com/drive/folders/1iWwhqp33lcAOx1RuCmcX1N18ZFLh2oTL
    Find all polynomials $f(x)$ with real roots such that for all $x\in\mathbb R$,
    \[f(x^2-1)=f(x)\cdot f(-x).\]
\end{question}

\begin{question}[name={1979 Bulgaria}]
    Find all polynomials $P(x)$ with real coefficients such that for all $x \in\mathbb R$,
    \[P(x) \cdot P(2x^2) = P(2x^3+x).\]
\end{question}


\begin{question}[name={1979 Hungary}]
    Prove that if the polynomial $P(x)$ with real coefficients is always non-negative for all $x\in\mathbb R$, then we can write
    \[P(x) = (Q_1(x))^2+(Q_2(x))^2+\cdots+(Q_n(x))^2,\]
    where $Q_1(x), Q_2(x), \dots, Q_n(x)$ are polynomials with real coefficients.
\end{question}


\begin{question}[name={1998 Bulgaria}]
    For all positive integers $n$, the two-variable polynomial $P_n(x,y)$ is defined initially by $P_1(x,y)=1$ and recursively for $n\geq 1$ by
    \[P_{n+1}(x,y) = (x+y-1)(y+1)P_n(x,y+2) + (y-y^2)P_n(x,y).\]
    Prove that for each $n\in\mathbb N$ and $x,y \in \mathbb R$,
    \[P_n(x,y) = P_n(y,x).\]
\end{question}



\begin{question}[name={1998 Canada}]
    Find all real roots of the following equation:
    \[x = \sqrt{x-\frac{1}{x}}+\sqrt{1-\frac{1}{x}}.\]
\end{question}

\begin{question}[name={1999 Japan}]
    For each positive integer $n$, prove that the polynomial
    \[f(x)=(x^2+1^2)(x^2+2^2)\cdots (x^2+n^2) + 1,\]
    cannot be written as a product of two non-constant polynomials with real coefficients.
\end{question}

\begin{question}
    Define a sequence $\{P_n\}_{n=0}^\infty$ of polynomials initially by $P_0(x)=1$ and $P_1(x)=x+1$, and recursively for $n \geq 1$ by
    \[P_{n+1}(x) = P_n(x) + xP_{n-1}(x).\]
    Prove that for all $n\in \mathbb N$, all the roots of $P_n(x)$ are real.
\end{question}



\begin{question}[name={1996 Romania}]
    For real numbers $a,b,c$ with $a\neq 0$, we know that $a$ and $4a+3b+2c$ have the same sign. Prove that the polynomial $ax^2+bx+c$ cannot have two roots in the interval $(1,2)$.
\end{question}



\begin{question}[name={1997 Iran}]
    Let $P(x)=ax^3+bx^2+cx+d$ be a polynomial with rational coefficients. If the three roots of $P(x)$ are $x_1,x_2,x_3$ such that $x_1/x_2$ is a rational number not equal to $0$ or $1$, prove that all three roots are rational.
\end{question}


\begin{question}[name={1997 Iran}]
    Find all polynomials $P(x)$ with real coefficients such that for all $x \in \mathbb R$,
    \[xP(x)P(1-x) + x^3 + 100 \geq 0.\]
\end{question}

\begin{question}[name={1981 USSR}]
    Consider the two-variable polynomial
    \[P(x,y) = 4 + x^2y^4 + x^4y^2 - 3x^2y^2.\]
    \begin{tasks}
        \task Find the smallest value that this polynomial can take.
        \task Prove that $P(x,y)$ cannot be written as a sum of squares of two-variable polynomials in $x$ and $y$.
    \end{tasks}
\end{question}


\begin{question}
    Can we find a polynomial $f(x)$ such that
    \[f(f'(x)) = 27x^6 - 27x^4 + 6x^2 + 2?\]
\end{question}

\begin{question}[name={1976 International Mathematics Olympiad}]
%https://artofproblemsolving.com/community/c6h61035p367419
    Let $P_{1}(x)=x^{2}-2$ and $P_{j}(x)=P_{1}(P_{j-1}(x))$ for $j=2,3,\dots$ Prove that for any positive integer $n$ the roots of the equation $P_{n}(x)=x$ are all real and distinct.
\end{question}


\begin{question}
    Let $P(x)$ be a polynomial of degree $7$ such that for seven distinct integer values of $x$, we have $P(x)$ equal to either $+1$ or $-1$. Prove that $P(x)$ cannot be factorized as a product of two polynomials with integer coefficients.
\end{question}


\begin{question}
    Find all polynomials $P(x)$ of the form
    \[P(x) = x^n + nx^{n-1} + a_2x^{n-2} + \cdots + a_{n-1}x + a_n,\]
    such that if $r_1,r_2,\dots,r_n$ are the roots of $P(x)$, then we have
    \[r_1^{16} + r_2^{16} + \cdots + r_n^{16}=n.\]
\end{question}



\begin{question}[name={1994 Romania}]
    Let $a,b,c$ and $A,B,C$ be positive real numbers such that the quadratic polynomials $p(x)=ax^2+bx+c$ and $P(x)=Ax^2+Bx+C$ have real roots. Prove that for any $u$ that lies between the roots of $p(x)$ and for any $U$ that lies between the roots of $P(x)$, we have
    \[(au+AU)\left(\frac{c}{u}+\frac{C}{U}\right) \leq \left(\frac{b+B}{2}\right).\]
\end{question}

\begin{question}[name={1998 Vietnam}]
    Prove that for all odd positive integers $n$, there exists a unique polynomial $P(x)$ of degree $n$ and with real coefficients such that for all real $x\neq 0$,
    \[P\left(x-\frac{1}{x}\right) = x^n - \frac{1}{x^n}.\]
    Moreover, find out when the given statement is true for even $n$.
\end{question}

\begin{question}[name={1998 Czech And Slovak}]
    Let $P(x)$ be a polynomial of degree $n\geq 5$ with integer coefficients which has $n$ distinct integer roots. If we assume that $P(0)=0$, find all integer roots of $P(P(x))$.
\end{question}

\begin{question}[name={1998 Russia}]
    Find all two-variable polynomials $P(x,y)$ such that for all $x,y\in\mathbb R$,
    \[P(x+y,x-y)=P(x,y).\]
\end{question}



\begin{question}[name={1998 Russia}]
    Does there exist a polynomial $P(x)$ with integer coefficients and a positive integer $k>1$ such that the numbers $P(k), P(k^2), P(k^3),\dots$ are pairwise coprime?
\end{question}


\begin{question}
    If $f(x)$ is a non-constant polynomial with integer coefficients such that for all primes $p$, we know that $f(p)$ is a power of a prime number, prove that there exists a positive integer $n$ such that $f(x)=x^n$.
\end{question}

\begin{question}[name={1979 Hungary}]
    Let $a\neq 0$ be a real number and define $P(x)=ax^2+bx+c$. Prove that for all positive integers $n$, there cannot be more than one polynomial $Q(x)$ of degree $n$ for which
    \[Q(P(x))=P(Q(x)).\]
\end{question}


\begin{question}[name={1983 Romania}]
    Let $\{F_n\}_{n=1}^\infty$ denote the Fibonacci sequence defined by $F_1=F_2=1$ and $F_{n+1}=F_n + F_{n-1}$ for all $n\geq 2$. We know that for the polynomial $P(x)$ of degree $990$, we have $P(k)=F_k$ for $k=992,993,\dots,1982$. Prove that \[P(1983)=F_{1983}-1.\]
\end{question}

\begin{question}[name={1977 Bulgaria}]
    Let $Q(x)$ be a non-zero polynomial. Prove that for all positive integers $n$, the polynomial $P(x)=(x-1)^nQ(x)$ has at least $n+1$ non-zero coefficients.
\end{question}

\begin{question}[name={1985 Sweden}]
    Let $P(x)$ be a polynomial of degree $n$ such that for all $x\in\mathbb R$, we have $P(x) \geq 0$. Prove, for all $x\in \mathbb R$, that
    \[P(x) + P'(x) + P''(x) + \cdots + P^{(n)}(x) \geq 0.\]
\end{question}

\begin{question}[name={2000 Iran}]
    Let $P(x)$ be a polynomial with integer coefficients. Prove that the polynomial
    \[Q(x)=P(x^4)\cdot P(x^3) \cdot P(x^2) \cdot P(x) + 1,\]
    does not have any integer roots.
\end{question}


\begin{question}[name={2003 Poland}]
    Define $W(x)=x^4-3x^3+5x^2-9x$. Find all pairs $(a,b)$ of distinct integers such that $W(a)=W(b)$.
\end{question}


\begin{question}[name={2000 Austrian--Polish}]
    Find all polynomials $P(x)$ with real coefficients that satisfy the following condition: there exists a positive integer $n$ such that the following equation holds for infinitely many real values of $x$:
    \[\sum_{k=1}^{2n+1} (-1)^k \left\lfloor\frac{k}{2}\right\rfloor P(x+k) = 0.\] 
\end{question}



\begin{question}[name={2000 Poland}]
    Let $P(x)$ be a polynomial of odd degree such that
    \[P(x^2-1)=(P(x))^2 - 1.\]
    Prove that $P(x)=x$ for all $x\in\mathbb R$.
\end{question}


\begin{question}[name={1996 Russia}]
    Prove that for all polynomials $P(x)$ of degree $10$ with integer coefficients, there exists an infinite arithmetic progression which does not contain the following numbers:
    \[\dots, P(-1), P(0), P(1), P(2), \dots\]
\end{question}


\begin{question}[name={1997 Romania}]
    Let $a_0,a_1,\dots,a_n$ be complex numbers such that for any complex $z$ with $|z|\leq 1$, we have \[|a_nz^n + a_{n-1}z^{n-1} + \cdots + a_1z + a_0| \leq 1.\] Prove that for all $k=0,1,\dots,n$, we have $|a_k| \leq 1$ and \[|a_0+a_1+\cdots+a_n-(n+1)a_k| \leq n.\]
\end{question}


\begin{question}[name={1994 Vietnam}]
    Let $P(x)$ be a polynomial of degree $4$ with $4$ positive real roots. Prove that the polynomial
    \[\frac{1-4x}{x^2}P(x) + \left(1 - \frac{1-4x}{x^2}\right)P'(x) - P''(x)\]
    also has $4$ positive real roots.
\end{question}


\begin{question}
    Let $P(x)$ be a quadratic polynomial such that for a sequence of rational numbers $q_0,q_1,q_2,\dots$ we have $q_n = P(q_{n+1})$ for all $n \geq 1$. Prove that there exists a positive integer $k$ such that for all $n\geq 1$, we have $q_{n+k}=q_n$.
\end{question}


\begin{question}[name={1994 China}]
    For all polynomials $f(x)=x_0x^n + c_1x^{n-1} + \cdots + c_n$ of degree $n$ with complex coefficients, prove that there exists a complex number $x_0$ such that $|x_0| \leq 1$ and \[|f(x_0)| \geq |c_0| + |c_n|.\]
\end{question}


\begin{question}
    Let $f(x)$ be a polynomial with rational coefficients such that for some $\alpha \in \mathbb R$,
    \[\alpha^3 - 1992 \alpha + 33 = (f(\alpha))^3 - 1992 f(\alpha) + 33 = 0.\]
    Prove that for all $n\geq 1$, we have
    \[(f^n(\alpha))^3 - 1992 f^n(\alpha) + 33=0,\]
    where $f^n(\alpha) = \underbrace{f(f(\cdots f(\alpha)))}_{n \text{ times}}$.
\end{question}


\begin{question}
    Prove that if a symmetric two-variable polynomial $P(x,y)$ is divisible by $x-y$, then $P(x,y)$ is also divisible by $(x-y)^2$.
\end{question}

\begin{question}
    Find all polynomials $P(x)$ with $P(0)=0$ and
    \[P(x^2+1) = (P(x))^2 + 1.\]
\end{question}


\begin{question}[name={1986 Czech And Slovak}]
%https://artofproblemsolving.com/community/c1068820h2008975p14074707
    Let $P(x)$ be a polynomial with integer coefficients of degree $n \ge 3$. If $x_1,x_2,\dots,x_m$ (with $m \ge 3$) are different integers such that \[P(x_1) = P(x_2) = \dots = P(x_m) = 1,\] prove that $P$ cannot have integer roots.
\end{question}


\begin{question}
    Define polynomial $P(x)$ of degree $n$ with integer coefficients by
    \[P(x) = a_0x^n + a_1x^{n-1} + \cdots + a_{n-1}x + a_n.\]
    If for two real numbers $\alpha > \beta$ we have $|P(\alpha)|=|P(\beta)|=1$, and $P(x)$ has a rational root $r$, then prove that $\alpha-\beta$ equals either $1$ or $2$, and that $r=(\alpha+\beta)/2$.
\end{question}


\begin{question}
    Prove that there does not exist a polynomial $P$ such that $P(x)$ is prime for all $x\in \{0,1,2,\dots\}$.
\end{question}


\begin{question}
    Let $f(x)$ be a non-constant polynomial with integer coefficients such that $f(0)>0$. Prove that there exists a sequence $p_1,p_2,p_3,\dots$ of prime numbers and a sequence $a_1,a_2,a_3,\dots$ of pairwise coprime positive integers such that for all $n=1,2,3,\dots$, we have \[f(p_n) = a_1a_2\cdots a_n.\]
\end{question}



\begin{question}[name={1994 Iran First Round}]
    Let $a,b,c$ be real numbers such that $9a+11b+29c=0$. Prove that the cubic polynomial $ax^3+bx+c$ has a root in the interval $[0,2]$.
\end{question}


\begin{question}[name={1998 Bulgaria}]
    Find all positive integers $n$ such that $x^n+64$ can be factorized into a product of two polynomials with integer coefficients.
\end{question}



\begin{question}[name={1998 India}]
    Let $N$ be a positive integer such that $N+1$ is a prime. Assume that for $i=0,1,2,\dots,N$, we have $a_i \in \{0,1\}$ and not all $a_i$ are equal to each other. Define the polynomial $f(x)$ so that for each $i=0,1,2,\dots,N$, we have $f(i)=a_i$. Show that the degree of $f(x)$ is at least $N$.
\end{question}



\begin{question}[name={1998 Romania}]
    For all positive integers $n$, prove that the polynomial
    \[f(x)=(x^2+x)^{2^n}+1,\]
    cannot be factorized into a product of two non-constant polynomials with integer coefficients.
\end{question}


\begin{question}[name={1999 China}]
% https://artofproblemsolving.com/community/c6h470403p2634003
    Let $a$ be a real number. Let $(f_n(x))_{n\ge 0}$ be a sequence of polynomials such that $f_0(x)=1$ and $f_{n+1}(x)=xf_n(x)+f_n(ax)$ for all non-negative integers $n$.
    \begin{tasks}
        \task Prove that \[f_n(x)=x^nf_n\left(x^{-1}\right),\] for all non-negative integers $n$.
        \task Find an explicit expression for $f_n(x)$.
    \end{tasks}
\end{question}


\begin{question}[name={1999 Hungary}]
    If the polynomial $x^4-2x^2+ax+b$ has four distinct real roots, prove that the absolute value of each of its roots is less than $\sqrt 3$.
\end{question}



\begin{question}[name={1999 Romania}]
    Let $a$ and $n$ be integers and $p$ be a prime number such that $p>|a|+1$. Prove that the polynomial $f(x)=x^n+ax+p$ cannot be factorized into a product of two polynomials with integer coefficients.
\end{question}



\begin{question}[name={1999 Ukraine}]
    Let $P(x)$ be a polynomial with integer coefficients and assume that the sequence $\{x_n\}_{n=1}^{n=2000}$ satisfies $x_1=x_{2000}=1999$. If we know that $x_{n+1}=P(x_n)$, then find the value of the following sum:
    \[\frac{x_1}{x_2}+\frac{x_2}{x_3}+\cdots+\frac{x_{1999}}{x_{2000}}.\]
\end{question}



\begin{question}[name={1975 International Mathematics Olympiad}]
    Determine all two-variable polynomials $P(x,y)$ so that:
    \begin{tasks}
        \task For any real numbers $t,x,y$ we have $P(tx,ty) = t^n P(x,y)$ where $n$ is a positive integer, the same for all $t,x,y$;
        \task For any real numbers $a,b,c$ we have \[P(a + b,c) + P(b + c,a) + P(c + a,b) = 0;\]
        \task $P(1,0) =1$.
    \end{tasks}
\end{question}


\begin{question}
    Let $p$ be an odd prime number. Prove that the polynomial
    \[\sum_{1 \leq m,n \leq p-1} x^{mn} + p -1,\]
    is divisible by $x^{p-1} + x^{p-2} + \cdots + x + 1$.
\end{question}

\begin{question}
    Find all polynomials $P(x)$ such that
    \[P(x^2) + P(x)P(x+1)=0.\]
\end{question}

\begin{question}[name={1997 Germany}]
    Define $f(x)$ and $g(x)$ by
    \begin{align*}
        f(x) &= x^5 + 5x^4 + 5x^3 + 5x^2 + 1,\\
        g(x) &= x^5 + 5x^4 + 3x^3 - 5x^2 - 1.
    \end{align*}
    Find all prime numbers $p$ for which there exists an integer $x$ with $0 \leq x \leq p$ such that both $f(x)$ and $g(x)$ are both divisible by $p$. Moreover, for each such $p$, find all $x$ that satisfy the condition.
\end{question}


\begin{question}[name={1997 Ukraine}]
% https://artofproblemsolving.com/community/c6h290562p1570720
    If we know that $ax^3+bx^2+cx+d$ has three distinct real roots, then how many root does the following equation have?
    \[4(ax^3+bx^2+cx+d)(3ax+b) = (3ax^2+2bx+c)^2.\]
\end{question}

\begin{question}[name={1997 British Math Olympiad}]
    Find all polynomials $P(x)$ of degree $5$ with distinct coefficients chosen from the set $\{1,2,3,\dots,9\}$ such that $P(x)$ is divisible by $x^2-x+1$.
\end{question}


\begin{question}
    Let $f(x)$ and $g(x)$ be single-variable polynomials with real coefficients and let $P(x,y)$ be a two-variable polynomial with real coefficients such that for all real numbers $x,y$, \[f(x)-f(y)=(g(x)-g(y)) \cdot P(x,y).\] Prove that there exists a single-variable polynomial $h(x)$ with real coefficients such that $f(x)=h(g(x))$ for all real numbers $x$.
\end{question}


\begin{question}
    Let $P(x)$ be a polynomial with real coefficients which satisfies the following inequalities:
    \begin{align*}
        P(0) >0, \quad P(1)>P(0), \quad P(2) > 2P(1) - P(0), \quad P(3) > 3P(2) - 3P(1) + P(0).
    \end{align*}
    Moreover, for each positive integer $n$, we know that
    \[P(n+4) > 4P(n+3) - 6P(n+2) + 4P(n+1) - P(n).\]
    Prove that $P(n)>0$ for all positive integers $n$.
\end{question}


\begin{question}[name={1995 Ireland}]
%https://artofproblemsolving.com/community/c6h286218p1546631
    Let $a,b,c$ be complex numbers such that all roots $z$ of the polynomial
    \[P(x)=x^3+ax^2+bx+c,\] satisfy the equation $|z|=1$. Prove that all roots $\omega$ of the polynomial \[Q(x)=x^3 + |a|x^2 + |b|x + |c|,\] also satisfy $|\omega|=1$.
\end{question}


\begin{question}[name={1995 Japan}]
    Let $k,n$ be integers such that $1\leq k\leq n,$ and let $a_1, a_2, \dots, a_k$ be numbers satisfying the following equations:
    \[ \begin{cases} a_1+a_2+\cdots+a_k &= n, \\ a_1^2+a_2^2 +\cdots +a_k^2 &= n,\\ \qquad \vdots &\phantom{=} \vdots \\ a_1^k+a_2^k+\cdots+a_k^k &= n. \end{cases} \]
    Prove that \[(x+a_1)(x+a_2)\cdots(x+a_k) = x^k+\binom{n}{1} x^{k-1} + \binom{n}{2} x^{k-2}+\cdots+ \binom{n}{k}.\]
\end{question}


\begin{question}
    For each positive integer $n$, define \[f(n) = 1! + 2! + \cdots + n!.\] Find polynomials $P(x)$ and $Q(x)$ such that for all positive integers $n$,
    \[f(n+2) = P(n)f(n+1) + Q(n) f(n).\]
\end{question}


\begin{question}
    Find all polynomials $P(x)$ such that
    \[1+ P(x) = \frac{P(x-1)+P(x+1)}{2}.\]
\end{question}


\begin{question}
    Is it possible to factorize $P(x)=x^{100}+5x^{99}+2x+2$ into a product of two polynomials with integer coefficients?
\end{question}


\begin{question}
    If $P$ and $Q$ are two polynomials such that for all $x\in\mathbb R$, \[P(x^2+x+1)=Q(x^2-x+1),\] prove that $P$ and $Q$ are constant polynomials.
\end{question}

\begin{question}
    Find all polynomials $P(x)$ with real coefficients such that $P(x)$ has distinct real roots $r_1>r_2>\dots > r_n$ and also, $(r_i+r_{i+1})/2$ are roots of $P'(x)$ for $i=1,2,\dots,n-1$.
\end{question}


\begin{question}[name={1997 Bulgaria}]
% https://artofproblemsolving.com/community/c6h601000p3567947
    For integer $n \geq 2$, consider the polynomial
    \[P_n(x) = \binom {n}{2}+\binom {n}{5}x+\binom {n}{8}x^2 + \cdots + \binom {n}{3k+2}x^{3k}, \quad \text{where} \quad k = \left\lfloor \frac{n-2}{3} \right \rfloor.\]
    \begin{tasks}
        \task Prove that $P_{n+3}(x)=3P_{n+2}(x)-3P_{n+1}(x)+(x+1)P_n(x)$.
        \task Find all integers $a$ such that $P_n(a^3)$ is divisible by $3^{ \lfloor \frac{n-1}{2} \rfloor}$ for all $n \ge 3$.
    \end{tasks}
\end{question}

\begin{question}
    Find all two-variable polynomials $P(x,y)$ with real coefficients such that for all $x,y\in\mathbb R$, \[P(x,y)=P(x+1,y+1).\]
\end{question}

\begin{question}
    Let $p(x)$ and $q(x)$ be non-zero polynomials such that for all $x\in \mathbb R$, \[p(x^2+x+1)=p(x)\cdot q(x).\] Prove that the degree of $p(x)$ is even.
\end{question}

\begin{question}[name={1999 China}]
% https://artofproblemsolving.com/community/c6h470401p2633999
    Determine the maximum value of $\lambda$ such that if $f(x) = x^3 +ax^2 +bx+c$ is a cubic polynomial with all its roots non-negative, then
    \[f(x)\geq\lambda(x -a)^3,\] for all $x\geq0$. Find the equality condition.
\end{question}

\begin{question}[name={1999 Poland}]
    Let $P(x)=2x^3-3x^2+2$ and define the sets
    \begin{align*}
        S &= \{P(n) \ | \ n \in \mathbb N, n \leq 999\},\\
        T &= \{n^2+1 \ | \ n \in \mathbb N\},\\
        U &= \{n^2+2 \ | \ n \in \mathbb N\}.
    \end{align*}
    Prove that the sets $S \cap T$ and $S \cap U$ have the same number of elements.
\end{question}


\begin{question}[name={1999 Vietnam}]
    Let $a$ and $b$ be real numbers such that all the roots of the following polynomial are positive real numbers: \[P(x)=ax^3-x^2+bx-1.\]
    Find the least value of the fraction \[\frac{5a^2-3ab+2}{a^2(b-a)}.\]
\end{question}


\begin{question}[name={1995 Korea}]
% https://artofproblemsolving.com/community/c6h1677731p10689433
    Let $a$ and $b$ be integers and $p$ be a prime number such that:
    \begin{itemize}
        \item[(i)] $p$ is the greatest common divisor of $a$ and $b$; and
        \item[(ii)] $p^2$ divides $a$.
    \end{itemize}
    Prove that the polynomial $x^{n+2}+ax^{n+1}+bx^{n}+a+b$ cannot be decomposed into the product of two polynomials with integer coefficients and degree greater than $1$.
\end{question}


\begin{question}
    For a monic polynomial $P(x)$ of degree $n$ with non-negative coefficients and $n$ real roots, we have $P(0)=1$. Prove that for all integers $k$, \[P(k) \geq (k+1)^n.\]
\end{question}


\begin{question}
    Let $P(x)$ be a polynomial with integer coefficients such that for all primes $q$, we know that $P(q)$ is a power of $2$. Prove that $P(x)$ must be a constant polynomial.
\end{question}


\begin{question}
    Let $P(x)$ and $Q(x)$ be polynomials with real coefficients such that either $P(x)$ and $Q(x)$ are both integers or they are both non-integers. Prove that $P(x) = \pm Q(x)$.
\end{question}


\begin{question}[name={2000 Romania TST}]
% https://artofproblemsolving.com/community/c6h44880p284098
    Let $P,Q$ be two monic polynomials with complex coefficients such that $P(P(x))=Q(Q(x))$ for all $x$. Prove that $P=Q$.
\end{question}

\begin{question}[name={1986 USSR}]
    If the roots of the quadratic polynomial \[P(x)=x^2+ax+b+1,\] are positive integers, prove that $a^2+b^2$ is a composite number.
\end{question}


\begin{question}
    The value of polynomial $P(x)$ is a perfect square for all positive integers $x$. Prove that there must exist a polynomial $Q(x)$ such that $P(x)=(Q(x))^2$.
\end{question}


\begin{question}[name={1998 Iran}]
    Prove that for any non-constant polynomial $f(x)$ with integer coefficients, there exists a sequence $p_1<p_2<p_3<\cdots$ of primes and a sequence $n_1<n_2<n_3<\cdots$ of positive integers such that $p_k \mid f(n_k)$ for all $k \in \mathbb N$.
\end{question}


\begin{question}[name={1996 Taiwan}]
% https://artofproblemsolving.com/community/c6h128152p726773
    Show that for any real numbers $a_{3},a_{4},\dots,a_{85}$, not all the roots of the equation 
    \[a_{85}x^{85}+a_{84}x^{84}+\cdots+a_{3}x^{3}+3x^{2}+2x+1=0,\] are real roots.
\end{question}



\begin{question}[name={1998 Iran}]
    The sequence $a_0,a_1,a_2,\dots$ satisfies $2a_i = a_{i-1} + a_{i+1}$. Define the polynomial $P_n(x)$ for each $n\in\mathbb N$ by
    \[P_n(x) = \sum_{i=0}^n a_i \binom{n}{i} x^i (1-x)^{n-i}.\]
    Prove that $P_n(x)$ is linear for all $n\in \mathbb N$.
\end{question}



\begin{question}[name={1995 Austrian--Polish}]
% https://artofproblemsolving.com/community/c1068820h2085506p15032469
    Let $P(x)=x^4+x^3+x^2+x+1$ and prove that there exist non-constant polynomials $Q(x)$ and $R(x)$ with integer coefficients such that for all $x\in\mathbb R$,
    \[Q(x) \cdot R(x) = P(5x^2).\]
\end{question}


\begin{question}[name={1995 Austrian--Polish}]
% https://artofproblemsolving.com/community/c1068820h2085518p15032588
    Find all polynomials $P(x)$ with real coefficients such that for all $x\neq 0$,
    \[(P(x))^2 + \left(P\left(\frac{1}{x}\right)\right)^2 = P(x^2) \cdot P\left(\frac{1}{x^2}\right).\]
\end{question}


\begin{question}[name={1995 Balkan}]
% https://artofproblemsolving.com/community/c6h85168p495437
    Let $a$ and $b$ be positive integers with $a > b$ and having the same parity. Prove that the solutions of the equation \[ x^2 - (a^2 - a + 1)(x - b^2 - 1) - (b^2 + 1)^2 = 0, \] are positive integers, none of which is a perfect square.
\end{question}



\begin{question}[name={1998 Iran}]
    The determinant of the cubic polynomial $P(x)=x^3+ax^2+bx+c$ is defined by \[\Delta = 18abc - 4a^3c + a^2b^2 -4b^3 - 27c^3.\]
    Prove that if $\Delta \geq 0$, then $P(x)$ will have real roots.
\end{question}


\begin{question}[name={1998 Iran}]
    Find the smallest positive integer $d$ for which there exists a monic polynomial of degree $d$ such that for all $n\in\mathbb N$, we have $100 \mid f(n)$.
\end{question}

\begin{question}
    Let $P(x)$ be a polynomial with integer coefficients such that $P(0)=P(1)=1$. For an arbitrary integer $a_0$, define $a_{n+1} = P(a_n)$ for all integers $n \geq 0$. Prove that the elements of the sequence $\{a_i\}_{i=0}^\infty$ are pairwise coprime.
\end{question}

\begin{question}[name={1977 USSR}]
    Two monic polynomials $P(x)$ and $Q(x)$ are \textit{commutable} if we have $P(Q(x))=Q(P(x))$ for all real $x$. 
    \begin{tasks}
        \task For any real $\alpha$, find all monic polynomials $Q(x)$ of maximum degree $3$ which are commutable with the polynomial $P(x)=x^2-\alpha$.
        \task For an arbitrary quadratic polynomial $P(x)$ and a positive integer $k$, prove that there exists at most one polynomial of degree $k$ which is commutable with $P(x)$.
        \task Find polynomials of degree $4$ and $8$ which are commutable with a given quadratic polynomial.
        \task Let $Q(x)$ and $R(x)$ be polynomials that are both commutable with a given quadratic polynomial $P(x)$. Prove that $Q(x)$ and $R(x)$ are commutable.
        \task Let $P_2(x)=x^2-2$ and for all positive integers $k$, let $P_k(x)$ be a polynomial of degree $k$. Prove that there exist an infinite sequence $P_2(x), P_3(x), P_4(x),\dots$ of polynomials each two of which are commutable.
    \end{tasks}
\end{question}


\begin{question}[name={1996 IMO Shortlist}]
% https://artofproblemsolving.com/community/c6h219650p1218556
    Let $P(x)$ be the cubic real-coefficient polynomial \[P(x) = ax^3 + bx^2 + cx + d.\] Prove that if $ |P(x)| \leq 1$ for all $ x$ such that $ |x| \leq 1,$ then, \[ |a| + |b| + |c| + |d| \leq 7.\]
\end{question}


\begin{question}[name={1996 Poland}]
    The polynomial $P(x)$ of degree $n$ satisfies the following equation:
    \[P(k)=\frac{1}{k},\qquad \text{for} \quad k=2^0,2^1,2^2,\dots,2^n.\]
    Find $P(0)$.
\end{question}



\begin{question}[name={1996 Poland}]
    Let $P(x)$ be a non-constant polynomial with integer coefficients, and let $m \geq 1$ be a given integer. Prove that if $P(x)$ has at least three distinct integer roots, then $P(x)+5^m$ will have at least one integer root.
\end{question}

\begin{question}[name={1998 Baltic Way}]
% https://artofproblemsolving.com/community/c6h385994p2143633
    Let $P$ be a polynomial of degree $6$ and let $a,b$ be real numbers such that $0<a<b$. Suppose that $P(a)=P(-a),P(b)=P(-b),P'(0)=0$. Prove that $P(x)=P(-x)$ for all real $x$.	
\end{question}


% \begin{solution}[name={1998 Baltic Way Solution by pco}]
%     $P(a)=P(-a)$ $\implies$ $P(x)=(x^2-a^2)(u_4x^4+u_3x^3+u_2x^2+u_1x+u_0)+c$

% $P'(0)=0$ $\implies$ $u_1=0$ and $P(x)=(x^2-a^2)(u_4x^4+u_3x^3+u_2x^2+u_0)+c$

% $P(b)=P(-b)$ and $b\ne 0$ and $b\ne a$ $\implies$ $u_4b^4+u_3b^3+u_2b^2+u_0=u_4b^4-u_3b^3+u_2b^2+u_0$ and so $u_3=0$

% And so $P(x)=(x^2-a^2)(u_4x^4+u_2x^2+u_0)+c$ is even.
% \end{solution}

\begin{question}[name={1995 UNESCO}]
    Let $p\geq 2$ and $a_0,a_1,\dots,a_n$ be non-negative integers and define $f(x)=a_0+a_1x+a_2x^2+\cdots+a_nx^n$. Prove that if the numbers 
    \[\sqrt[p]{f(0)}, \sqrt[p]{f(1)}, \sqrt[p]{f(2)}, \dots\]
    are all rational, then there exists a polynomial $g(x)$ with integer coefficients such that $f(x)=(g(x))^p$. 
\end{question}


\begin{question}[name={1995 Russia}]
    Let $f,g,h$ be quadratic polynomials. Is it possible for $x=1,2,\dots,8$ to be the roots of the equation $f(g(h(x)))=0$?
\end{question}

\begin{question}[name={1998 Iran}]
    Let $P(x)$ and $Q(x)$ be two polynomials with complex coefficients and let $a,b\geq 2$ be integers such that for all $x\in\mathbb R$,
    \[\left(P(x)\right)^a-\left(Q(x)\right)^b = x.\]
    Prove that $a=b=2$.
\end{question}

\begin{question}[name={1983 IMO Longlist}]
% https://artofproblemsolving.com/community/c6h366068p2013677
    Let $p$ and $q$ be integers. Show that there exists an interval $I$ of length $1/q$ and a polynomial $P$ with integral coefficients such that
    \[ \left|P(x)-\frac pq \right| < \frac{1}{q^2},\] for all $x \in I.$
\end{question}

\begin{question}[name={1998 Poland}]
    Let $n\geq 2$ be a positive integer. Find all polynomials \[P(x)=a_0+a_1x+\cdots+a_nx^n,\] with $n$ real roots all less than or equal to $-1$, and such that
    \[a_0^2+a_1a_n=a_n^2+a_{0}a_{n-1}.\]
\end{question}

\begin{question}[name={1998 Baltic Way}]
% https://artofproblemsolving.com/community/c6h385988p2143623
    Let $P$ be a polynomial with integer coefficients. Suppose that for $n=1,2,3,\ldots ,1998$ the number $P(n)$ is a three-digit positive integer. Prove that the polynomial $P$ has no integer roots.	
\end{question}


\begin{question}[name={1996 IMO Shortlist}]
% https://artofproblemsolving.com/community/c6h91052p532496
    Let $a_{1}, a_{2},\dots, a_{n}$ be non-negative reals, not all zero. Show that that
    \begin{tasks}
        \task The polynomial \[p(x) = x^{n} - a_{1}x^{n - 1} + \cdots - a_{n - 1}x - a_{n},\] has precisely 1 positive real root $R$.
        \task Let \[A = \sum_{i = 1}^n a_{i} \qquad \text{and} \qquad B = \sum_{i = 1}^n ia_{i}.\] Show that $A^{A} \leq R^{B}$.
    \end{tasks}
\end{question}


\begin{question}[name={1997 Romania}]
    Find all polynomials $f(x)$ with integer coefficients such that $f(x)$ is bijective (that is, both injective and surjective) and for some real constant $a$ and all $x$, \[(f(x))^2 = f(x^2)-2f(x)+a.\]
\end{question}

\begin{question}[name={1995 Romania}]
    Let $m,n\geq 2$ be integers. Find the number of polynomials of degree $2n-1$ with distinct coefficients from the set $\{1,2,\dots,m\}$ which are divisible by $x^{n-1}+\cdots+x+1$.
\end{question}

\begin{question}[name={1995 Romania}]
    Let $f(x)$ be an irreducible monic polynomial with integer coefficients and of an odd degree greater than $3$. Assume that the absolute value of the roots of $f(x)$ are greater than $1$ and $f(0)$ is a square-free number (that is, not divisible by square of anything). Prove that the polynomial $g(x)=f(x^3)$ cannot be decomposed into a product of two polynomials with integer coefficients.
\end{question}

\begin{question}[name={1998 Iran}]
    Find all polynomials $P(x)$ with complex coefficients such that
    \[P(2x^2-1) = \frac{(P(x))^2}{2}-1.\]
\end{question}


\begin{question}[name={1998 Iran}]
    Let $a_0,a_1,\dots,a_n$ be real numbers such that
    \[0< a_0 < - \sum_{k=0}^{\lfloor n/2 \rfloor} \frac{a_{2k}}{2k+1}.\]
    Prove that the polynomial $P(x)=a_0+a_1x+\cdots+a_n x^{n}$ has a real root in the interval $[-1,1]$.
\end{question}

\begin{question}
    Let $P(x,y)$ be a two-variable polynomial. Prove or disprove the following statement: the inequality \[|x^y-y^x|\leq |P(x,y)|,\] has only a finite number of solutions $(x,y)$ in which $x$ and $y$ are distinct integers with $x,y \geq 2$.
\end{question}

\begin{question}[name={1988 IMO Longlist}]
% https://artofproblemsolving.com/community/c6h59463p361266
This problem comes in four questions:
\begin{tasks}
    \task The polynomial \[x^{2 \cdot k} + 1 + (x+1)^{2 \cdot k},\] is not divisible by $x^2 + x + 1.$ Find the value of $k$.
    \task If $p,q$ and $r$ are distinct roots of $x^3 - x^2 + x - 2 = 0$ the find the value of $p^3 + q^3 + r^3$.
    \task If $r$ is the remainder when each of the numbers 1059, 1417 and 2312 is divided by $d,$ where $d$ is an integer greater than one, then find the value of $d-r$.
    \task What is the smallest positive odd integer $n$ such that the product of
    \[2^{\frac{1}{7}}, 2^{\frac{3}{7}}, \dots, 2^{\frac{2 \cdot n + 1}{7}},\]
    is greater than $1000$?
\end{tasks}
\end{question}


\begin{question}[name={1988 IMO Longlist}]
% https://artofproblemsolving.com/community/c6h57269p352652
Let $n$ be a positive integer. Find the number of odd coefficients of the polynomial \[ u_n(x) = (x^2 + x + 1)^n.\]
\end{question}

\begin{question}[name={1998 Baltic Way}]
% https://artofproblemsolving.com/community/c6h385998p2143638
    Let $P_k(x)=1+x+x^2+\ldots +x^{k-1}$. Show that
    \[ \sum_{k=1}^n \binom{n}{k} P_k(x)=2^{n-1} P_n \left( \frac{x+1}{2} \right), \]
    for every real number $x$ and every positive integer $n$.
\end{question}


\begin{question}[name={1998 Iran}]
    Let $P(x)=a_nx^n+\cdots+a_1x+a_0$ with $n\geq 2$ and positive real coefficients $a_i$ such that all the roots of $P(x)$ are positive real numbers in the interval $(0,1)$. Prove that if $0\leq k \leq n-2$, then
    \[\sum_{i=k}^{n-2} \binom{i}{k} a_i > 0.\]
\end{question}



\begin{question}[name={1998 Iran}]
    Let $P(x)$ be a polynomial with rational coefficients such that for any rational $r$, there exists rational $s$ such that $P(s)=r$. Prove that $P(x)$ is linear.
\end{question}



\begin{question}[name={1998 Iran}]
    For two polynomials $f$ and $g$ with rational coefficients, if $\alpha_1,\alpha_2,\dots,\alpha_n$ are the roots of $f$, and we have
    \[g(\alpha_1)=g(\alpha_2)=\cdots = g(\alpha_n) = A,\] then prove that $A$ is a rational number.
\end{question}




\begin{question}[name={1998 Iran}]
    Define \[f(x)=1-x+x^2-x^3+\cdots+x^{16}-x^{17}.\]
    Let $y=x+1$ and $f(y)=a_0+a_1y+\cdots+a_{17}y^{17}$. Find the coefficients $a_0,a_1,\dots,a_{17}$.
\end{question}



\begin{question}[name={1998 Iran}]
    Let $P(x)=a_nx^n+\cdots+a_1x+a_0$ be a polynomial with integer coefficients and $a_0\neq 0$. If we know that \[|a_{n-1}| > 1 + |a_{n-2} + \cdots + |a_1| + |a_0|,\] prove that $P(x)$ is irreducible.
\end{question}



\begin{question}[name={1998 Iran}]
    Prove that for any prime $p$ with decimal representation \[p=(a_na_{n-1}\cdots a_1a_0)_{10},\] the polynomial $f(x)=a_nx^n+\cdots+a_1x+a_0$, is irreducible.
\end{question}



\begin{question}[name={1998 Iran}]
    Let $f$ be a polynomial with real coefficients among which $2m$ consecutive coefficients (except for the first and last ones) are zero. Prove that $f$ has at least $2m$ real roots.
\end{question}



\begin{question}[name={1998 Iran}]
    Let $f$ be a polynomial with real coefficients and four of its consecutive coefficients form an arithmetic progression. Prove that $f$ has at least one non-real root.
\end{question}



\begin{question}[name={1995 Taiwan}]
% https://artofproblemsolving.com/community/c6h128784p730209
    Let $P(x)=a_{0}+a_{1}x+\cdots+a_{n}x^{n}\in\mathbb{C}[x]$, where $a_{n}=1$. The roots of $P(x)$ are $b_{1},b_{2},\dots,b_{n}$, where $|b_{1}|,|b_{2}|,\dots,|b_{j}|>1$ and $|b_{j+1}|,\dots,|b_{n}|\leq 1$. Prove that \[\prod_{i=1}^{j} |b_{i}| \leq \sqrt{|a_{0}|^{2}+|a_{1}|^{2}+\cdots+|a_{n}|^{2}}.\]
\end{question}



\begin{question}[name={1995 Taiwan}]
% https://artofproblemsolving.com/community/c6h128813p730311
    Let $m_{1},m_{2},\dots,m_{n}$ be mutually distinct integers. Prove that there exists a $f(x)\in\mathbb{Z}[x]$ of degree $n$ satisfying the following two conditions:
    \begin{tasks}
        \task $f(m_{i})=-1$, for all $i=1,2,\dots,n$; and
        \task $f(x)$ is irreducible.
    \end{tasks}
\end{question}



\begin{question}[name={2003 APMO}]
    Let $a,b,c,d,e,f$ be real numbers such that the polynomial
    \[ p(x)=x^8-4x^7+7x^6+ax^5+bx^4+cx^3+dx^2+ex+f, \]
    factorises into eight linear factors $x-x_i$, with $x_i>0$ for $i=1,2,\dots,8$. Determine all possible values of $f$.
\end{question}



\begin{question}[name={1992 Baltic Way}]
% https://artofproblemsolving.com/community/c6h258900p1409834
    A polynomial $f(x)=x^3+ax^2+bx+c$ is such that $b<0$ and $ab=9c$. Prove that the polynomial $f$ has three different real roots.	
\end{question}


\begin{question}[name={1992 Baltic Way}]
% https://artofproblemsolving.com/community/c6h258901p1409836
    Find all quartic (fourth-degree) polynomial $p(x)$ such that the following four conditions are satisfied:
    \begin{itemize}
        \item[(i)]  $p(x)=p(-x)$ for all $x$,
        \item[(ii)] $p(x)\ge0$ for all $x$,
        \item[(iii)] $p(0)=1$,
        \item[(iv)] $p(x)$ has exactly two local minimum points $x_1$ and $x_2$ such that $|x_1-x_2|=2$.
    \end{itemize}
\end{question}


\begin{question}[name={1996 Baltic Way}]
% https://artofproblemsolving.com/community/c6h397503p2210878
    Real numbers $x_1,x_2,\dots ,x_{1996}$ have the following property: For any polynomial $W$ of degree $2$ at least three of the numbers $W(x_1),W(x_2),\dots ,W(x_{1996})$ are equal. Prove that at least three of the numbers $x_1,x_2,\dots ,x_{1996}$ are equal.
\end{question}


\begin{question}[name={1997 Baltic Way}]
% https://artofproblemsolving.com/community/c6h388799p2160094
    Let $P$ and $Q$ be polynomials with integer coefficients. Suppose that the integers $a$ and $a+1997$ are roots of $P$, and that $Q(1998)=2000$. Prove that the equation $Q(P(x))=1$ has no integer solutions.	
\end{question}


\begin{question}
    Find all polynomials $P$ for which $P(x^2) = P(x) \cdot P(x-1)$.
\end{question}

\begin{question}[name={1963 Dutch Mathematical Olympiad}]
    % https://artofproblemsolving.com/community/c1068820h3006010p26997913
    One considers for $n > 2$ the polynomial: $$(x^2-x+1)^n - (x^2-x+2)^n+ (1+x)^n+(2-x)^n.$$ Show that the degree of this polynomial is $2n - 2$.
    Moreover, assume that the polynomial is written in the form $$a_0+a_1x+a_2x^2+\cdots +a_{2n-2}x^{2n-2}.$$ Prove that $a_2+a_3+\cdots+a_{2n-2}=0$
\end{question}

\begin{question}[name={1970 Dutch Mathematical Olympiad}]
% https://artofproblemsolving.com/community/c1068820h3004127p26980440
    The equation$ x^3 - x^2 + ax - 2^n = 0$ has three integer roots. Determine $a$ and $n$.
\end{question}

\begin{question}[name={1990 Dutch Mathematical Olympiad}]
% https://artofproblemsolving.com/community/c6h285508p1542469
    A polynomial $f(x)=ax^4+bx^3+cx^2+dx$ with $ a,b,c,d>0$ is such that $ f(x)$ is an integer for $ x \in \{ -2,-1,0,1,2 \}$ and $ f(1)=1$ and $ f(5)=70$.
    \begin{tasks}
        \task Show that \[a=\frac{1}{24}, \quad b=\frac{1}{4},\quad c=\frac{11}{24}, \quad d=\frac{1}{4}.\]
        \task Prove that $f(x)$ is an integer for all $ x \in \mathbb{Z}$.
    \end{tasks}
\end{question}


\begin{question}[name={2001 Dutch Mathematical Olympiad}]
% https://artofproblemsolving.com/community/c6h1919098p13157797
    The function is given
    \[f(x) = \frac{2x^3 -6x^2 + 13x + 10}{2x^2 - 9x}.\]
    Determine all positive integers $x$ for which $f(x)$ is an integer.
\end{question}

\begin{question}[name={1996 Belgium Flanders}]
% https://artofproblemsolving.com/community/c6h53675p336219
    Consider a real polynomial $p(x)=a_nx^n+\cdots+a_1x+a_0$.
    \begin{tasks}
        \task If $\deg(p(x))>2$ prove that $\deg(p(x)) = 2 + \deg(p(x+1)+p(x-1)-2p(x))$.
        \task Let $p(x)$ a polynomial for which there are real constants $r,s$ so that for all real $x$ we have\[ p(x+1)+p(x-1)-rp(x)-s=0. \] Prove that $\deg(p(x))\le 2$.
        \task Show, with the notation of the second part, that $s=0$ implies $a_2=0$.
    \end{tasks}
\end{question}

\begin{question}[name={1996 Germany}]
% https://artofproblemsolving.com/community/c1068820h2013605p14122765
    Prove the following statement: if a polynomial $p(x) = x^3 + Ax^2 + Bx +C$ has three real roots at least two of which are distinct, then $A^2 +B^2 +18C > 0$.
\end{question}

\begin{question}[name={1998 Germany}]
% https://artofproblemsolving.com/community/c6h607673p3611319
    Let $a$ be a positive real number. Then prove that the polynomial
    \[ p(x)=a^3x^3+a^2x^2+ax+a, \]
    has integer roots if and only if $a=1$ and determine those roots.
\end{question}


\begin{question}[name={1979 Brazil}]
% https://artofproblemsolving.com/community/c6h1563916p9574622
    The remainder on dividing the polynomial $p(x)$ by $x^2 - (a+b)x + ab$ (where $a \not = b$) is $mx + n$. Find the coefficients $m, n$ in terms of $a, b$. Find $m, n$ for the case $p(x) = x^{200}$ divided by $x^2 - x - 2$ and show that they are integral.
\end{question}



\begin{question}[name={1985 Brazil}]
% https://artofproblemsolving.com/community/c6h1680852p10715662
    $a, b, c, d$ are integers. Show that $x^2 + ax + b = y^2 + cy + d$ has infinitely many integer solutions if and only if $a^2 - 4b = c^2 - 4d$.
\end{question}

\begin{question}[name={1987 Brazil}]
% https://artofproblemsolving.com/community/c6h1680502p10713176
    Let $p(x_1, x_2, \dots , x_n)$ be a polynomial with integer coefficients. For each positive integer $r, k(r)$ is the number of $n$-tuples $(a_1, a_2,\dots , a_n)$ such that $0 \le a_i  \le r-1 $ and $p(a_1, a_2, \dots , a_n)$ is prime to $r$. Show that if $u$ and $v$ are coprime then $k(u\cdot v) = k(u)\cdot  k(v)$, and if p is prime then $k(p^s) = p^{n(s-1)} k(p)$.
\end{question}


\begin{question}[name={1991 Brazil}]
% https://artofproblemsolving.com/community/c6h80047p458003
    Given $k > 0$, the sequence $a_n$ is defined by its first two members and \[ a_{n+2} = a_{n+1} + \frac{k}{n}a_n.\]
    \begin{tasks}
        \task For which $k$ can we write $a_n$ as a polynomial in $n$?
        \task For which $k$ can we write \[\frac{a_{n+1}}{a_n} = \frac{p(n)}{q(n)},\] where $p,q$ are polynomials in $\mathbb R[X]$?
    \end{tasks}
\end{question}


\begin{question}[name={1992 Brazil}]
% https://artofproblemsolving.com/community/c6h79804p456697
    The equation $x^3+px+q=0$ has three distinct real roots. Show that $p<0$.
\end{question}


\begin{question}[name={1994 Brazil}]
% https://artofproblemsolving.com/community/c6h79648p455907
    Let $a, b > 0$ be reals such that
    \[ a^3=a+1 \qquad \text{and} \qquad  b^6=b+3a.  \]
    Show that $a>b$.
\end{question}


\begin{question}[name={1994 Brazil}]
% https://artofproblemsolving.com/community/c6h79549p455437
    Show that no one $n$-th root of a rational (for $n$ a positive integer) can be a root of the polynomial $x^5 - x^4 - 4x^3 + 4x^2 + 2$.
\end{question}

\begin{question}[name={1996 Brazil}]
% https://artofproblemsolving.com/community/c6h70222p410162
    Let $p(x)$ be the polynomial $x^3 + 14x^2 - 2x + 1$. Let $p^n(x)$ denote $p(p^(n-1)(x))$. Show that there is an integer N such that $p^N(x) - x$ is divisible by $101$ for all integers $x$.
\end{question}


\begin{question}[name={1997 Brazil}]
% https://artofproblemsolving.com/community/c6h78482p449658
    Let $f(x)= x^2-C$ where $C$ is a rational constant.
    Show that exists only finitely many rationals $x$ such that 
    $\{x,f(x),f(f(x)),\dots\}$ is finite.
\end{question}

\begin{question}[name={2007 Brazil}]
% https://artofproblemsolving.com/community/c6h173067p957986
    Let $ f(x) = x^2 + 2007x + 1$. Prove that for every positive integer $ n$, the equation \[\underbrace{f(f(\ldots(f}_{n \text{ times}}(x))\ldots)) = 0,\] has at least one real solution.
\end{question}

\begin{question}[name={2010 Brazil}]
% https://artofproblemsolving.com/community/c6h373065p2058534
    Let $P(x)$ be a polynomial with real coefficients. Prove that there exist positive integers $n$ and $k$ such that $k$ has $n$ digits and more than $P(n)$ positive divisors.
\end{question}




\begin{question}[name={2000 Czech and Slovak}]
% https://artofproblemsolving.com/community/c6h1446852p8270269
    Let $P(x)$ be a polynomial with integer coefficients. Prove that the polynomial $Q(x) = P(x^4)P(x^3)P(x^2)P(x)+1$ has no integer roots.
\end{question}



\begin{question}[name={2002 Czech and Slovak}]
% https://artofproblemsolving.com/community/c6h531848p3039042
    Let $n \ge 2$ be a fixed even integer. We consider polynomials of the form \[P(x) = x^n + a_{n-1}x^{n-1} + \cdots + a_1x + 1,\]
    with real coefficients, having at least one real roots. Find the least possible value of $a^2_1 + a^2_2 + \cdots + a^2_{n-1}$.
\end{question}


\begin{question}[name={2004 Czech and Slovak}]
% https://artofproblemsolving.com/community/c6h498300p2799947
    Show that real numbers, $p, q, r$ satisfy the condition $p^4(q-r)^2 + 2p^2(q+r) + 1 = p^4$ if and only if the quadratic equations $x^2 + px + q = 0$ and $y^2 - py + r = 0$ have real roots (not necessarily distinct) which can be labeled by $x_1,x_2$ and $y_1,y_2$, respectively, in such a way that $x_1y_1 - x_2y_2 = 1$.
\end{question}


\begin{question}[name={2005 Czech and Slovak}]
% https://artofproblemsolving.com/community/c6h531731p3038043
    Find all integers $n \ge 3$ for which the polynomial
    \[W(x) = x^n - 3x^{n-1} + 2x^{n-2} + 6,\]
    can be written as a product of two non-constant polynomials with integer coefficients.
\end{question}


\begin{question}[name={2007 Czech and Slovak}]
% https://artofproblemsolving.com/community/c6h430920p2436294
    Find all polynomials $P$ with real coefficients satisfying $P(x^2)=P(x)\cdot P(x+2)$ for all real numbers $x$.
\end{question}



\begin{question}[name={2008 Czech and Slovak}]
% https://artofproblemsolving.com/community/c6h529439p3017945
    Determine all triples $(x, y, z)$ of positive real numbers which satisfies the following system of equations
    \begin{align*}
        \begin{cases}
            2x^3 &= 2y(x^2+1)-(z^2+1),\\
            2y^4 &= 3z(y^2+1)-2(x^2+1),\\
            2z^5 &= 4x(z^2+1)-3(y^2+1).
        \end{cases}
    \end{align*}
\end{question}


\begin{question}[name={2011 Czech and Slovak}]
% https://artofproblemsolving.com/community/c6h423192p2393463
    A polynomial $P(x)$ with integer coefficients satisfies the following: if $F(x)$, $G(x)$, and $Q(x)$ are polynomials with integer coefficients satisfying $P\Big(Q(x)\Big)=F(x)\cdot G(x)$, then $F(x)$ or $G(x)$ is a constant polynomial. Prove that $P(x)$ is a constant polynomial.
\end{question}

\begin{question}[name={2012 Czech and Slovak}]
% https://artofproblemsolving.com/community/c6h529318p3016172
    Let $a,b,c,d$ be positive real numbers such that $abcd=4$ and
    \[a^2+b^2+c^2+d^2=10.\]
    Find the maximum possible value of $ab+bc+cd+da$.
\end{question}


\begin{question}[name={2013 Czech and Slovak}]
% https://artofproblemsolving.com/community/c6h1520925p9084444
    Let $a$ and $b$ be integers, where $b$ is not a perfect square. Prove that $x^2 + ax + b$ may be the square of an integer only for finite number of integer values of $x$.
\end{question}


\begin{question}[name={2014 Czech and Slovak}]
% https://artofproblemsolving.com/community/c6h1520931p9084525
    Prove that if the positive real numbers $a, b, c$ satisfy the equation
    \[a^4 + b^4 + c^4 + 4a^2b^2c^2 = 2 (a^2b^2 + a^2c^2 + b^2c^2),\]
    then there is a triangle $ABC$ with internal angles $\alpha, \beta, \gamma$ such that
    \[\sin \alpha = a, \qquad \sin \beta = b, \qquad \sin  \gamma= c.\]
\end{question}



\begin{question}[name={1991 China TST}]
    % https://artofproblemsolving.com/community/c6h42549p269077
    Let real coefficient polynomial $f(x) = x^n + a_1 \cdot x^{n-1} + \ldots + a_n$ has real roots $b_1, b_2, \dots, b_n$, $n \geq 2,$ prove that $\forall x \geq \max\{b_1, b_2, \ldots, b_n\}$, we have
    \[f(x+1) \geq \frac{2 \cdot n^2}{\displaystyle \frac{1}{x-b_1} + \frac{1}{x-b_2} + \cdots + \frac{1}{x-b_n}}.\]
\end{question}



\begin{question}[name={1995 China TST}]
    % https://artofproblemsolving.com/community/c6h37599p234868
    $ A$ and $ B$ play the following game with a polynomial of degree at least 4: \[ x^{2n} + \square x^{2n - 1} +  \square x^{2n - 2} + \cdots + \square x + 1 = 0.\]
    $A$ and $B$ take turns to fill in one of the blanks with a real number until all the blanks are filled up. If the resulting polynomial has no real roots, $A$ wins. Otherwise, $B$ wins. If $A$ begins, which player has a winning strategy?
\end{question}



\begin{question}[name={1995 China TST}]
    % https://artofproblemsolving.com/community/c6h37601p234870
    Prove that the interval $[0,1]$ can be split into black and white intervals for any quadratic polynomial $P(x)$, such that the sum of weights of the black intervals is equal to the sum of weights of the white intervals. Define the weight of the interval $[a,b]$ as $P(b) - P(a)$. Does the same result hold with a degree $3$ or degree $5$ polynomial?
\end{question}




\begin{question}[name={1996 China TST}]
    % https://artofproblemsolving.com/community/c6h37607p234888
    Let $\alpha_1, \alpha_2, \dots, \alpha_n$, and $\beta_1, \beta_2, \ldots, \beta_n$, where $n \geq 4$, be 2 sets of real numbers such that 
    \[\sum_{i=1}^{n} \alpha_i^2 < 1 \qquad \text{and} \qquad \sum_{i=1}^{n} \beta_i^2 < 1.\]
    Define
    \begin{align*}
        A^2 &= 1 - \sum_{i=1}^{n} \alpha_i^2,\\ 
        B^2 &= 1 - \sum_{i=1}^{n} \beta_i^2,\\
        W &= \frac{1}{2} (1 - \sum_{i=1}^{n} \alpha_i \beta_i)^2.
    \end{align*}
    Find all real numbers $\lambda$ such that the polynomial \[x^n + \lambda (x^{n-1} + \cdots + x^3 + Wx^2 + ABx + 1) = 0,\] only has real roots.
\end{question}




\begin{question}[name={1997 China TST}]
    % https://artofproblemsolving.com/community/c6h38390p239181
    Find all real-coefficient polynomials $f(x)$ which satisfy the following conditions:
    \begin{itemize}
        \item[(i)]  $f(x) = a_0 x^{2n} + a_2 x^{2n - 2} + \cdots + a_{2n - 2} x^2 + a_{2n}, a_0 > 0$;
        \item[(ii)] $\displaystyle \sum_{j=0}^n a_{2j} a_{2n - 2j} \leq \binom{2n}{n} a_0 a_{2n}$;
        \item[(iii)] All the roots of $f(x)$ are imaginary numbers with no real part.
    \end{itemize}
\end{question}




\begin{question}[name={2000 China TST}]
    % https://artofproblemsolving.com/community/c6h38449p239377
    Let $F$ be the set of all polynomials $\Gamma$ such that all the coefficients of $\Gamma (x)$ are integers and $\Gamma (x) = 1$ has integer roots. Given a positive integer $k$, find the smallest integer $m(k) > 1$ such that there exist $\Gamma \in F$ for which $\Gamma (x) = m(k)$ has exactly $k$ distinct integer roots.
\end{question}




\begin{question}[name={2002 China TST}]
    % https://artofproblemsolving.com/community/c6h55865p346543
    Let \[f(x_1,x_2,x_3) = -2 \cdot (x_1^3+x_2^3+x_3^3) + 3 \cdot (x_1^2(x_2+x_3) + x_2^2 \cdot (x_1+x_3) + x_3^2 \cdot ( x_1+x_2 ) - 12x_1x_2x_3.\] For any reals $r,s,t$, we denote \[g(r,s,t)=\max_{t\leq x_3\leq t+2} |f(r,r+2,x_3)+s|.\] Find the minimum value of $g(r,s,t)$.
\end{question}




\begin{question}[name={2002 China TST}]
    % https://artofproblemsolving.com/community/c6h254109p1389505
    Let $P_n(x)=a_0 + a_1x + \cdots + a_nx^n$, with $n \geq 2$, be a real-coefficient polynomial. Prove that if there exists $a > 0$ such that
    \begin{align*}
        P_n(x) = (x + a)^2 \left( \sum_{i=0}^{n-2} b_i x^i \right),
    \end{align*}
    where $b_i$ are positive real numbers, then there exists some $i$, with $1 \leq i \leq n-1$, such that \[a_i^2 - 4a_{i-1}a_{i+1} \leq 0.\]
\end{question}




\begin{question}[name={2002 China TST}]
    % https://artofproblemsolving.com/community/c6h254163p1389607
    For positive integers $a,b,c$ let $ \alpha, \beta, \gamma$ be pairwise distinct positive integers such that
    \[ \begin{cases}{c} \displaystyle a &= \alpha + \beta + \gamma, \\ 
    b &= \alpha \beta + \beta \gamma + \gamma \alpha, \\
    c^2 &= \alpha\beta\gamma. \end{cases} \]
    Also, let $ \lambda$ be a real number that satisfies the condition
    \[\lambda^4 -2a\lambda^2 + 8c\lambda + a^2 - 4b = 0.\]
    Prove that $\lambda$ is an integer if and only if $\alpha, \beta, \gamma$ are all perfect squares.
\end{question}



\begin{question}[name={2003 China TST}]
    % https://artofproblemsolving.com/community/c6h99375p561083
    The $n$ roots of a complex-coefficient polynomial \[f(z) = z^n + a_1z^{n - 1} + \cdots + a_{n - 1}z + a_n,\] are $z_1, z_2, \dots, z_n$. If $ \sum_{k = 1}^n |a_k|^2 \leq 1$, then prove that $ \sum_{k = 1}^n |z_k|^2 \leq n$.
\end{question}




\begin{question}[name={2003 China TST}]
    % https://artofproblemsolving.com/community/c6h99379p561100
    Can we find positive reals $a_1, a_2, \dots, a_{2002}$ such that for any positive integer $k$, with $1 \leq k \leq 2002$, every complex root $z$ of the following polynomial $f(x)$ satisfies the condition $|\text{Im } z| \leq |\text{Re } z|$, 
    \[f(x)=a_{k+2001}x^{2001}+a_{k+2000}x^{2000}+ \cdots + a_{k+1}x+a_k,\]  where $a_{2002+i}=a_i$, for $i=1,2, \dots, 2001$.
\end{question}




\begin{question}[name={2004 China TST}]
    % https://artofproblemsolving.com/community/c6h254972p1393051
    Given integer $ n$ larger than $ 5$, solve the system of equations (assuming $x_i \geq 0$, for $ i=1,2, \dots n$):
    \[ \begin{cases} \displaystyle x_1+ \phantom{2^2} x_2+ \phantom{3^2} x_3 + \cdots + \phantom{n^2} x_n &= n+2, \\ x_1 + 2\phantom{^2}x_2 + 3\phantom{^2}x_3 + \cdots + n\phantom{^2}x_n &= 2n+2, \\ x_1 + 2^2x_2 + 3^2 x_3 + \cdots + n^2x_n &= n^2 + n +4, \\ x_1+ 2^3x_2 + 3^3x_3+ \cdots + n^3x_n &= n^3 + n + 8. \end{cases} \]
\end{question}




\begin{question}[name={2005 China TST}]
    % https://artofproblemsolving.com/community/c6h98963p558618
    Let $a_1,a_2 \dots a_n$ and $x_1, x_2 \dots x_n$ be integers and $r\geq 2$ be an integer. It is known that \[\sum_{j=0}^{n} a_j x_j^k =0 \qquad \text{for} \quad k=1,2, \dots r.\]
    Prove that
    \[\sum_{j=0}^{n} a_j x_j^m \equiv 0 \pmod m, \qquad \text{for all}\quad  m \in \{ r+1, r+2, \cdots, 2r+1 \}.\]
\end{question}


\begin{question}
% https://artofproblemsolving.com/community/c6h87293
    Prove the following statements on irrationality.
    \begin{tasks}
        \task Show that $\sqrt{2} + \sqrt{3} + \sqrt{5} + \sqrt{7}$ is irrational.
        \task Suppose $a_i$, with $i=1,2, \dots ,n$ are rationals such that $\sqrt{a_i}$ is irrational for at least one value of $i$. Prove that \[\sqrt{a_1} + \sqrt{a_2} + \cdots + \sqrt{a_n}\] is irrational.
    \end{tasks}
\end{question}


\begin{question}[name={2005 China TST}]
    % https://artofproblemsolving.com/community/c6h98969p558633
    Determine whether $\sqrt{1001^2+1}+\sqrt{1002^2+1}+ \cdots + \sqrt{2000^2+1}$ be a rational number or not?
\end{question}


\begin{question}[name={Zhaobin vs Vess}]
% https://artofproblemsolving.com/community/c6h27089
    Let $a_1,a_2,\dots,a_n$ be $n$ positive rational numbers and assume that $k_1,k_2,\dots,k_n$ are $n$ positive integers such that $\sqrt[{k_i}]{a_i}$ is irrational. Prove that the sum $\sum_{i=1}^{n} \sqrt[{k_i}]{a_i}$ is irrational.
\end{question}


\begin{question}[name={2006 China TST}]
    % https://artofproblemsolving.com/community/c6h97486p550563
    Let $a_{i}$ and $b_{i}$ (for $i=1,2, \dots, n$) be rational numbers such that for any real number $x$, we have: \[x^{2}+x+4=\sum_{i=1}^{n}(a_{i}x+b)^{2}.\] Find the least possible value of $n$.
\end{question}




\begin{question}[name={2003 China TST}]
    % https://artofproblemsolving.com/community/c6h97496p550613
    Find all second degree polynomial $d(x)=x^{2}+ax+b$ with integer coefficients, so that there exists an integer-coefficient polynomial $p(x)$ and a non-zero integer-coefficient polynomial $q(x)$ that satisfy: \[\left( p(x) \right)^{2}-d(x) \left( q(x) \right)^{2}=1, \qquad \text{for all} \quad x \in \mathbb R.\]
\end{question}




\begin{question}[name={2006 China TST}]
    % https://artofproblemsolving.com/community/c6h97508p550639
    Let $k$ be an odd number that is greater than or equal to $3$. Prove that there exists a $k^{th}$-degree integer-valued polynomial with non-integer-coefficients that has the following properties:
    \begin{enumerate}
        \item $f(0)=0$ and $f(1)=1$; and
        \item There exist infinitely many positive integers $n$ so that if the following equation: \[ n= f(x_1)+\cdots+f(x_s), \] has integer solutions $x_1, x_2, \dots, x_s$, then $s \geq 2^k-1$.
    \end{enumerate} 
\end{question}


\begin{question}[name={2008 China TST}]
% https://artofproblemsolving.com/community/c6h140594p795284
    After multiplying out and simplifying polynomial \[(x - 1)(x^2 - 1)(x^3 - 1)\cdots(x^{2007} - 1),\] getting rid of all terms whose powers are greater than $2007$, we acquire a new polynomial $f(x)$. Find its degree and the coefficient of the term having the highest power. In other words, if \[P(x) = (1 - x)(1 - x^{2})\cdots(1 - x^{2007}),\] find the degree of $P(x) \pmod {x^{2008}}$.
\end{question}

\begin{question}[name={2008 China TST}]
% https://artofproblemsolving.com/community/c6h198501p1091443
    Let $ z_{1},z_{2},z_{3}$ be three complex numbers of modulii less than or equal to $1$. Let $ w_{1},w_{2}$ be two roots of the equation \[(z - z_{1})(z - z_{2}) + (z - z_{2})(z - z_{3}) + (z - z_{3})(z - z_{1}) = 0.\] Prove that, for $ j = 1,2,3$, we have \[\min\{|z_{j} - w_{1}|,|z_{j} - w_{2}|\}\leq 1.\]
\end{question}


\begin{question}[name={2008 China TST}]
% https://artofproblemsolving.com/community/c6h198296p1090306
    Let $n>m>1$ be odd integers, and define \[f(x)=x^n+x^m+x+1.\] Prove that $f(x)$ can't be expressed as the product of two polynomials having integer coefficients and positive degrees.
\end{question}



\begin{question}[name={2009 China TST}]
% https://artofproblemsolving.com/community/c6h268966p1457632
    Find all complex polynomial $P(x)$ such that for any three integers $a,b,c$ satisfying $ a + b + c\not = 0$, and \[\displaystyle \frac{P(a) + P(b) + P(c)}{a + b + c} \quad \text{is an integer.}\]
\end{question}


\begin{question}[name={2010 China TST}]
% https://artofproblemsolving.com/community/c6h366315p2015063
    Given positive integer $n$, find the largest real number $\lambda=\lambda(n)$, such that for any degree-$n$ polynomial with complex coefficients \[f(x)=a_n x^n+a_{n-1} x^{n-1}+\cdots+a_0,\] and any permutation $x_0,x_1,\dots,x_n$ of $0,1,\dots,n$, the following inequality holds: \[\sum_{k=0}^n|f(x_k)-f(x_{k+1})|\geq \lambda |a_n|,\] where $x_{n+1}=x_0$.
\end{question}


\begin{question}[name={2012 China TST}]
% https://artofproblemsolving.com/community/c6h471563p2639961
    Find the smallest possible value of a real number $c$ such that for any $2012^{th}$-degree monic polynomial
    \[P(x)=x^{2012}+a_{2011}x^{2011}+\cdots+a_1x+a_0,\] with real coefficients, we can obtain a new polynomial $Q(x)$ by multiplying some of its coefficients by $-1$ such that every root $z$ of $Q(x)$ satisfies the inequality
    \[ \left\lvert \text{Im } z \right\rvert \le c \left\lvert \text{Re } z \right\rvert. \]
\end{question}



\begin{question}[name={1981 Austrian--Polish}]
    %https://artofproblemsolving.com/community/c1068820h2076961p14911620
    Let $P(x) = x^4 + a_1x^3 + a_2x^2 + a_3x + a_4$ be a polynomial with rational coefficients. Show that if $P(x)$ has exactly one real root $\xi$, then $\xi$ is a rational number.
\end{question}


\begin{question}[name={1986 Austrian--Polish}]
    % https://artofproblemsolving.com/community/c1068820h2082179p14984054
    The monic polynomial $P(x) = x^n + a_{n-1}x^{n-1} +\cdots+ a_0$ of degree $n > 1$ has $n$ distinct negative roots. Prove that $a_1P(1) > 2n^2a_o$.
\end{question}




\begin{question}[name={1986 Austrian--Polish}]
    % https://artofproblemsolving.com/community/c1068820h2082188p14984143
    Find all real solutions $x,y,u,v$ of the system of equations
    \begin{align*}
        \begin{cases}
            x^2 + y^2 + u^2 + v^2 &= 4,\\
            xu + yv + xv + yu &= 0,\\
            xyu + yuv + uvx + vxy &= - 2,\\
            xyuv = -1
        \end{cases}
    \end{align*}
\end{question}




\begin{question}[name={1987 Austrian--Polish}]
    % https://artofproblemsolving.com/community/c1068820h2082212p14984316
    Let $n$ be the square of an integer whose each prime divisor has an even number of decimal digits. Consider $P(x) = x^n - 1987x$. Show that if $x,y$ are rational numbers with $P(x) = P(y)$, then $x = y$.
\end{question}




\begin{question}[name={1988 Austrian--Polish}]
    % https://artofproblemsolving.com/community/c1068820h2082253p14984954
    Let $P(x)$ be a polynomial with integer coefficients. Show that if $Q(x) = P(x) +12$ has at least six distinct integer roots, then $P(x)$ has no integer roots.
\end{question}




\begin{question}[name={1988 Austrian--Polish}]
    % https://artofproblemsolving.com/community/c1068820h2082309p14985732
    If $a_1 \le  a_2 \le  \cdots \le  a_n$ are natural numbers ($n \ge 2$), show that the inequality $$\sum_{i=1}^n a_ix_i^2 +2\sum_{i=1}^{n-1} x_ix_{i+1}  >0,$$ holds for all $n$-tuples $(x_1,\dots,x_n) \ne (0,\dots, 0)$ of real numbers if and only if $a_2 \ge 2$.
\end{question}




\begin{question}[name={1990 Austrian--Polish}]
    % https://artofproblemsolving.com/community/c1068820h2091350p15108732
    Show that there are two real solutions $(x,y,z)$ to:
    \begin{align*}
        \begin{cases}
            x + y^2 + z^4 &= 0,\\
            y + z^2 + x^4 &= 0,\\
            z + x^2 + y^5 &= 0.
        \end{cases}
    \end{align*}
\end{question}



\begin{question}[name={1990 Austrian--Polish}]
    % https://artofproblemsolving.com/community/c1068820h2091357p15108812
    Given a positive integer $n\geq 2$, find all solutions $(x_i,y_i)$ to the following system of equations, where $1 \leq i \leq n$:
    \begin{align*}
        \begin{cases}
            x_1^4 + 14x_1x_2 + 1  &= y_1^4,\\
            x_2^4 + 14x_2x_3 + 1  &= y_2^4,\\
            \qquad \vdots &\phantom{=} \vdots\\
            x_n^4 + 14x_nx_1 + 1  &= y_n^4.
        \end{cases}
    \end{align*}
\end{question}



\begin{question}[name={1990 Austrian--Polish}]
    % https://artofproblemsolving.com/community/c1068820h2091360p15108862
    Let $p(x)$ be a polynomial with integer coefficients. The sequence of integers $a_1, a_2, \dots , a_n$ (where $n > 2$) satisfies
    \[a_2 = p(a_1), \quad a_3 = p(a_2),\quad \dots , \quad a_n = p(a_{n-1}),\quad a_1 = p(a_n).\] Show that $a_1 = a_3$.
\end{question}




\begin{question}[name={1991 Austrian--Polish}]
    % https://artofproblemsolving.com/community/c1068820h2082346p14986527
    Let $P(x)$ be a real polynomial with $P(x) \ge 0$ for $0 \le x \le  1$. Show that there exist polynomials $P_i (x)$ (for $i = 0, 1,2$) with $P_i (x) \ge 0$ for all real x such that \[P (x) = P_0 (x) + xP_1 (x)( 1- x)P_2 (x).\]
\end{question}




\begin{question}[name={1991 Austrian--Polish}]
    % https://artofproblemsolving.com/community/c1068820h2082364p14986800
    For a given positive integer $n$ determine the maximum value of the function \[f (x) = \frac{x + x^2 +\cdots+ x^{2n-1}}{(1 + x^n)^2}, \qquad \text{for all} \quad x \geq 0,\] and find all positive $x$ for which the maximum is attained.
\end{question}




\begin{question}[name={1992 Austrian--Polish}]
    % https://artofproblemsolving.com/community/c1068820h2091394p15109289
    Let $k$ be a positive integer and $u, v$ be real numbers, and \[P(x) = (x - u^k) (x - uv) (x -v^k) = x^3 + ax^2 + bx + c.\]
    \begin{tasks}
        \task For $k = 2$ prove that if $a, b, c$ are rational then so is $uv$.
        \task Is that also true for $k = 3$?
    \end{tasks}
\end{question}




\begin{question}[name={1993 Austrian--Polish}]
    % https://artofproblemsolving.com/community/c1068820h2085130p15028129
    Solve in real numbers the system
    \begin{align*}
        \begin{cases}
            x^3 + y  &= 3x + 4,\\
            2y^3 + z &= 6y + 6,\\
            3z^3 + x &= 9z + 8.
        \end{cases}
    \end{align*}
\end{question}




\begin{question}[name={1993 Austrian--Polish}]
    % https://artofproblemsolving.com/community/c1068820h2085148p15028306
    Determine all real polynomials $P(z)$ for which there exists a unique real polynomial $Q(x)$ satisfying the conditions $Q(0)= 0$, and \[x + Q(y + P(x))= y + Q(x + P(y)),\] for all $x,y \in mathbb R$.
\end{question}




\begin{question}[name={1994 Austrian--Polish}]
    % https://artofproblemsolving.com/community/c1068820h2085493p15032329
    Let $n > 1$ be an odd positive integer. Assume that positive integers $x_1, x_2,\dots, x_n \ge 0$ satisfy:
    \begin{align*}
        \begin{cases}
            (x_2 - x_1)^2 + 2(x_2 +x_1) + 1 &= n^2,\\
            (x_3 -x_2)^2 + 2(x_3 +x_2) + 1 &= n^2,\\
            \qquad \vdots &\phantom{=} \vdots\\
            (x_1 - x_n)^2 + 2(x_1 + x_n)+ 1 &= n^2.
        \end{cases}
    \end{align*}
    Show that there exists $j$ with $1 \le j \le n$, such that $x_j = x_{j+1}$. Here, assume $x_{n+1} = x_1$.
\end{question}




\begin{question}[name={1994 Austrian--Polish}]
    % https://artofproblemsolving.com/community/c6h76350p438681
    Solve in integers the following equation
    \[\frac{1}{2}(x + y)(y + z)(z + x) + (x + y + z)^3 = 1 - xyz.\]
\end{question}




\begin{question}[name={1995 Austrian--Polish}]
    % https://artofproblemsolving.com/community/c1068820h2085525p15032637
    Consider the equation $3y^4 + 4cy^3 + 2xy + 48 = 0$, where $c$ is an integer parameter. Determine all values of $c$ for which the number of integral solutions $(x,y)$ satisfying the conditions (i) and (ii) is maximal:
    \begin{itemize}
        \item[(i)] $|x|$ is a square of an integer;
        \item[(ii)] $y$ is a square-free number.
    \end{itemize}
    Remember that a square-free number is an integer which is not divisible by the square of any prime.
\end{question}




\begin{question}[name={1996 Austrian--Polish}]
    % https://artofproblemsolving.com/community/c6h566114p3313195
    The polynomials $P_{n}(x)$ are defined initially by $P_{0}(x)=0$ and $P_{1}(x)=x$, and then recursively, for $n\geq 2$, by
    \[P_{n}(x)=xP_{n-1}(x)+(1-x)P_{n-2}(x).\] For every natural number $n\geq 1$, find all real numbers $x$ satisfying the equation $P_{n}(x)=0$.
\end{question}




\begin{question}[name={1996 Austrian--Polish}]
    % https://artofproblemsolving.com/community/c1068820h2085603p15034194
    Given natural numbers $n > k > 1$, find all real solutions $x_1,\dots, x_n$ of the system $$x_i^3(x_i^2 + x_{i+1}^2+ \cdots +x_{i+k-1}^2) = x_{i-1}^2,$$ for 1 $\le i \le n$. Here $x_{n+i} = x_i$ for all$ i$.
\end{question}


\begin{question}[name={1999 Austrian--Polish}]
    % https://artofproblemsolving.com/community/c1068820h2085844p15037788
    Solve in the non-negative real numbers the system of equations
    \[x_n^2 + x_nx_{n-1} + x_{n-1}^4 = 1,\]
    for $n = 1,2,\dots,1999$, assuming $x_0 = x_{1999}$.
\end{question}


\begin{question}[name={2000 Austrian--Polish}]
    % https://artofproblemsolving.com/community/c1068820h2085849p15037856
    For each integer $n \ge 3$ solve in real numbers the system of equations:
    \begin{align*}
        \begin{cases}
            x_1^3 &= x_2 + x_3 + 1,\\
            x_2^3 &= x_3 + x_4 + 1,\\
            \, \vdots &\phantom{=} \quad\vdots\\
            x_{n-1}^3 &= x_n+ x_1 + 1,\\
            x_{n}^3 &= x_1+ x_2 + 1.
        \end{cases}
    \end{align*}
\end{question}



\begin{question}[name={2003 Austrian--Polish}]
    % https://artofproblemsolving.com/community/c1068820h2075236p14891304
    Find all real polynomials $p(x) $ such that \[p(x-1)p(x+1)= p(x^2-1).\]
\end{question}


\begin{question}[name={2003 Austrian--Polish}]
    % https://artofproblemsolving.com/community/c1068820h2075288p14891583
    For each positive integer $n>1$, define \[f(n) = \frac{n^n - 1}{n - 1}.\]
    \begin{tasks}
        \task Show that $n!^{f(n)}$ divides $(n^n)!$.
        \task Find as many positive integers as possible for which $n!^{f(n)+1}$ does not divide $(n^n)!$.
    \end{tasks} 
\end{question}


\begin{question}[name={2004 Austrian--Polish}]
    % https://artofproblemsolving.com/community/c6h26790p167451
    Solve the following system of equations in $\mathbb{R}$ where all square roots are non-negative:
    \begin{align*}
        \begin{cases}
            a  - \sqrt{1-b^2} + \sqrt{1-c^2} &= d, \\
            b -  \sqrt{1-c^2} + \sqrt{1-d^2} &= a, \\
            c -  \sqrt{1-d^2} + \sqrt{1-a^2} &= b, \\
            d -  \sqrt{1-a^2} + \sqrt{1-b^2} &= c.
        \end{cases}
    \end{align*}
\end{question}



\begin{question}[name={2004 Austrian--Polish}]
    % https://artofproblemsolving.com/community/c6h26531p166040
    Determine all $n$ for which the system with of equations can be solved in $\mathbb{R}$: 
    \begin{align*}
        \sum^{n}_{k=1} x_k &= 27,\\
        \prod^{n}_{k=1} x_k &= \left( \frac{3}{2} \right)^{24}.
    \end{align*}
\end{question}



\begin{question}[name={2004 Austrian--Polish}]
    % https://artofproblemsolving.com/community/c6h26535p166050
    For each polynomial $Q(x)$ let $M(Q)$ be the set of non-negative integers $x$ with $0 < Q(x) < 2004$. We consider polynomials $P_n(x)$ of the form \[P_n(x) = x^n + a_1 \cdot x^{n-1} + \cdots + a_{n-1} \cdot x + 1,\] with coefficients $a_i \in \{ \pm1\}$ for $i = 1, 2, \dots, n-1.$ For each $n = 3^k$, with $k > 0$, determine:
    \begin{tasks}
        \task $m_n$, which represents the maximum of elements in $M(P_n)$ for all such polynomials $P_n(x)$; and
        \task all polynomials $P_n(x)$ for which $|M(P_n)| = m_n$.
    \end{tasks}
\end{question}




\begin{question}[name={2005 Austrian--Polish}]
    % https://artofproblemsolving.com/community/c6h1110611p5059849
    Determine all polynomials $P$ with integer coefficients satisfying
    \[P(P(P(P(P(x)))))=x^{28}\cdot P(P(x)),\qquad \text{for all} \quad x\in\mathbb{R}.\]
\end{question}


\begin{question}[name={2005 Austrian--Polish}]
    % https://artofproblemsolving.com/community/c6h1110617p5059858
    For each natural number $n\geq 2$, solve the following system of equations in the integers $x_1, x_2, \dots, x_n$: $$(n^2-n)x_i+\left(\prod_{j\neq i}x_j\right)S=n^3-n^2,\qquad \text{for} \quad i=1,2,\dots, n,$$ where, $$S=x_1^2+x_2^2+\dots+x_n^2.$$
\end{question}

\begin{question}[name={2006 Austrian--Polish}]
    % https://artofproblemsolving.com/community/c6h107523p608083
    Find all polynomials $P(x)$ with real coefficients satisfying the equation \[(x+1)^{3}P(x-1)-(x-1)^{3}P(x+1)=4(x^{2}-1) P(x),\] for all real numbers $x$.
\end{question}



\begin{question}[name={2000 Austria}]
    % https://artofproblemsolving.com/community/c1068820h2934621p26259744
    For any real number $a$, find all real numbers $x$ that satisfy the following equation: $$(2x + 1)^4 + ax(x + 1) - \frac{x}{2}= 0.$$
\end{question}


\begin{question}[name={2002 Austria}]
    % https://artofproblemsolving.com/community/c6h1984377p13803521
    Solve the following system of equations over the real numbers:
    \begin{align*}
        \begin{cases}
            2x_1 = x_5 ^2 - 23,\\
            4x_2 = x_1 ^2 + 7,\\
            6x_3 = x_2 ^2 + 14,\\
            8x_4 = x_3 ^2 + 23,\\
            10x_5 = x_4 ^2 + 34.
        \end{cases}
    \end{align*}
\end{question}


\begin{question}[name={2004 Austria}]
    % https://artofproblemsolving.com/community/c6h257338p1403150
    Solve the following equation for real numbers (all square roots are non negative):
    \[\sqrt{4-x\sqrt{4-(x-2)\sqrt{1+(x-5)(x-7)}}}=\frac{5x-6-x^2}{2}.\]
\end{question}


\begin{question}[name={2010 Austria}]
    % https://artofproblemsolving.com/community/c6h346898p1858172
    Solve the following in equation in $\mathbb{R}^3$:
    \[4x^4-x^2(4y^4+4z^4-1)-2xyz+y^8+2y^4z^4+y^2z^2+z^8=0.\]
\end{question}


\begin{question}[name={2010 Donova (Danube)}]
    % https://artofproblemsolving.com/community/c6h390073p2167356
    Let $n\ge3$ be a positive integer. Find non-negative real numbers $x_1,x_2,\dots,x_n$, with $x_1+x_2+\cdots +x_n=n$, for which the expression \[(n-1)(x_1^2+x_2^2+\cdots+x_n^2)+nx_1x_2\cdots x_n,\] takes a minimal value.
\end{question}


\begin{question}[name={2017 Donova (Danube)}]
    % https://artofproblemsolving.com/community/c6h1536291p9268226
    Find all polynomials $P(x)$ with integer coefficients such that $a^2+b^2-c^2$ divides $P(a)+P(b)-P(c)$, for all integers $a,b,c$.
\end{question}

\begin{question}[name={1959--1966 IMO Longlist}]
     If $a,b, c,d$ are integers such that $ad$ is odd and $bc$ is even, prove that at least one root of the polynomial $ax^3 +bx^2 +cx+d$ is irrational.
\end{question}

\begin{question}[name={1968 IMO Shortlist}]
    A polynomial $p(x) = a_0x^k +a_1x^{k-1} +\cdots+a_k$ with integer coefficients is said to be divisible by an integer $m$ if $p(x)$ is divisible by $m$ for all integers $x$. Prove that if $p(x)$ is divisible by $m$, then $k!a_0$ is also divisible by $m$. Also prove that if $a_0, k,$ and $m$ are non-negative integers for which $k!a_0$ is divisible by $m$, then there exists a polynomial $p(x) = a_0x^k +\cdots+a_k$ divisible by $m$.
\end{question}


\begin{question}[name={1968 IMO Shortlist}]
    Find all complex numbers m such that polynomial
    \[x^3 +y^3 +z^3 +mxyz,\]
    can be represented as the product of three linear trinomials.
\end{question}


\begin{question}[name={1969 IMO Longlist}]
    Let us define $u_0 = 0, u_1 = 1$ and for $n \geq 0$, \[u_{n+2} = au_{n+1} +bu_n ,\] where $a$ and $b$ are positive integers. Express $u_n$ as a polynomial in $a$ and $b$. Prove the result. Given that $b$ is prime, prove that $b$ divides $a(u_b -1)$.
\end{question}





\begin{question}[name={1969 IMO Longlist}]
    Given a polynomial $f(x)$ with integer coefficients whose value is divisible by $3$ for three integers $k, k +1$, and $k +2$, prove that $f(m)$ is divisible by $3$ for all integers $m$.
\end{question}





\begin{question}[name={1969 IMO Longlist}]
    Prove that if $0\leq a_0 \leq a_1 \leq a_2$, then 
    \[(a_0+a_1x-a_2x^2)^2 \leq (a_0+a_1+a_2)^2 \left(1+\frac{x}{2}+\frac{x^2}{3}+\frac{x^3}{2}+x^4\right),\]
    and formulate and prove the analogous result for polynomials of third degree.
\end{question}


\begin{question}[name={1970 IMO Longlist}]
    Given a polynomial
    \begin{multline*}
        P(x) = ab(a-c)x^3 +(a^3-a^2c+2ab^2-b^2c+abc)x^2 +\\
        (2a^2b+b^2c+a^2c+b^3-abc)x+ab(b+c),
    \end{multline*}
    where $a,b, c \neq  0$, prove that $P(x)$ is divisible by $Q(x) = abx^2 +(a^2 +b^2)x+ab$ and conclude that $P(x_0)$ is divisible by $(a+b)^3$ for $x_0 = (a+b+1)^n$, for all $n \in\mathbb N$.
\end{question}


\begin{question}[name={1970 IMO Longlist}]
    Let a polynomial $p(x)$ with integer coefficients take the value $5$ for five different integer values of $x$. Prove that $p(x)$ does not take the value $8$ for any integer $x$.
\end{question}

\begin{question}[name={1970 IMO Shortlist}]
% https://artofproblemsolving.com/community/c6h368340p2027351
    Let $P,Q,R$ be polynomials and let $S(x) = P(x^3) + xQ(x^3) + x^2R(x^3)$ be a polynomial of degree $n$ whose roots $x_1,\dots, x_n$ are distinct. Construct with the aid of the polynomials $P,Q,R$ a polynomial $T$ of degree $n$ that has the roots $x_1^3 , x_2^3 , \dots, x_n^3.$
\end{question}

\begin{question}[name={1971 IMO Shortlist}]
% https://artofproblemsolving.com/community/c6h368177p2026501
    Consider a sequence of polynomials $\{P_i(x)\}_{i=0}^{\infty}$, where $P_0(x) = 2, P_1(x) = x$ and for every $n \geq 1$ the following equality holds: \[P_{n+1}(x) + P_{n-1}(x) = xP_n(x).\]
    Prove that there exist three real numbers $a, b, c$ such that for all $n \geq 1,$ \[(x^2 - 4)[P_n^2(x) - 4] = [aP_{n+1}(x) + bP_n(x) + cP_{n-1}(x)]^2.\]
\end{question}

\begin{question}[name={1971 IMO Shortlist}]
% https://artofproblemsolving.com/community/c6h384577p2134901
    Prove that the polynomial $x^4+\lambda x^3+\mu x^2+\nu x+1$ has no real roots if $\lambda, \mu , \nu $ are real numbers satisfying 
    \[|\lambda|+|\mu|+|\nu|\le \sqrt{2}.\]
\end{question}

\begin{question}[name={1976 IMO Longlist}]
% https://artofproblemsolving.com/community/c6h388987p2161345
    Prove that if for a polynomial $P(x, y)$, we have
\[P(x - 1, y - 2x + 1) = P(x, y),\]
then there exists a polynomial $\Phi(x)$ with $P(x, y) = \Phi(y - x^2).$
\end{question}


\begin{question}[name={1976 IMO Longlist}]
% https://artofproblemsolving.com/community/c6h367765p2024519
    The polynomial $1976(x+x^2+ \cdots +x^n)$ is decomposed into a sum of polynomials of the form $a_1x + a_2x^2 + \cdots + a_nx^n$, where $a_1, a_2, \ldots , a_n$ are distinct positive integers not greater than $n$. Find all values of $n$ for which such a decomposition is possible.	
\end{question}

\begin{question}[name={1976 IMO Longlist}]
% https://artofproblemsolving.com/community/c6h388771p2159992
    Let $g(x)$ be a fixed polynomial with real coefficients and define $f(x)$ by $f(x) =x^2 + xg(x^3)$. Show that $f(x)$ is not divisible by $x^2 - x + 1$.
\end{question}

\begin{question}[name={1976 IMO Longlist}]
% https://artofproblemsolving.com/community/c6h367763p2024515
    Let $P$ be a polynomial with real coefficients such that $P(x) > 0$ if $x > 0$. Prove that there exist polynomials $Q$ and $R$ with non-negative coefficients such that \[P(x) = \frac{Q(x)}{R(x)} \qquad \text{if} \quad x > 0.\]
\end{question}

\begin{question}[name={1976 IMO Longlist}]
% https://artofproblemsolving.com/community/c6h388155p2156557
    Prove that if $P(x) = (x-a)^kQ(x)$, where $k$ is a positive integer, $a$ is a nonzero real number, $Q(x)$ is a nonzero polynomial, then $P(x)$ has at least $k + 1$ nonzero coefficients.
\end{question}

\begin{question}[name={1978 IMO Longlist}]
% https://artofproblemsolving.com/community/c6h375003p2069567
    Given the expression
\[P_n(x) =\frac{1}{2^n}\left[(x +\sqrt{x^2 - 1})^n+(x-\sqrt{x^2 - 1})^n\right],\]
prove that:
\begin{tasks}
    \task $P_n(x)$ satisfies the identity
    \[P_n(x) - xP_{n-1}(x) + \frac{1}{4}P_{n-2}(x) \equiv 0.\]
    \task $P_n(x)$ is a polynomial in $x$ of degree $n.$
\end{tasks}
\end{question}

\begin{question}[name={1982 IMO Longlist}]
% https://artofproblemsolving.com/community/c6h366195p2014393
    Determine all real values of the parameter $a$ for which the equation
\[16x^4 -ax^3 + (2a + 17)x^2 -ax + 16 = 0,\]
has exactly four distinct real roots that form a geometric progression.
\end{question}

\begin{question}[name={1984 IMO Longlist}]
% https://artofproblemsolving.com/community/c6h371366p2048065
    Let $f_1(x) = x^3+a_1x^2+b_1x+c_1 = 0$ be an equation with three positive roots $\alpha>\beta>\gamma > 0$. From the equation $f_1(x) = 0$, one constructs the equation $f_2(x) = x^3 +a_2x^2 +b_2x+c_2 = x(x+b_1)^2 -(a_1x+c_1)^2 = 0$. Continuing this process, we get equations $f_3,\cdots, f_n$. Prove that
\[\lim_{n\to\infty}\sqrt[2^{n-1}]{-a_n} = \alpha.\]
\end{question}

\begin{question}[name={1984 IMO Longlist}]
% https://artofproblemsolving.com/community/c6h368510p2028585
    Let $P,Q,R$ be the polynomials with real or complex coefficients such that at least one of them is not constant. If $P^n+Q^n+R^n = 0$, prove that $n < 3.$
\end{question}



\begin{question}[name={1985 IMO Longlist}]
% https://artofproblemsolving.com/community/c6h364118p2000176
    Find a method by which one can compute the coefficients of $P(x) = x^6 + a_1x^5 + \cdots+  a_6$ from the roots of $P(x) = 0$ by performing not more than $15$ additions and $15$ multiplications.
\end{question}

\begin{question}[name={1987 IMO Longlist}]
% https://artofproblemsolving.com/community/c6h365495p2010133
    Let $P,Q,R$ be polynomials with real coefficients, satisfying $P^4+Q^4 = R^2$. Prove that there exist real numbers $p, q, r$ and a polynomial $S$ such that $P = pS, Q = qS$ and $R = rS^2$.
\end{question}


\begin{question}[name={1989 IMO Longlist}]
% https://artofproblemsolving.com/community/c6h226761p1257847
    Let $f(x) = \prod^n_{k=1} (x - a_k) - 2,$ where $ n \geq 3$ and $a_1, a_2, \dots,$ an are distinct integers. Suppose that $f(x) = g(x)h(x),$ where $ g(x), h(x)$ are both non-constant polynomials with integer coefficients. Prove that $n = 3.$
\end{question}


\begin{question}[name={1989 IMO Longlist}]
% https://artofproblemsolving.com/community/c6h226767p1257860
    Let $ P_1(x), P_2(x), \dots, P_n(x)$ be real polynomials, i.e., they have real coefficients. Show that there exist real polynomials $A_r(x),B_r(x) \quad (r = 1, 2, 3)$ such that
    \begin{align*}
        \sum^n_{s=1} \left\{ P_s(x) \right \}^2 &\equiv \left( A_1(x) \right)^2 + \left( B_1(x) \right)^2,\\
        \sum^n_{s=1} \left\{ P_s(x) \right \}^2 &\equiv \left( A_2(x) \right)^2 + x \left( B_2(x) \right)^2,\\
        \sum^n_{s=1} \left\{ P_s(x) \right \}^2 &\equiv \left( A_3(x) \right)^2 - x \left( B_3(x) \right)^2.
    \end{align*}
\end{question}

\begin{question}[name={1992 IMO Longlist}]
% https://artofproblemsolving.com/community/c6h364939p2005889
    Let $P_1(x, y)$ and $P_2(x, y)$ be two relatively prime polynomials with complex coefficients. Let $Q(x, y)$ and $R(x, y)$ be polynomials with complex coefficients and each of degree not exceeding $d$. Prove that there exist two integers $A_1, A_2$ not simultaneously zero with $|A_i| \leq d + 1 \  (i = 1, 2)$ and such that the polynomial $A_1P_1(x, y) + A_2P_2(x, y)$ is coprime to $Q(x, y)$ and $R(x, y).$
\end{question}

\begin{question}[name={1992 IMO Longlist}]
% https://artofproblemsolving.com/community/c6h364951p2005932
    Let $f(x) = x^m + a_1x^{m-1} + \cdots+ a_{m-1}x + a_m$ and $g(x) = x^n + b_1x^{n-1} + \cdots + b_{n-1}x + b_n$ be two polynomials with real coefficients such that for each real number $x, f(x)$ is the square of an integer if and only if so is $g(x)$. Prove that if $n +m > 0$, then there exists a polynomial $h(x)$ with real coefficients such that $f(x) \cdot g(x) = (h(x))^2.$
\end{question}


