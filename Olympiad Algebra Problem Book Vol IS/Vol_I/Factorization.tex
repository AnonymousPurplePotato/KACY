\section{Factorization}

%%%% Equations in one variable
\subsection{One--Variable Identities}


\begin{tcolorbox}
\SetupExSheets{headings=runin}
\begin{question}
Factorize $2x^4 + x^3 + 4x^2 + x + 2$.
\end{question}
\end{tcolorbox}

\begin{solution}[name=Solution by Parviz Shahriari]
Answer: $(x^2+1)(2x^2+x+2)$.
\end{solution}




\begin{tcolorbox}
\SetupExSheets{headings=runin}
\begin{question}
Factorize $x^5+x^4+x^3+x^2+x+1$.
\end{question}
\end{tcolorbox}

\begin{solution}[name=Solution by Parviz Shahriari]
Answer: $(x+1)(x^2+x+1)(x^2-x+1)$.
\end{solution}



\begin{tcolorbox}
\SetupExSheets{headings=runin}
\begin{question}
Factorize $x^4-x^2+7$.
\end{question}
\end{tcolorbox}

\begin{solution}[name=Solution by Parviz Shahriari]
Answer: $(x^2+\sqrt{2\sqrt{7}+1}x + \sqrt{7})(x^2-\sqrt{2\sqrt{7}+1}x + \sqrt{7})$.
\end{solution}


\begin{tcolorbox}
\SetupExSheets{headings=runin}
\begin{question}
Factorize $x^4+4x-1$.
\end{question}
\end{tcolorbox}

\begin{solution}[name=Solution by Parviz Shahriari]
Answer: $(x^2+\sqrt{2}x + 1 - \sqrt{2})(x^2 - \sqrt{2}x+1 + \sqrt{2})$.
\end{solution}


\begin{tcolorbox}
\SetupExSheets{headings=runin}
\begin{question}
Factorize $(1+x+x^2+\cdots+x^n)^2-x^n$.
\end{question}
\end{tcolorbox}

\begin{solution}[name=Solution by Parviz Shahriari]
Answer: $(x^{n+1}+x^n+\cdots+x+1)(x^{n-1}+x^{n-2}+\cdots+x+1)$.
\end{solution}


\begin{tcolorbox}
\SetupExSheets{headings=runin}
\begin{question}
Factorize $2x^4+x^3+3x^2+x+2$.
\end{question}
\end{tcolorbox}

\begin{solution}[name=Solution by Parviz Shahriari]
Answer: $(x^2+x+1)(2x^2-x+2)$.
\end{solution}



\begin{tcolorbox}
\SetupExSheets{headings=runin}
\begin{question}
Factorize $x^6+2x^5+3x^4+24x^3+23x^2+22x+21$.
\end{question}
\end{tcolorbox}

\begin{solution}[name=Solution by Parviz Shahriari]
Begin with calculating the square root of the expression, equal to $x^3+x^2+x+11$ with a remainder of $-100$, then use difference of squares to arrive at $(x^3+x^2+x+1)(x^3+x^2+x+21)$. The final answer is $(x+1)(x+3)(x^2+1)(x^2-2x+7)$.
\end{solution}


\begin{tcolorbox}
\SetupExSheets{headings=runin}
\begin{question}
Factorize $x^4+6x^2+18$.
\end{question}
\end{tcolorbox}

\begin{solution}[name=Solution by Parviz Shahriari]
Answer: $(x^2+x\sqrt{6(\sqrt{2}-1)}+3\sqrt{2})(x^2-x\sqrt{6(\sqrt{2}-1)}+3\sqrt{2})$.
\end{solution}




\begin{tcolorbox}
\SetupExSheets{headings=runin}
\begin{question}
If $n$ is a positive integer, factorize $a^{5n}+a^n+1$.
\end{question}
\end{tcolorbox}

\begin{solution}[name=Solution by Parviz Shahriari]
Answer: $(a^{2n}+a^n+1)(a^{3n}-a^{2n}+1)$.
\end{solution}




\begin{tcolorbox}
\SetupExSheets{headings=runin}
\begin{question}
Factorize $(x+1)(x+3)(x+5)(x+7)+15$.
\end{question}
\end{tcolorbox}

\begin{solution}[name=Solution by Parviz Shahriari]
Answer: $(x+2)(x+6)(x+4+\sqrt{6})(x+4-\sqrt{6})$.
\end{solution}

\begin{tcolorbox}
\SetupExSheets{headings=runin}
\begin{question}
Factorize $x^3+5x^2+3x-9$.
\end{question}
\end{tcolorbox}

\begin{solution}[name=Solution by Parviz Shahriari]
Answer: $(x-1)(x+3)^2$.
\end{solution}


\begin{tcolorbox}
\SetupExSheets{headings=runin}
\begin{question}
Factorize $x^3+9x^2+11x-21$.
\end{question}
\end{tcolorbox}

\begin{solution}[name=Solution by Parviz Shahriari]
Answer: $(x-1)(x+3)(x+7)$.
\end{solution}


\begin{tcolorbox}
\SetupExSheets{headings=runin}
\begin{question}
Factorize $x^3(x^2-7)^2-36x$.
\end{question}
\end{tcolorbox}

\begin{solution}[name=Solution by Parviz Shahriari]
Answer: $x(x+1)(x-1)(x-3)(x+2)(x+3)(x-2)$.
\end{solution}

%%%% Equations in two variables
\subsection{Two--Variable Identities}



\begin{tcolorbox}
\SetupExSheets{headings=runin}
\begin{question}[name=(Positive Double--Variable Identity)]
Factorize $x^2 + 2xy + y^2$.
\end{question}
\end{tcolorbox}

\begin{solution}[name=Solution by Amir Parvardi]
This can be represented as $x^2+xy+xy+y^2$, and by factoring $x$ from the first two terms and also factoring $y$ from the last two terms, we get $x(x+y)+(x+y)y$. This last expression would be the same as $x(x+y)+y(x+y)$ in our commutative algebra, and factoring $(x+y)$ results in the First Double--Variable Identity: $$x^2+2xy+y^2=(x+y)^2.$$
\end{solution}




\begin{tcolorbox}
\SetupExSheets{headings=runin}
\begin{question}[name=(Negative Double--Variable Identity)]
Factorize $x^2 - 2xy + y^2$.
\end{question}
\end{tcolorbox}

\begin{solution}[name=Solution by Amir Parvardi]
This can be represented as $x^2-xy-xy+y^2$, and by factoring $x$ from the first two terms and also factoring $y$ from the last two terms, we get $x(x-y)-(x-y)y$. This last expression would be the same as $x(x-y)-y(x-y)$ in our commutative algebra, and factoring $(x-y)$ results in the Second Double--Variable Identity: $$x^2-2xy+y^2=(x-y)^2.$$
\end{solution}




\begin{tcolorbox}
\SetupExSheets{headings=runin}
\begin{question}[name=(Difference of Squares Identity)]
Factorize $x^2 - y^2$.
\end{question}
\end{tcolorbox}


\begin{solution}[name=Solution by Amir Parvardi]
Add and subtract $xy$ to find $x^2+xy-xy-y^2$, factor $x$ from the first two terms and $y$ from the other two: $x(x+y)-(x+y)y$. Thus, using commutativity once again as in the Positive and Negative Double--Variable Identities, we arrive at the identity for the difference between two squares: $$x^2-y^2=(x-y)(x+y).$$
\end{solution}

\begin{tcolorbox}
\SetupExSheets{headings=runin}
\begin{question}[name=($n^{\text{th}}$ Positive Double--Variable Identity)]
Factorize $x^n+y^n$ for odd $n$.
\end{question}
\end{tcolorbox}

\begin{solution}[name=Solution by Amir Parvardi]
It is necessary for $n$ to be odd for $x^n+y^n$ to be factorizable in real numbers, hence the $n^{th}$ Positive Double--Variable Identity is only defined for odd values of $n$. Since $n$ is odd, plugging $x=-y$ in $x^n+y^n$ results in zero, meaning that $(x+y)$ is a factor of $x^n+y^n$. Dividing $x^n+y^n$ by $x+y$ can be done smoothly by choosing the appropriate quotient with alternating positive and negative terms that cancel each other perfectly without leaving any remainder:
$$\frac{x^n+y^n}{x+y} = x^{n-1}-x^{n-2}y+\cdots-xy^{n-2}+y^{n-1}.$$
Again, the above arrangements are possible only for odd $n$. To finish up, this is the $n^{th}$ Positive Double--Variable Identity:
$$x^n+y^n = (x+y)\left(x^{n-1}-x^{n-2}y+\cdots-xy^{n-2}+y^{n-1}\right).$$
\end{solution}




\begin{tcolorbox}
\SetupExSheets{headings=runin}
\begin{question}[name=($n^{\text{th}}$ Negative Double--Variable Identity)]
Factorize $x^n-y^n$ for all $n$.
\end{question}
\end{tcolorbox}

\begin{solution}[name=Solution by Amir Parvardi]
Since $x=y$ yields $x^n-y^n=0$, we know that $(x-y)$ is a factor of $x^n-y^n$. We can easily see that the quotient of the division of $x^n-y^n$ by $x-y$ contains only positive terms:
$$\frac{x^n-y^n}{x-y} = x^{n-1}+x^{n-2}y+\cdots+xy^{n-2}+y^{n-1}.$$
Since all the terms are positive, there would be no problem of matching alternating positive and negative terms that cancel each other, and the $n^{th}$ Negative Double--Variable Identity holds for all positive integers $n$:
$$x^n-y^n = (x-y)\left(x^{n-1}+x^{n-2}y+\cdots+xy^{n-2}+y^{n-1}\right).$$
\end{solution}



\begin{tcolorbox}
\SetupExSheets{headings=runin}
\begin{question}[name=(${2^k}^{th}$ Negative Double--Variable Identity)]
Factorize $x^{2^k}-y^{2^k}$ for all $k$.
\end{question}
\end{tcolorbox}

\begin{solution}[name=Solution by Amir Parvardi]
Since $x=y$ yields $x^{2^k}-y^{2^k}=0$, we know that $(x-y)$ is a factor of $x^{2^k}-y^{2^k}$. We also get the quotient as in the $n^{th}$ Negative Double--Variable Identity:
$$\frac{x^{2^k}-y^{2^k}}{x-y} = x^{2^k-1}+x^{2^k-2}y+\cdots+xy^{2^k-2}+y^{2^k-1}.$$
We see that the degree of $x$ in the quotient is $2^k-1$, which happens to be equal to $1+2+2^2+\cdots+2^{k-1}$, meaning that the leading term in the quotient, $x^{2^k-1}$, is in fact a product of $k$ terms $x \cdot x^2 \cdot x^{2^2} \cdots x^{2^{k-1}}$, and there must be an identity in this form:
$$ x^{2^k-1}+x^{2^k-2}y+\cdots+xy^{2^k-2}+y^{2^k-1} = (x+\dots)(x^2+\dots)(x^{2^2}+\dots)\cdots (x^{2^{k-1}}+\dots),$$
and the same technique could be applied on $y$ since everything is symmetric, and the missing terms are easily found:
$$\frac{x^{2^k}-y^{2^k}}{x-y} =  (x+y)(x^2+y^2)(x^{2^2}+y^{2^2})\cdots (x^{2^{k-1}}+x^{2^{k-1}}),$$
giving us the magical ${2^k}^{th}$ Negative Double--Variable Identity:
$$x^{2^k}-y^{2^k} = (x-y)(x+y)(x^2+y^2)(x^{2^2}+y^{2^2})\cdots (x^{2^{k-1}}+y^{2^{k-1}}).$$
\end{solution}


\begin{tcolorbox}
\SetupExSheets{headings=runin}
\begin{question}[name=(Sophie Germain Identity)]
Factorize $x^4+4y^4$.
\end{question}
\end{tcolorbox}

\begin{solution}[name=Solution by Sophie Germain]
Adding $4x^2y^2$ to $x^4+4y^4$ completes the square to $(x^2+2y^2)^2$. Now just subtract the added term $4x^2y^2$ from $(x^2+2y^2)^2$ and use the Difference of Squares Identity to finish:
$$x^4+4y^4 = (x^2+2y^2+2xy)(x^2+2y^2-2xy).$$
\end{solution}


\begin{tcolorbox}
\SetupExSheets{headings=runin}
\begin{question}[name=(Sophie Parker Identity)]
Factorize $x^4+x^2y^2+y^4$.
\end{question}
\end{tcolorbox}

\begin{solution}[name=Solution by Sophie Parker]

\begin{itemize}
    \item \textbf{Difference of Squares}: It is easy to see that adding $x^2y^2$ to the given expression completes the square, making it
    $(x^2+y^2)^2 = x^4+2x^2y^2+y^4$. The Difference of Squares Identity yields the final factorization:
    \begin{align*}
        x^4+x^2y^2+y^4 &= (x^2+y^2)^2 - (xy)^2\\ &=(x^2+y^2-xy)(x^2+y^2+xy).
    \end{align*}
    
    \item \textbf{Difference of Squares and Cubes}: What if we begin with $x^6-y^6$? If I apply The Difference of Squares on this expression, I would have on one hand $x^6-y^6=(x^3-y^3)(x^3+y^3)$, and on the other hand I can apply the $n^{th}$ Negative Double--Variable Identity for $n=3$ on $x^6-y^6= (x^2-y^2)(x^4+x^2y^2+y^4)$.
    \begin{align*}
        x^6 - y^6 &= (x^3-y^3)(x^3+y^3)\\
                  &= (x-y)(x^2+xy+y^2)\cdot (x+y)(x^2-xy+y^2)\\
        x^6 - y^6 &= (x^2-y^2)(x^4+x^2y^2+y^4)\\
                  &= (x-y)(x+y)(x^4+x^2y^2+y^4).
    \end{align*}
    Therefore,
    \begin{align*}
        (x-y)(x^2+xy+y^2)\cdot (x+y)(x^2-xy+y^2) &= (x-y)(x+y)(x^4+x^2y^2+y^4).
    \end{align*}
    Assuming $x\neq \pm y$, we can cancel the terms $x-y$ and $x+y$ from both sides of the equation and obtain the factorization of $x^4+x^2y^2+y^4$ as a consequence:
    \begin{align*}
        (x^2+xy+y^2)\cdot(x^2-xy+y^2) &= x^4+x^2y^2+y^4.
    \end{align*}
\end{itemize}
\end{solution}


\begin{tcolorbox}
\SetupExSheets{headings=runin}
\begin{question}
Factorize $x^4+y^4+(x+y)^4$.
\end{question}
\end{tcolorbox}

\begin{solution}
Answer: $2(x^2+xy+y^2)^2$.
\end{solution}


\begin{tcolorbox}
\SetupExSheets{headings=runin}
\begin{question}
Factorize $x^4+y^4+(x-y)^4$.
\end{question}
\end{tcolorbox}

\begin{solution}
Answer: $2(x^2-xy+y^2)^2$.
\end{solution}



\begin{tcolorbox}
\SetupExSheets{headings=runin}
\begin{question}
Factorize $(x+y)^3 - x^3 - y^3$.
\end{question}
\end{tcolorbox}

\begin{solution}
Answer: $3xy(x+y)$.
\end{solution}


\begin{tcolorbox}
\SetupExSheets{headings=runin}
\begin{question}
Factorize $(x+y)^5 - x^5 - y^5$.
\end{question}
\end{tcolorbox}

\begin{solution}
Answer: $5xy(x+y)(x^2+xy+y^2)$.
\end{solution}


\begin{tcolorbox}
\SetupExSheets{headings=runin}
\begin{question}
Factorize $(x+y)^7 - x^7 - y^7$.
\end{question}
\end{tcolorbox}

\begin{solution}
Answer: $7xy(x+y)(x^2+xy+y^2)^2$.
\end{solution}


\begin{tcolorbox}
\SetupExSheets{headings=runin}
\begin{question}
Show that $(x+y)^n - x^n - y^n$ always has a factor of \[nxy(x+y)(x^2+xy+y^2)^2,\] if $n=6k+1$ for some integer $k\geq 1$.
\end{question}
\end{tcolorbox}

\begin{solution}
Answer: 
\end{solution}



\begin{tcolorbox}
\SetupExSheets{headings=runin}
\begin{question}
Factorize $4(x^2+xy+y^2)^3 - 27x^2y^2(x+y)^2$.
\end{question}
\end{tcolorbox}

\begin{solution}[name=Solution by Parviz Shahriari]
The expression is symmetric with respect to $x$ and $y$, and it becomes zero by plugging $x=y$ and $x=-2y$, so the factorization is $[(x+2y)(2x+y)(x-y)]^2$.
\end{solution}



%%%% Equations in three variables
\subsection{Three--Variable Identities}


\begin{tcolorbox}
\SetupExSheets{headings=runin}
\begin{question}
Factorize $x^2+y^2+z^2+2xy+2yz+2zx$.
\end{question}
\end{tcolorbox}

\begin{solution}
The answer is $(x+y+z)^2$.
\end{solution}

\begin{tcolorbox}
\SetupExSheets{headings=runin}
\begin{question}
Factorize $(xy^3+yz^3+zx^3) - (x^3y+y^3z+z^3x)$.
\end{question}
\end{tcolorbox}

\begin{solution}
The answer is $(x+y+z)(x-y)(y-z)(z-x)$.
\end{solution}

\begin{tcolorbox}
\SetupExSheets{headings=runin}
\begin{question}
Factorize $(x^2y^3+y^2z^3+z^2x^3) - (x^3y^2+y^3z^2+z^3x^2)$.
\end{question}
\end{tcolorbox}

\begin{solution}
The answer is $(xy+yz+zx)(x-y)(y-z)(z-x)$.
\end{solution}

\begin{tcolorbox}
\SetupExSheets{headings=runin}
\begin{question}
Factorize $x^3+y^3+z^3+(x+y)^3+(y+z)^3+(z+x)^3$.
\end{question}
\end{tcolorbox}

\begin{solution}
The answer is $3(x+y+z)(x^2+y^2+z^2)$.
\end{solution}


\begin{tcolorbox}
\SetupExSheets{headings=runin}
\begin{question}
Factorize $(x+y)(y+z)(z+x)+xyz$.
\end{question}
\end{tcolorbox}

\begin{solution}
The answer is $(x+y+z)(xy+yz+zx)$.
\end{solution}


\begin{tcolorbox}
\SetupExSheets{headings=runin}
\begin{question}
Factorize $xy(x+y)+yz(y-z)-xz(x+z)$.
\end{question}
\end{tcolorbox}

\begin{solution}[name=Solution by Parviz Shahriari]
Plug $y=z$ and observe that the result is zero, so $(y-z)$ is a factor. The answer is $(x+z)(y-z)(x+y)$.
\end{solution}

\begin{tcolorbox}
\SetupExSheets{headings=runin}
\begin{question}
Factorize $2a^2b+4ab^2-a^2c+ac^2-4b^2c+2bc^2-4abc$.
\end{question}
\end{tcolorbox}

\begin{solution}[name=Solution by Parviz Shahriari]
Answer: $(a+2b)(2b-c)(a-c)$.
\end{solution}


\begin{tcolorbox}
\SetupExSheets{headings=runin}
\begin{question}
Factorize $a^4+b^4+c^4 - 2a^2b^2 - 2a^2c^2 - 2b^2c^2$.
\end{question}
\end{tcolorbox}

\begin{solution}[name=Solution by Parviz Shahriari]
Answer: $(a+b+c)(a-b-c)(a+b-c)(a-b+c)$.
\end{solution}


\begin{tcolorbox}
\SetupExSheets{headings=runin}
\begin{question}
Factorize $x^2y^2z^2 + (x^2+yz)(y^2+zx)(z^2+xy)$.
\end{question}
\end{tcolorbox}

\begin{solution}
Answer: $(xy^2+yz^2+zx^2)(x^2y+y^2x+z^2x)$.
\end{solution}



\begin{tcolorbox}
\SetupExSheets{headings=runin}
\begin{question}
Factorize $y(x-2z)^2 + 8xyz + x(y-2z)^2 - 2z (x+y)^2$.
\end{question}
\end{tcolorbox}

\begin{solution}[name=Solution by Parviz Shahriari]
Answer: $(x-2y)(y-2z)(x+y)$.
\end{solution}




\begin{tcolorbox}
\SetupExSheets{headings=runin}
\begin{question}
Factorize $(a+b+c)^3 - a^3 - b^3 - c^3$.
\end{question}
\end{tcolorbox}

\begin{solution}[name=Solution by Parviz Shahriari]
Answer: $3(a+b)(b+c)(c+a)$.
\end{solution}




\begin{tcolorbox}
\SetupExSheets{headings=runin}
\begin{question}
Factorize $(ab+bc+ca)(a+b+c)-abc$.
\end{question}
\end{tcolorbox}

\begin{solution}[name=Solution by Parviz Shahriari]
Answer: $(a+b)(b+c)(c+a)$.
\end{solution}


\begin{tcolorbox}
\SetupExSheets{headings=runin}
\begin{question}
Factorize $(xy^2+yz^2+zx^2)-(x^2y+y^2z+z^2x)$.
\end{question}
\end{tcolorbox}

\begin{solution}
Answer: $(x-y)(y-z)(z-x)$.
\end{solution}

\begin{tcolorbox}
\SetupExSheets{headings=runin}
\begin{question}
Factorize $(x-y)^3+(y-z)^3+(z-x)^3$.
\end{question}
\end{tcolorbox}

\begin{solution}
Answer: $3(x-y)(y-z)(z-x)$.
\end{solution}


\begin{tcolorbox}
\SetupExSheets{headings=runin}
\begin{question}
Define $g(x,y,z)=x^2+y^2+z^2-xy-yz-zx$. Show that \[(x-y)^2+(y-z)^2+(z-x)^2,\] is divisible by $g(x,y,z)$ and that \[(x-y)^4+(y-z)^4+(z-x)^4,\] is divisible by $\left(g(x,y,z)\right)^2$.
\end{question}
\end{tcolorbox}

\begin{solution}
Expanding $(x-y)^2+(y-z)^2+(z-x)^2$ results in $2(x^2+y^2+z^2-xy-yz-zx)$. Moreover,
\begin{align*}
    (x-y)^4+(y-z)^4+(z-x)^4 = 2(x^2+y^2+z^2-xy-yz-zx)^2.
\end{align*}
\end{solution}




\begin{tcolorbox}
\SetupExSheets{headings=runin}
\begin{question}
Factorize $(x-y)^5 + (y-z)^5 + (z-x)^5$.
\end{question}
\end{tcolorbox}

\begin{solution}[name=Solution by Parviz Shahriari]
Answer: $5(x-y)(y-z)(z-x)(x^2+y^2+z^2-xy-yz-zx)$.
\end{solution}


\begin{tcolorbox}
\SetupExSheets{headings=runin}
\begin{question}
Factorize $(x-y)^7 + (y-z)^7 + (z-x)^7$.
\end{question}
\end{tcolorbox}

\begin{solution}[name=Solution by Parviz Shahriari]
Answer: $7(x-y)(y-z)(z-x)(x^2+y^2+z^2-xy-yz-zx)^2$.
\end{solution}


\begin{tcolorbox}
\SetupExSheets{headings=runin}
\begin{question}
Factorize $(x^2+y^2+z^2)^2 - 2(x^4+y^4+z^4)$.
\end{question}
\end{tcolorbox}

\begin{solution}[name=Solution by Parviz Shahriari]
Answer: $(x+y+z)(-x+y+z)(x-y+z)(x+y-z)$.
\end{solution}



\begin{tcolorbox}
\SetupExSheets{headings=runin}
\begin{question}
Factorize $x^3+y^3+z^3-3xyz$.
\end{question}
\end{tcolorbox}

\begin{solution}[name=Solution by Parviz Shahriari]
Answer: $(x+y+z)(x^2+y^2+z^2-xy-yz-zx)$.
\end{solution}


\begin{tcolorbox}
\SetupExSheets{headings=runin}
\begin{question}
Factorize \[(a^2-bc)^3+(b^2-ac)^3+(c^2-ab)^3-3(a^2-bc)(b^2-ac)(c^2-ab).\]
\end{question}
\end{tcolorbox}

\begin{solution}[name=Solution by Parviz Shahriari]
Answer: $(a+b+c)^2(a^2+b^2+c^2-ab-bc-ca)^2$.
\end{solution}



\begin{tcolorbox}
\SetupExSheets{headings=runin}
\begin{question}
Factorize $(x+y+z)^5-x^5-y^5-z^5$.
\end{question}
\end{tcolorbox}

\begin{solution}[name=Solution by Parviz Shahriari]
Answer: $5(x+y)(y+z)(z+x)(x^2+y^2+z^2+xy+yz+zx)$.
\end{solution}




\begin{tcolorbox}
\SetupExSheets{headings=runin}
\begin{question}
Factorize $a^3(x-y) + x^3(a-y) + y^3(a-x)$.
\end{question}
\end{tcolorbox}

\begin{solution}[name=Solution by Parviz Shahriari]
Answer: $(x-y)(a-x)(a-y)(x+y+a)$.
\end{solution}


\begin{tcolorbox}
\SetupExSheets{headings=runin}
\begin{question}
Factorize $a^2b^2(b-a) + b^2c^2(c-b) + c^2a^2(a-c)$.
\end{question}
\end{tcolorbox}

\begin{solution}[name=Solution by Parviz Shahriari]
Answer: $(a-b)(b-c)(c-a)(ab+bc+ca)$.
\end{solution}


\begin{tcolorbox}
\SetupExSheets{headings=runin}
\begin{question}
Factorize $8x^3(y+z)-y^3(z+2x)-z^3(2x-y)$.
\end{question}
\end{tcolorbox}

\begin{solution}[name=Solution by Parviz Shahriari]
Answer: $(y+z)(2x-y)(2x+z)(2x+y-z)$.
\end{solution}


\begin{tcolorbox}
\SetupExSheets{headings=runin}
\begin{question}
Factorize $x^2y+xy^2+x^2z+xz^2+y^2z+yz^2+2xyz$.
\end{question}
\end{tcolorbox}

\begin{solution}[name=Solution by Parviz Shahriari]
Answer: $(x+y)(y+z)(z+x)$.
\end{solution}


\begin{tcolorbox}
\SetupExSheets{headings=runin}
\begin{question}
Factorize $x^2y+xy^2+x^2z+xz^2+y^2z+yz^2+3xyz$.
\end{question}
\end{tcolorbox}

\begin{solution}[name=Solution by Parviz Shahriari]
Answer: $(x+y+z)(xy+yz+zx)$.
\end{solution}

\begin{tcolorbox}
\SetupExSheets{headings=runin}
\begin{question}
Factorize $(x^2+y^2)^3+(z^2-x^2)^3-(y^2+z^2)^3$.
\end{question}
\end{tcolorbox}

\begin{solution}[name=Solution by Parviz Shahriari]
Answer: $3(y^2+z^2)(x^2+y^2)(x-z)(x+z)$.
\end{solution}


\begin{tcolorbox}
\SetupExSheets{headings=runin}
\begin{question}
Factorize $x^3(y-z) + y^3(z-x) + z^3(x-y)$.
\end{question}
\end{tcolorbox}

\begin{solution}[name=Solution by Parviz Shahriari]
Answer: $-(x+y+z)(x-y)(y-z)(z-x)$.
\end{solution}


\begin{tcolorbox}
\SetupExSheets{headings=runin}
\begin{question}
Factorize $x^3(z-y^2)+y^3(x-z^2)+z^3(y-x^2)+xyz(xyz-1)$.
\end{question}
\end{tcolorbox}

\begin{solution}[name=Solution by Parviz Shahriari]
Answer: $(x^2-y)(y^2-z)(z^2-x)$.
\end{solution}


\begin{question}[name={2006 Korea}]
    Find the number of positive integer triples $(a,b,c)$ such that
    \[\frac{a^2+b^2-c^2}{ab}+\frac{b^2+c^2-a^2}{bc}+\frac{c^2+a^2-b^2}{ca}=2+\frac{15}{abc}.\]
\end{question}


%%%% Identities with more than three variables
% \subsection{High--Variable Identities}

\begin{question}
Factorize \[[(x^2+y^2)(a^2+b^2)+4abxy]^2 - 4[xy(a^2+b^2)+ab(x^2+y^2)]^2.\]
\end{question}

\begin{solution}[name=Solution by Parviz Shahriari]
The term in the first bracket simplifies to $(ax+by)^2+(ay+bx)^2$, and the term in the second bracket is $(ay+bx)(ax+by)$. Use the difference of squares to see that the given expression factorizes into $(a-b)^2(a+b)^2(x-y)^2(x+y)^2$. 
\end{solution}



